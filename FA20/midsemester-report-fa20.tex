\documentclass[10pt,usenames,dvipsnames]{article}

\usepackage{amsmath}
\usepackage{amssymb}
\usepackage{amsthm}
\usepackage{bm}
\usepackage{bbm}
\usepackage{mathtools}
\usepackage{enumerate}
\usepackage[margin=1.25in]{geometry}
\usepackage[T1]{fontenc}
\usepackage{kpfonts}
\usepackage{subfigure}
\usepackage{listings}
\usepackage{xcolor}

%\newcommand{\reed}[1]{\relax}
%\newcommand{\eric}[1]{\relax}
%\newcommand{\christian}[1]{\relax}
%\newcommand{\philipp}[1]{\relax}
%\newcommand{\Fix}[1]{\relax}
\newcommand{\reed}[1]{{\color{magenta}\bfseries [#1]}}
\newcommand{\eric}[1]{{\color{green}\bfseries [#1]}}
\newcommand{\christian}[1]{{\color{orange}\bfseries [#1]}}
\newcommand{\philipp}[1]{{\color{blue}\bfseries [#1]}}
\newcommand{\Fix}[1]{{\color{red}\bfseries [#1]}}
\newcommand{\Comment}[1]{}
\newcommand{\Space}[1]{}
\newcommand{\Num}[1]{#1}

\newcommand{\term}[1]{\emph{#1}}

\newcommand{\Free}{\textbf{Free}}

\newcommand{\evaluates}{\Downarrow}

\newcommand{\R}{\mathbb{R}}
\newcommand{\Q}{\mathbb{Q}}
\newcommand{\Z}{\mathbb{Z}}
\newcommand{\N}{\mathbb{N}}

\newcommand{\join}{\wedge}
\newcommand{\meet}{\vee}
\newcommand{\proves}{\vdash}

\newcommand{\dom}{\text{dom}~}

\newcommand{\proj}{\text{proj}}

\newcommand{\implicits}{\texttt{implicit}}
\newcommand{\params}{\texttt{params}}
\newcommand{\nonvar}{\texttt{nonvar}}

\newcommand{\tand}{\ensuremath{~\text{and}~}}
\newcommand{\tor}{\ensuremath{~\text{or}~}}
\newcommand{\twhere}{\ensuremath{~\text{where}~}}
\newcommand{\tif}{\ensuremath{~\text{if}~}}
\newcommand{\tsuchthat}{\ensuremath{~\text{s.t.}~}}
\newcommand{\owise}{\ensuremath{~\text{otherwise}~}}

\newcommand{\brackets}[3]{\ensuremath{{\left#1 {#3} \right#2}}}
\newcommand{\parens}[1]{\brackets{(}{)}{#1}}
\newcommand{\angles}[1]{\brackets{<}{>}{#1}}
\newcommand{\curlys}[1]{\brackets{\{}{\}}{#1}}
\newcommand{\squares}[1]{\brackets{[}{]}{#1}}

\newcommand{\bnfdef}{\ensuremath{\Coloneqq}}
\newcommand{\bnfalt}{\ensuremath{\mid}\xspace}

\newcommand{\inferred}{\texttt{Inferred}}
\newcommand{\prop}[1]{#1 ~ \text{prop}}
\newcommand{\typ}[1]{\text{typ}\parens{#1}}

\theoremstyle{definition}
\newtheorem{definition}{Definition}[section]
\theoremstyle{remark}
\newtheorem{remark}[definition]{Remark}
\theoremstyle{remark}
\newtheorem{example}[definition]{Example}
\theoremstyle{plain}
\newtheorem{theorem}[definition]{Theorem}
\newtheorem{conjecture}[definition]{Conjecture}

\lstdefinelanguage{pecan}{
	keywords=[1]{forall, exists, are, is},
	keywordstyle=[1]\color{blue}\bfseries,
	keywords=[2]{false, true},
	keywordstyle=[2]\color{orange}\bfseries,
	keywords=[3]{assert_prop,type},
	keywordstyle=[3]\color{teal}\bfseries,
	literate=%
	    {\#}{{{\color{teal}\bfseries\#}}}1
	    {+}{{{\color{red}+}}}1
        {:=}{{{\color{red}:=}}}1
        {..}{{{\color{red}..}}}1
        {*}{{{\color{red}*}}}1
        {:}{{{\color{red}:}}}1
        {>}{{{\color{red}>}}}1
        {<}{{{\color{red}<}}}1
        {=}{{{\color{red}=}}}1,
    escapeinside={\&}{)},
    sensitive=false, % keywords are not case-sensitive
    morecomment=[l]{//}, % l is for line comment
    morecomment=[s]{/*}{*/}, % s is for start and end delimiter
    morestring=[b]" % defines that strings are enclosed in double quotes
}


\usepackage[
backend=biber,
style=numeric,
sorting=ynt
]{biblatex}

\addbibresource{biblio.bib}

\title{Pecan: A Theorem Prover}

\author{%
IGL Scholars: Zhengyao Lin, Eric Ma, Reed Oei, Yikai Teng, Pavle Vuksanovic \\
Graduate Mentor: Christian Schulz, Mary-Angelica Tursi \\
Faculty Advisor: Philipp Hieronymi%
}

\begin{document}

\maketitle

\reed{This is old, but a lot of it can be reused}

\section{Introduction}

An \textbf{automated theorem prover} is a program that takes as input a statement and \emph{decides} (i.e., proves or disproves) it. 
Theorem provers can be very useful: computers are reliable, and they never get tired or bored, allowing us to quickly explore new ideas.
Though it is impossible to decide \emph{all} statements, we can still use theorem provers to solve many interesting problems. 

\section{Background}

A \textbf{finite automaton} is a ``machine'' that accepts some \textbf{finite word}, or string of letters, such as ``abc.''
They can be represented as a \emph{finite} collection of \textbf{states} and \textbf{transitions}.

Automata have nice closure properties: we can combine them using first-order logic operations.
Therefore, there is an algorithm that can decide any statement expressed solely in terms of first order logic operations and properties defined by finite automata.

\begin{theorem}{\cite{aut_theory}}
    Suppose we have predicates $P$ and $Q$, such that automaton $\mathcal{A}$ accepts words that makes $P$ true, and $\mathcal{B}$ accepts words that makes $Q$ true, then we can construct, from $\mathcal{A}$ and $\mathcal{B}$, the automata that accepts words that makes $P \land Q$, $P \lor Q$, $\lnot P$ true.
\end{theorem}

For the quantifiers, $\forall$ and $\exists$, we need to define automata that accepts multiple inputs. 
\begin{definition}
    An automaton $\mathcal{A}$ accepts a pair of words $ (x_1x_2\dots x_n, y_1y_2\dots y_n)$ if it accepts the word $(x_1,y_1)(x_2,y_2)\dots(x_n,y_n)$.
\end{definition}

\begin{theorem}\cite{aut_theory}
    Suppose $p(x,y)$ is a predicate with two inputs, and automaton $\mathcal{A}$ accepts pairs of words that makes $p$ true, then we can construct an automaton, from $\mathcal{A}$, which accepts words that makes $p'(y) = \exists x. p(x,y)$ true.
\end{theorem}

Therefore, we can combine predicates using first-order quantifiers and construct the automaton associated with the predicate.

\textbf{B\"uchi Automata} are an extension of finite automata to infinite inputs; they accept words that visit an accepting state infinitely often \cite{aut_theory}. 
Using B\"uchi automata has a number of advantages:
\begin{itemize}
        \item Some problems are more naturally expressed with B\"uchi automata
        \item Some problems can \emph{only} be expressed with B\"uchi automata
        \item We can still express properties about finite strings
\end{itemize}
A major application of B\"uchi automata is in model checking: verifying that programs have the expected behavior.

\section{Pecan}

\textbf{Pecan} is a system for automated theorem proving that represents logical predicates using B\"uchi automata.
Pecan programs are made up of \textbf{predicates} and \textbf{directives}:

\begin{itemize}
    \item predicates: defined either by loading automata defined in files, translating LTL (linear temporal logic) formulas into B\"uchi automata, or some combination of these using first order logic and equality.

\begin{lstlisting}[language=pecan, basicstyle=\normalsize\ttfamily, mathescape=true, frame=single]
eq_sym() := $\forall$x. $\forall$y. x =$\ $y $\color{red} \iff$ y =$\ $x
\end{lstlisting}

    \item directives: commands to the Pecan interpreter, such as: \texttt{assert\_prop}, which asks Pecan to prove a theorem
    
\begin{lstlisting}[language=pecan, basicstyle=\normalsize\ttfamily, mathescape=true, frame=single]
#assert_prop(true, eq_sym)
\end{lstlisting}

\end{itemize}

The Pecan language is a statically-typed, interpreted programming language.
The basic datatype, or universal set, in Pecan is the infinite binary word---elements of $\Sigma^{\omega}$ where $\Sigma = \{0,1\}$.
All types are \textbf{refinement types}: they are defined as a subset of $\Sigma^{\omega}$ satisfying some predicate defined in terms of B\"uchi automata.

\begin{example}
A word $x \in \Sigma^\omega$ has type \texttt{binary} if $x \in \{ w \in \Sigma^* : x = w0^\omega \}$---that is, binary words that are eventually $0$.
Equivalently, we can define $\texttt{binary}$ as the words satisfying the LTL formula $F(G(\neg x))$.
\end{example}

Pecan is capable of proving any statement expressed solely in terms of B\"uchi automata and first order logic connectives, such as $\wedge$, $\vee$, $\neg$, $\forall$, and $\exists$.
In particular, it is powerful enough to decide any statement in Presburger arithmetic (and its natural extension to $\mathbb{Z}$ or $\mathbb{R}$), the first order theory of the natural numbers with addition (but \emph{not} multiplication) and comparison ($<$).

\section{Results}

Our main result is the creation of Pecan; this section serves as a demonstration of some of the theorems that Pecan is capable of proving.

\subsection*{Example: Properties of the Thue-Morse Word}

Below, we develop a Pecan program that proves two well-known properties of the Thue-Morse word, $T$, which is defined by: the $n$-th digit of the Thue-Morse word, $T[n]$, is $1$ if the binary representation of $n$ has an odd number of $1$'s, and $0$ otherwise \cite{auto_seq}.

The Thue-Morse word starts with: $1101001100101101001011001101001\ldots$
\begin{definition}
    A word $w$ is a \textbf{square} if it is of the form $w = xx$ for some nonempty word $x$.
    Similarly, $w$ is a \textbf{cube} if it is of the form $w = xxx$ for some nonempty word $x$.
\end{definition}

\begin{theorem}\cite{auto_seq}
    The Thue-Morse word contains squares, but does not contain any cubes.
\end{theorem}

Here is the equivalent definition of $T$ in Pecan (\texttt{odd\_ones} is defined as an automaton):
\begin{lstlisting}[language=pecan, basicstyle=\normalsize\ttfamily, mathescape=true, frame=single]
T(x) := odd_ones(x)
\end{lstlisting}

Below are definitions of square and cube for the Thue-Morse word in Pecan.

\begin{lstlisting}[language=pecan, basicstyle=\normalsize\ttfamily, mathescape=true, frame=single]
i,j,n $\color{red} \in$ binary
square(i, n) := n > 0 $\color{red} \wedge$ T[i $\color{red} \ldots$ i+n] =$\ $T[i+n $\color{red} \ldots$ i+2*n]
cube(i, n) := square(i, n) $\color{red} \wedge$ square(i+n, n)
\end{lstlisting}


Finally, we state the theorems we would like to prove, and ask Pecan to attempt to prove that there are squares, but there are no cubes.

\begin{lstlisting}[language=pecan, basicstyle=\normalsize\ttfamily, mathescape=true, frame=single]
squares_exist() := $\exists$i. $\exists$n. square(i, n)
#assert_prop(true, squares_exist)
cubes_exist() := $\exists$i. $\exists$n. cube(i, n)
#assert_prop(false, cubes_exist)
\end{lstlisting}

The output from Pecan is the following, indicating that the theorem is true.
\begin{lstlisting}[basicstyle=\normalsize\ttfamily, mathescape=true, frame=single]
[INFO] Checking if squares_exist is true.
$\text{\color{ForestGreen}squares\_exist is true.}$
[INFO] Checking if cubes_exist is false.
$\text{\color{ForestGreen}cubes\_exist is false.}$
\end{lstlisting}

Here is another theorem about the Thue-Morse word that we can check.
\begin{theorem}\cite{auto_seq}
There are no overlapping squares, i.e. words of form $0x0x0$ or $1x1x1$ for some nonempty word $x$. 
\end{theorem}

We can express this theorem in Pecan as:

\begin{lstlisting}[language=pecan, basicstyle=\normalsize\ttfamily, mathescape=true, frame=single]
i,j,n $\color{red} \in$ binary
o_square(i, n) := n > 0 $\color{red} \wedge$ 
                  T[i+1 $\color{red} \ldots$ i+n+1] = $\ $T[2+i+n $\color{red} \ldots$ 2+i+2*n] $\color{red} \wedge$  
                  T[i] =$\ $T[i+n] $\color{red} \wedge$ T[i] =$\ $T[2+i+2*n]
o_squares_exist() := $\exists$i. $\exists$n. o_square(i, n)
#assert_prop(false, o_squares_exist)
\end{lstlisting}

Pecan verifies the theorem: 

\begin{lstlisting}[basicstyle=\normalsize\ttfamily, mathescape=true, frame=single]
[INFO] Checking if o_squares_exist is false.
$\text{\color{ForestGreen} o\_squares\_exist is false.}$
\end{lstlisting}

\section{Future Work}

We plan to continue to develop Pecan and to write a manual for it. 
We proved various theorems about special classes of infinite binary sequences known as \textbf{Sturmian words}.
Using Pecan, we hope to automatically prove theorems about \emph{all} Sturmian words at once.
Finally, because Pecan uses B\"uchi automata, which accept infinite words, we can represent and prove theorems about real numbers.
This was not possible with other automated theorem provers, such as Walnut \cite{walnut}---Pecan opens many new possibilities to explore.

\printbibliography

\end{document}
