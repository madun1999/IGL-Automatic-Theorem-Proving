\documentclass[11pt]{article}
\usepackage{graphicx}
\usepackage{amssymb}
\usepackage{amsmath}
\usepackage{amsthm}
\usepackage{amsfonts}
\usepackage{bbm}
\usepackage{amsrefs}
\usepackage{tikz}
\usepackage{setspace,kantlipsum}
\usepackage[toc,page]{appendix}


\theoremstyle{definition}
\newtheorem{thm}{Theorem}
\newtheorem{prop}{Proposition}
\newtheorem*{defn}{Definition}
\newtheorem*{note}{Note}
\newtheorem{claim}{Claim}
\newtheorem{lemma}{Lemma}
\newtheorem{step}{Step}

\usepackage[margin=2cm]{geometry}

\title{Pecan: An Automated Theorem Prover}

\author{
IGL Scholars: Zhengyao Lin, Eric Ma, Reed Oei, Yikai Teng, Pavle Vuksanovic \\
Graduate Mentor: Christian Schulz, Mary-Angelica Tursi \\
Faculty Advisor: Philipp Hieronymi%
}


\begin{document}
% \begin{spacing}{1.5}

\maketitle

\begin{abstract}

Pecan is an automated theorem prover for reasoning about \emph{automatic sequences}, which are sequences that can be recognized by some (typically finite) automaton.
Automated theorem provers and automatic sequences have diverse applications: in computer science, they are commonly used for program verification; in mathematics, they have found uses in logic, number theory, and combinatorics. Pecan is capable of proving any statement expressed in terms of B\"uchi automata and first-order logic connectives.

We have used Pecan to prove many theorems about a special class of automatic sequences called \emph{Sturmian words}, and we are currently exploring extensions including deciding sentences
involving linear inequalities with integer and quadratic irrational coefficients, and
visualization of fractals defined by B\"uchi automata.

This work is done in the research project ``Pecan: An Automated Theorem Prover'' at the Illinois Geometry Lab in 2019-2020.

% We discuss Pecan, some of the theorems we've proven, as well as our approaches and results in constraint solving and fractal visualization.
% An \emph{automated theorem prover} is a program capable of proving or disproving mathematical statements with no human guidance.
% We can work with constraint problems involving not only rational numbers, but also \emph{quadratic irrationals}, We can also use Pecan to visualize fractals defined by B\"uchi automata.
\end{abstract}

% \end{spacing}
\end{document}
