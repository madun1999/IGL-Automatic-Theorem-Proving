\documentclass[12pt]{article}
\usepackage[T1]{fontenc}
\usepackage{xparse}
\usepackage{enumitem}
\usepackage[a4paper, total={6.5in, 8.5in}]{geometry}
\setlist[description]{
  font={\sffamily\bfseries},
  labelsep=5pt,
  labelwidth=0pt,
  leftmargin=0pt,
}
\usepackage{fancyhdr}
\pagestyle{fancy}
\fancyhf{}
\rhead{Pecan - An Automated Theorem Prover}
\lhead{Text transcript}
\rfoot{\thepage}


\begin{document}
\begin{description}
\item[0:00-0:02]\hfill \\
Hello! Our project is about automated theorem proving. 


\item[0:03-0:17] \hfill \\
An automated theorem prover is a program that takes a statement and automatically decides if it is true or false. Even though it's impossible to automatically decide all statements. We can still use theorem provers to solve many interesting problems. Pecan is a system that we're developing for automated theorem proving this semester and the past semester. 

\item[0:18-0:44]\hfill \\
Pecan programs are made up of predicates and directives. Predicates are essentially logical relationships. Here, we define when y is the successor of $x$ meaning that $y$ is basically $x+1$. Directives are commands to the Pecan interpreter itself that tell it to actually do something, such as the ``theorem directive'' which tells pecan to prove a theorem. Here, we tell pecan to prove that equality is symmetric, that means that for all $x$ and $y$, $x=y$ if and only if $y=x$. 

\item[0:45-0:59]\hfill \\
Pecan is capable of proving any statement that involves simple logical operations like this: ``and'', ``or'', ``not'', ``equality'', and quantification like ``for all'' and ``there exists''. You can also prove things that are expressed in terms of B\"uchi automata, which are another model of computation that works with infinite strings.


\item[1:00-1:15]\hfill \\
Okay. Now we'll talk about some examples of things that we can do with pecan. The first example is the Chicken McNuggets problem, which is ``what is the biggest number of chicken nuggets that you cannot order using only boxes of 6, 9, and 20.'' Let's switch now to the website where we can try it out. 

\item[1:16-1:44]\hfill \\
Here is code that will solve it. We first need to define what numbers we can buy. We can buy a number if it's a natural number, and there are $a$, $b$, and $c$ such that $n$ is $6a + 9b + 20c$. Those represent the numbers of boxes of 6, 9, and 20 that we buy. Then we define non-purchasable as just the negation of purchasable and define largest non-purchasable number as a non-purchasable number that is at least as large as every other non-purchasable number. 

\item[1:45-2:09]\hfill \\
Now, we can just say, ``Pecan, give us an example of the biggest non purchasable number.'' Because there's only one, it will just give us a unique number which is 43. That's the answer to the question. The other thing you can do is you can ask Pecan to tell you what automaton it generated. So, you can look at this automaton. If you decode the binary, you'll be able to see that this accepts number 43. Just as Pecan said! 


\item[2:10-2:28]\hfill \\
For the next example, we'll talk about the Thue-Morse word, which is an infinite binary sequence defined by the simple rule that the $n^{th}$ digit of the sequence is 1 if the binary representation of $n$ has an odd number of 1's in it and 0 otherwise. It starts with 01101 and so on. Once we defined it in Pecan will be able to talk about various repetitions within it.

\item[2:29-2:49]\hfill \\
One such repetition is called a square which is when a word is just the same thing repeated twice. Then we also have cubes which are the same thing but just repeated three times. If you look at the Thue-Morse word right here, you can see that 1010 is a square, so the Thue-Morse word does contain squares. But does it contain any cubes or any other sort of repetition? 

\item[2:50-3:10]\hfill \\
To answer this question, we will switch back to the website so we can try it out. On the screen, you can see the code for this problem. If we run this program and wait a second, it will tell us that, the Thue-Morse word does contain squares, it does not contain cubes, and it does not contain overlapping squares. So, the longest repetition you can have in the Thue-Morse word is a square. 


\item[3:11-3:29]\hfill \\
One of the main objects were interested in studying is called ``characteristics Sturmian words'', or just ``Sturmian words'' for short. If you imagine standing at the origin and then throwing out a ball at some angle $\alpha$, you'll get a sequence of times of the ball crosses the horizontal and vertical grid lines. This gives a binary sequence. If you associate 0 with the vertical grid-lines, and 1 with the horizontal grid-lines then you'll get a sequence for every line like this. So, if I pick this line, which has slope $\frac{1}{\phi}$ (one over golden ratio), I get the sequence $01001\dots$

\item[3:40-4:01]\hfill \\
One of the things about these sequences is that they are automatic sequences. Given $n$ in a special numeration system, we can easily calculate what the $n^{th}$ digit is. Once we have the automaton that let us easily calculate the $n^{th}$ digit of a Sturmian word we can ask questions about Sturmian words such as ``Are they eventually periodic?'' and ``Does a Sturmian word eventually become the same sequence repeated over and over again?'' On the screen you can see the code for these problems.

\item[4:02-4:19] \hfill \\
Using Pecan we can prove that that's not true. So, these automata that we get end up having hundreds of states, so it doesn't really make any sense to show them here and we won't, unlike in the chicken nugget example. But pecan can still prove these theorems relatively quickly. We can also prove other theorems about Sturmian words. 

\item[4:20-4:44]\hfill \\
The main point is that these examples highlight Pecan's ability to quantify over uncountable sets. Some earlier theorem provers, they're based on similar approaches but only using finite strings, can only prove things about countable sets of strings or numbers. This allows us to state and prove theorems about all Sturmian words at once instead of just one or two at a time. They also let us prove things about real numbers, which there are uncountably many of. 


\item[4:45-4:58]\hfill \\
In the future, we hope to continue to develop Pecan. We're currently writing a paper about Pecan and all the various things that were able to prove with it. And as I mentioned we can use Pecan to proof theorems about real numbers, so we're hoping to look into that in the future as well. That's all. Thank you!


\end{description}
\end{document}