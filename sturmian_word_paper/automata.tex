\section{$k$-bounded Sturmian Words}\label{sec:automata}

\subsection{Definitions}

Let $\alpha = [a_0, a_1, \ldots]$ be the continued fraction representation of some irrational number $\alpha$.
Define $\alpha_{\leq k}$, the set of \textit{$k$-bounded continued fractions}, by:

\begin{equation*}
    \alpha_{<k} := \{ \alpha : \alpha = [ a_0, a_1, \ldots] \text{ where } \forall a_i, 0 < a_i \leq k\}
\end{equation*}

A \textit{$k$-bounded Ostrowski numeration system} is an Ostrowski numeration system for some $\alpha \in \alpha_{\leq k}$. 
Let $\Sigma_k$ denote the alphabet of possible digits in a $k$-bounded Ostrowski numeration system, $\{0,\ldots,k\}$.
Let $N_{\alpha}$ denote the set of words which are valid representations in the Ostrowski-$\alpha$ numeration system.
For example, $\sqrt{2} \in \alpha_{\leq 2}$, because $\sqrt{2} = [1; \overline{2}]$.
However, $\sqrt{5} = [2; \overline{4}] \not\in \alpha_{\leq 2}$.

Note that $\left( \alpha_{<k} \right)_{k \in \mathbb{N}}$ is an increasing sequence---if a proposition holds for all $\alpha \in \alpha_{\leq k}$, then it also holds for all smaller $\alpha_{\leq j}$ for $j \leq k$.

\subsection{Automata Generation}

We are interested in creating the following automata which take any $k$-bounded continued fraction $\alpha$ as an input in continued fraction form.

\begin{enumerate}[(i)]
    \item $R_{\alpha}$, which recognizes strings $w$ such that $w$ is a valid Ostrowski-$\alpha$ representation of some natural number $n$.
    \item $<_{\alpha}$, which recognizes pairs $(a,b)_{\alpha}$ such that $a < b$.
    \item $=_{\alpha}$, which recognizes pairs $(a,b)_{\alpha}$ such that $a = b$.
    \item $+_{\alpha}$, which recognizes triples $(a,b,c)_{\alpha}$ such that $a +_{\alpha} b = c$.
\end{enumerate}

In Ostrowski numeration systems, $<_{\alpha}$ is simple to define for any $\alpha$, as we may simply compare the two \word{}s place-by-place.
Similarly, equality of two numbers in an Ostrowski-$\alpha$ numeration system is simply equality on the \letter{}s in the \word{}s.

Recognition can be implemented by following the definition of Ostrowski-$\alpha$ representation given in Section~\ref{sec:background}, which lends itself to an easy definition in terms of an automaton.
Note that when the expression $1$ is ambiguous, that is, when both $1_{\alpha} = 1$ and $10_{\alpha} = 1$, we choose $10_{\alpha} = 1$ as the unique representation of $1$, as it is the one that matches our definition of a valid Ostrowski-$\alpha$ representation.

Following \citeauthor{ht-ostrowski}, we define four automata which combine to form the bounded addition automaton.

The first of these automata, $\mathcal{A}_0$ takes the Ostrowski-$\alpha$ representations of two natural numbers, $a = x_j x_{j-1} \cdots x_0$ and $b = y_j y_{j-1} \cdots y_0$ and accepts the word $(x_j + y_j) (x_{j-1} + y_{j-1}) \cdots (x_0 + y_0)$.
Note this automaton does not depend on $\alpha$, and therefore does not need to be modified.

The next automaton, $\mathcal{A}_1$ takes this representation and ensures that every symbol is in the proper range, from $0$ to $a_k$, though there may be consecutive symbols which are not allowed in the final representation.
For example, in the Zeckendorf numeration system, that is, the Ostrowski numeration system with $\alpha = \phi$, suppose we add $100_{\phi} + 10_{\phi}$.
The first automaton gives us the new word $110_{\phi}$, and the second automaton leaves this word unchanged, as each value is in bounds---however, this is not a valid Zeckendorf representation as it has two consecutive ones.
The next two automata, $\mathcal{A}_2$ and $\mathcal{A}_3$, correct this issue in two passes: $\mathcal{A}_2$ with one right-to-left pass, and $\mathcal{A}_3$ with one left-to-right pass.

Below, we describe the modifications to the original automata to make each accept the continued fraction of $\alpha$, the irrational number defining the numeration system.
By letting the coefficients of $\alpha$ be between $1$ and $k$, we obtain the addition and recognition automata for $k$-bounded Ostrowski numeration system.
The automaton for the characteristic Sturmian word is essentially the same for any $\alpha$: that is, it only cares about the number of zeroes at the end of the Ostrowski-$\alpha$ representation of $n$ to find the $n$-th digit of $C_{\alpha}$.
Using this, we may define a single automaton for every $\alpha \in \alpha_{<k}$, and we write $C_{\alpha}[n]$ to denote it's $n$-th digit.
This allows us to index into any $k$-bounded Sturmian word, and therefore, to automatically decide theorems about such words.

As noted above, $\mathcal{A}_0$ needs no modifications, so we begin with $\mathcal{A}_1 = (S_1, I_1, T_1, F_1)$.
The set of states, $S_1$, is given by:
\begin{equation*}\label{def:alg1-states}
    S_1 := \Sigma_k^3 \times \Sigma_{2k+1}^3 \times \Sigma_k^3 \times \{0,1\}
\end{equation*}

The initial state, $I_1$ is $((0,0,0),(0,0,0),(0,0,0),0)$.

The transition function is given below, and comes from the rules A1-A3 defined in Algorithm 1~\cite{ht-ostrowski}.
The domain is the subset of the tuples $S_1 \times \{1,\ldots,k\} \times \Sigma_{2k+1} \times \Sigma_k$ that satisfy one of the following rules.
Below, let $s = ((u_1,u_2,u_3),(v_1,v_2,v_3),(w_1,w_2,w_3),g)$ and $v_4 = v_4' + g$.

\begin{itemize}
    \item[(A1)] If $w_1 = v_1$, $v_2 < u_2$, $v_3 > u_3$, and $v_4 = 0$, then 
    
    $$T_1(s,u_4,v_4',w_4) := ((u_2,u_3,u_4),(v_2 + 1, v_3-(u_3+1),u_4-1),(w_2,w_3,w_4),1)$$
    
    \item[(A2)] If $w_1 = v_1$, $v_2 < u_2$, $u_3 \leq v_3 \leq 2u_3$, and $v_4 > 0$, then
    
    $$T_1(s,u_4,v_4',w_4) := ((u_2,u_3,u_4),(v_2+1,v_3-u_3,v_4-1),(w_2,w_3,w_4),0)$$
    
    \item[(A3)] If $v_1 = w_1$ and $v_2 = w_2$, then
    
    $$T_1(s,u_4,v_4',w_4) := ((u_2,u_3,u_4),(v_2,v_3,v_4),(w_2,w_3,w_4),0)$$
\end{itemize}

Similarly, we define the final states to be the states $s \in S_1$ such that $F_1(s) = (w_1,w_2,w_3)$ holds.
$F_1$ is given by the rules B1-B5 defined in Algorithm 1 and is shown below:

\begin{equation*}\label{def:alg1-final-func}
\begin{split}
    F_1&: S_1 \rightarrow \Sigma_k^3\\
    F_1&(s) := 
    \begin{cases}
        F_1((u_2,u_3,u_4),(v_1 - 1, v_2 + u_2 + 1, 0),(w_1,w_2,w_3),0) & \text{if } g = 1\\
        (v_1 + 1,v_2-u_2-1,u_3-1) & \text{if } v_1 < u_1,v_2 > u_2, v_3 = 0\\
        (v_1 + 1, v_2 - u_2, v_3 - 1) & \text{if } v_1 < u_2, v2 \geq u_2, 0 < v_3 \leq u_3\\
        (v_1 + 1, v_2 - u_2 + 1, v_3 - u_3 - 1) & \text{if } v_1 < u_2, v2 \geq u_2, v_3 > u_3\\
        (v_1, v_2 + 1, v_3 - u_3) & \text{if } v_2 < u_2, v_3 \geq u_3\\
        (v_1,v_2,v_3) & \text{otherwise}
    \end{cases}
\end{split}
\end{equation*}

Below we define the automaton $\mathcal{A}_2 = (S_2, I_2, T_2, F_2)$.
Note that $\mathcal{A}_3$ is just $\mathcal{A}_2$, but processing in reverse order.
We may therefore simply reverse the automaton $\mathcal{A}_2$ to obtain $\mathcal{A}_3$ \reed{cite}.

The set of states, $S_2$, is given by $S_2 := (\Sigma_k^2)^3$.
The initial state is $I_2 = ((0,0),(0,0),(0,0))$.
The transition function, $T_2$, is defined below.
Let $s = ((u_2,u_3),(v_2,v_3),(w_2,w_3))$ below.
Again, its domain is the subset of tuples $S_2 \times \{1,\ldots,k\} \times \Sigma_k \times \Sigma_k$ that satisfy one of the following rules:

\begin{itemize}
    \item If $v_1 < u_1$, $v_2 = u_2$, $v_3 > 0$, and $v_3 - 1 = w_3$, then:
    
    $$T_2(s,u_1,v_1,w_1) := ((u_1,u_2),(v_1+1,0),(w_1,w_2))$$
    
    \item If $v_3 = w_3$, then:
    
    $$T_2(s,u_1,v_1,w_1) := ((u_1,u_2),(v_1,v_2),(w_1,w_2))$$
\end{itemize}

Finally, a state $s \in S_2$ is a final state if $v_1 = w_1$ and $v_2 = w_2$ (i.e., $F_2(s) := (v_1,v_2)$).

We can think of each automaton as a predicate on finite sequences of input symbols.
For example, $\mathcal{A}_0(a,b,c)$ holds when $a + b = c$, where $+$ denotes digit-wise addition.
Using this notation, we define the general bounded addition automaton $+_{\alpha}$ which accepts four inputs: $\alpha$, $x$, $y$, and $z$, and accepts when $x +_{\alpha} y = z$.
We can define this automaton in terms of the four automata defined above and standard logical operators and quantifiers, ensuring that there is an automaton that accepts precisely the desired tuples.

Let $A^R$ denote the reversed automaton of $A$; i.e., if $A$ accepts $x$, then $A^R$ accepts $x^R$.

\begin{equation*}
    x +_{\alpha} y = z
\end{equation*}
\begin{equation*}
    \iff
\end{equation*}
\begin{equation*}
    \exists w : \mathcal{A}_0(x,y,w) \text{ and } \exists r : \mathcal{A}_1^R(a,w,r) \text{ and } \exists s : \mathcal{A}_2(a,r,s) \text{ and } \mathcal{A}_3^R(a,s,z)
\end{equation*}

\subsection{Automata Verification}

Similar to \reed{cite pell critical exponent paper because they do a similar thing}, we verify that each of our automata is correct using Walnut.

To do so, we use the following recursive definition of addition:

\begin{equation*}
    \begin{split}
        0 + n =&~n \\
        a + (b + 1) =&~(a + b) + 1\\
    \end{split}
\end{equation*}

Therefore, we must define the \term{successor}, which can be done without reference to addition, making it's use in the correctness proof non-circular.

\begin{definition}
Let $\alpha \in \alpha_{<k}$ and let $x \in N_{\alpha}$.
Then the \term{successor} of $x$ is the number $y \in N_{\alpha}$ such that $x <_{\alpha} y$ and for all $k \in N_{\alpha}$, either $k \leq_{\alpha} x$ or $y \leq_{\alpha} k$.
\end{definition}

That is, $y$ is the successor of $x$ if there are no numbers between $x$ and $y$, and $x <_{\alpha} y$.

Now, we check the base case: for all $\alpha \in \alpha_{<k}$ and $x,y \in N_{\alpha}$, $x = y$ if and only if $x +_{\alpha} = y$.
And the inductive case: for all $\alpha \in \alpha_{<k}$ and $u,v,x,y,z \in N_{\alpha}$, if $u$ is the successor of $y$ and $v$ is the successor of $z$, then $x +_{\alpha} y = z$ if and only if $x +_{\alpha} u = v$.

Below is the output from Walnut for the $k=2$ case.
First, defining the successor:

\begin{lstlisting}[basicstyle=\scriptsize\ttfamily]
def successor_2 "($recog_2(a,x) & $recog_2(a,y) & $lt_2(x,y) & Ak ($recog_2(a,k) 
    => ($lte_2(k,x) | ~($lt_2(k,y)))))":
    
(recog_2(a,x)&recog_2(a,y)) has 4 states: 0ms
 x<y has 2 states: 0ms
  ((recog_2(a,x)&recog_2(a,y))&x<y) has 6 states: 1ms
   k<=x has 2 states: 1ms
    y<=k has 2 states: 1ms
     (k<=x|y<=k) has 4 states: 3ms
      (recog_2(a,k)=>(k<=x|y<=k)) has 9 states: 10ms
       (A k (recog_2(a,k)=>(k<=x|y<=k))) has 9 states: 12ms
        (((recog_2(a,x)&recog_2(a,y))&x<y)&(A k (recog_2(a,k)=>(k<=x|y<=k)))) has 6 states: 1ms
total computation time: 30ms
\end{lstlisting}

Then proving the base case:
\begin{lstlisting}[basicstyle=\scriptsize\ttfamily]
eval base_proof_2 "Ez ($zero_2(z) & Aa,x,y (($recog_2(a,x) & $recog_2(a,y)) => 
    ($eq_2(x,y) <=> $add_2(a,x,z,y))))":

(recog_2(a,x)&recog_2(a,y)) has 4 states: 0ms
 x=y has 1 states: 0ms
  (x=y<=>add_2(a,x,0,y)) has 4 states: 2ms
   ((recog_2(a,x)&recog_2(a,y))=>(x=y<=>add_2(a,x,0,y))) has 1 states: 2ms
    (A a , x , y ((recog_2(a,x)&recog_2(a,y))=>(x=y<=>add_2(a,x,0,y)))) has 1 states: 3ms
total computation time: 21ms
\end{lstlisting}

And finally, proving the inductive case:
\begin{lstlisting}[basicstyle=\scriptsize\ttfamily]
eval successor_proof_2 "Aa,x ($recog_2(a,x) => Ay ($recog_2(a,y) => Az ($recog_2(a,z) => 
    Au (($recog_2(a,z) & $successor_2(a,y,u)) => Av (($recog_2(a,v) & $successor_2(a,z,v)) =>
        ($add_2(a,x,y,z) <=> $add_2(a,x,u,v)))))))":
        
(recog_2(a,z)&successor_2(a,y,u)) has 10 states: 1ms
 (recog_2(a,v)&successor_2(a,z,v)) has 6 states: 1ms
  (add_2(a,x,y,z)<=>add_2(a,x,u,v)) has 344 states: 1095ms
   ((recog_2(a,v)&successor_2(a,z,v))=>(add_2(a,x,y,z)<=>add_2(a,x,u,v))) has 222 states: 2892ms
    (A v ((recog_2(a,v)&successor_2(a,z,v))=>(add_2(a,x,y,z)<=>add_2(a,x,u,v)))) has 249 states: 788ms
     ((recog_2(a,z)&successor_2(a,y,u))=>(A v ((recog_2(a,v)&successor_2(a,z,v))=>
        (add_2(a,x,y,z)<=>add_2(a,x,u,v))))) has 1 states: 768ms
      (A u ((recog_2(a,z)&successor_2(a,y,u))=>(A v ((recog_2(a,v)&successor_2(a,z,v))=>
        (add_2(a,x,y,z)<=>add_2(a,x,u,v)))))) has 1 states: 2ms
       (recog_2(a,z)=>(A u ((recog_2(a,z)&successor_2(a,y,u))=>
        (A v ((recog_2(a,v)&successor_2(a,z,v))=>
            (add_2(a,x,y,z)<=>add_2(a,x,u,v))))))) has 1 states: 1ms
        (A z (recog_2(a,z)=>(A u ((recog_2(a,z)&successor_2(a,y,u))=>
            (A v ((recog_2(a,v)&successor_2(a,z,v))=>
                (add_2(a,x,y,z)<=>add_2(a,x,u,v)))))))) has 1 states: 0ms
         (recog_2(a,y)=>(A z (recog_2(a,z)=>(A u ((recog_2(a,z)&successor_2(a,y,u))=>
                (A v ((recog_2(a,v)&successor_2(a,z,v))=>(add_2(a,x,y,z)<=>add_2(a,x,u,v))))))))) has 1 states: 1ms
          (A y (recog_2(a,y)=>(A z (recog_2(a,z)=>(A u ((recog_2(a,z)&successor_2(a,y,u))=>
                (A v ((recog_2(a,v)&successor_2(a,z,v))=>
                    (add_2(a,x,y,z)<=>add_2(a,x,u,v)))))))))) has 1 states: 0ms
           (recog_2(a,x)=>(A y (recog_2(a,y)=>(A z (recog_2(a,z)=>
                (A u ((recog_2(a,z)&successor_2(a,y,u))=>
                    (A v ((recog_2(a,v)&successor_2(a,z,v))=>
                        (add_2(a,x,y,z)<=>add_2(a,x,u,v))))))))))) has 1 states: 0ms
            (A a , x (recog_2(a,x)=>(A y (recog_2(a,y)=>
                (A z (recog_2(a,z)=>(A u ((recog_2(a,z)&successor_2(a,y,u))=>
                    (A v ((recog_2(a,v)&successor_2(a,z,v))=>
                        (add_2(a,x,y,z)<=>add_2(a,x,u,v)))))))))))) has 1 states: 0ms
total computation time: 5549ms
\end{lstlisting}

\subsection{Discussion}

Though attempting to prove properties of even a single Sturmian word can often result in long, tedious proofs, doing the same with our automata and Walnut is generally quite simple and straightforward \reed{find some papers who just prove one property of fib word, for example}.
This ease makes exploring properties of various Sturmian words, or indeed, any automatic sequence, much easier.
We put this into practice in the following two sections, proving many known, and some unknown \reed{I think?} theorems about Sturmian words.

Many of the following theorems, in particular those in Section~\ref{sec:specific-theorems}, were proven using an ordinary laptop, and require no more than a few minutes at most to run.
However, many \reed{most?} of those found in Section~\ref{sec:conjectures} took much longer, because the automata for addition recognition are far larger, primarily in transitions, than those for any single Ostrowski numeration system.
Proving such propositions often exceeded the limits of Walnut or our computer, which had 52GB of RAM.
Nevertheless, we were still able to prove the conjectures for several cases.
Future work could improve the efficiency of the algorithms used and remove some of Walnut's limitations which merely exist because of its implementation.
