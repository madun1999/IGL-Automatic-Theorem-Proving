\documentclass[dvipsnames,runningheads]{llncs}
%
\usepackage{graphicx}
\usepackage{hyperref}
\usepackage{amsmath}
\usepackage{amssymb}
\usepackage{bm}
\usepackage{bbm}
\usepackage{mathtools}
\usepackage{enumitem}
\usepackage{mathpartir}
\usepackage{subfigure}
\usepackage{listings}
\usepackage{xcolor}
\usepackage{xspace}
\usepackage{tikz}
\usepackage{url}
\usepackage{tkz-euclide}
\usepackage[T1]{fontenc}

%\newcommand{\reed}[1]{\relax}
%\newcommand{\eric}[1]{\relax}
%\newcommand{\christian}[1]{\relax}
%\newcommand{\philipp}[1]{\relax}
%\newcommand{\Fix}[1]{\relax}
\newcommand{\reed}[1]{{\color{magenta}\bfseries [#1]}}
\newcommand{\eric}[1]{{\color{green}\bfseries [#1]}}
\newcommand{\christian}[1]{{\color{orange}\bfseries [#1]}}
\newcommand{\philipp}[1]{{\color{blue}\bfseries [#1]}}
\newcommand{\Fix}[1]{{\color{red}\bfseries [#1]}}
\newcommand{\Comment}[1]{}
\newcommand{\Space}[1]{}
\newcommand{\Num}[1]{#1}

\newcommand{\term}[1]{\emph{#1}}

\newcommand{\Free}{\textbf{Free}}

\newcommand{\evaluates}{\Downarrow}

\newcommand{\R}{\mathbb{R}}
\newcommand{\Q}{\mathbb{Q}}
\newcommand{\Z}{\mathbb{Z}}
\newcommand{\N}{\mathbb{N}}

\newcommand{\join}{\wedge}
\newcommand{\meet}{\vee}
\newcommand{\proves}{\vdash}

\newcommand{\dom}{\text{dom}~}

\newcommand{\proj}{\text{proj}}

\newcommand{\implicits}{\texttt{implicit}}
\newcommand{\params}{\texttt{params}}
\newcommand{\nonvar}{\texttt{nonvar}}

\newcommand{\tand}{\ensuremath{~\text{and}~}}
\newcommand{\tor}{\ensuremath{~\text{or}~}}
\newcommand{\twhere}{\ensuremath{~\text{where}~}}
\newcommand{\tif}{\ensuremath{~\text{if}~}}
\newcommand{\tsuchthat}{\ensuremath{~\text{s.t.}~}}
\newcommand{\owise}{\ensuremath{~\text{otherwise}~}}

\newcommand{\brackets}[3]{\ensuremath{{\left#1 {#3} \right#2}}}
\newcommand{\parens}[1]{\brackets{(}{)}{#1}}
\newcommand{\angles}[1]{\brackets{<}{>}{#1}}
\newcommand{\curlys}[1]{\brackets{\{}{\}}{#1}}
\newcommand{\squares}[1]{\brackets{[}{]}{#1}}

\newcommand{\bnfdef}{\ensuremath{\Coloneqq}}
\newcommand{\bnfalt}{\ensuremath{\mid}\xspace}

\newcommand{\inferred}{\texttt{Inferred}}
\newcommand{\prop}[1]{#1 ~ \text{prop}}
\newcommand{\typ}[1]{\text{typ}\parens{#1}}

\input{pecan-lang}

\renewcommand\UrlFont{\color{blue}\rmfamily}

\begin{document}

\title{Pecan: An Automated Theorem Prover for Automatic Sequences using B\"uchi Automata}

\titlerunning{Pecan: An Automated Theorem Prover}

\author{Reed Oei \and
Dun Ma \and
Christian Schulz \and
Philipp Hieronymi}

\institute{University of Illinois at Urbana-Champaign, Urbana, USA\\
\email{reedoei2,dunma2,cschulz3,phierony@illinois.edu}}

\maketitle

\begin{abstract}
Pecan is an automated theorem prover for reasoning about properties of Sturmian words, an important object in the field of combinatorics on words. It is capable of efficiently proving non-trivial mathematical theorems about all Sturmian words.
\keywords{Automatic theorem proving  \and Sturmian words \and Implementation.}
\end{abstract}

\section{Introduction}

\textbf{Pecan} is a system for \emph{automated theorem proving} originally designed to decide mathematical statements about families of infinite words, in particular about Sturmian words, and based on well-known decision procedures for B\"uchi automata due to B\"uchi \cite{Buechi}. Pecan is inspired by Walnut~\cite{walnut} by Mousavi, another automated theorem prover for deciding combinatorical properties of automatic words. \textbf{Automatic words} are sequences of terms characterized by finite automata. The main motivation to create this new tool is to decide whether a statement is true for every element of an infinite family of words rather than just determining the truth of the statement for a single given words. In such a situation not every word in this family of words is automatic, but the whole family can be recognized by an automaton.
Since the infinite families of words we want to consider are often indexed by real numbers, it is convenient to work with B\"uchi automata instead of finite automata. The canonical example of such a automatic family of words are the \textbf{Sturmian words}, that is the family $(\mathbf{w}_{\alpha,\rho})$ of all words $w=(w_n)$ over the alphabet $\{0,1\}$ such that there is $\rho \in [ 0,1 )$, called the \emph{intercept}, and an irrational $\alpha \in (0,1)$, called the \emph{slope}, with
\[
w_{n}=\lfloor n\alpha +\rho\rfloor -\lfloor (n-1)\alpha +\rho\rfloor
\]
for all $n\in \N$. Using Pecan, we can automatically reprove classical and recent theorems about Sturmian words, like the fact that they are not periodic, within minutes, and even have been able to prove completely new mathematical theorems using this software. 

The idea of using automata-based decision procedures to prove theorems in combinatorics on words has been championed by Jeffrey Shallit and successfully implemented in several papers of Shallit and his many co-authors (see Shallit \cite{Shallit-survey} for a survey and Baranwal, Schaeffer, Shallit \cite{BARANWAL2021} for implementations of decision procedure for individual Sturmian words). The development of Pecan is our contribution to this exciting research program. We leave the detailed discussion of the mathematical background such as why Sturmian words can represented using automata and which statements about Sturmian words can be proved using Pecan, to the upcoming paper \cite{DecStuWor}. Here we describe the implementation of Pecan and discuss its performance.

% Pecan comes with \textbf{Praline}\footnote{Because it is syntactic \textbf{sugar} for Pecan.}, a scripting language designed to make working with Pecan more pleasant.
% It provides the primary interface to the counterexample generation capabilities of Pecan, and also allows some degree of metaprogramming: for example, programmatically generating predicate definitions.

\begin{subsection}{Related work} 
Pecan improves on Walnut~\cite{walnut}, a similar automata-based theorem prover for automatic sequences, by using B\"uchi automata instead of finite automata.
This difference enables Pecan to handle uncountable families of sequences, allowing us quantify over all Sturmian words. Additionally, the Pecan language is able to use multiple numeration systems at a time, has a concept of types outside of numeration systems, and has meta-programming language, Praline.

Many other theorem provers exist, such as SMT solvers and proof assistants, like Coq~\cite{the_coq_development_team_2020_3744225} or Isabelle~\cite{nipkow2002isabelle}.
To our knowledge, no SMT solver supports reasoning about Sturmian words.
Systems like Coq or Isabelle have projects attempting to formalize some aspects of combinatorics on words and automatic sequences~\cite{hivert2018littlewoodrichardson,holub2020binary}.
However, proofs in these systems are mostly human written, with some help from heuristics or specialized solvers, rather than being fully automatic, as in Pecan.

B\"uchi automata have also been used extensively in program verification in systems such as SPIN~\cite{gerard2003spin}.
However, we are interested in proving mathematical results, rather than proofs about properties of programs.
For this reason, we must allow unrestricted use of logical operations, such as negation, rather than restricting to more limited forms of expressing properties, such as linear temporal logic, which such systems tend to use for performance reasons.

\end{subsection}

\begin{subsection}{Acknowledgements}
Support for this project was provided by the Illinois Geometry Lab. This project was partially supported by NSF grant DMS-1654725. \end{subsection}
\section{Background}\label{sec:background}

This section contains an informal introduction to words, automata, and the notation that we use.
For precise statements and proof, we refer the reader to Allouche and Shallit \cite{MR1997038} or Khoussainov and Nerode ~\cite{aut_theory}.

Let $\Sigma^*$ denote the set of finite words on the alphabet $\Sigma$, let $\Sigma^+$ denote the set of nonempty finite words on the alphabet $\Sigma$, and let $\Sigma^\omega$ denote the set of $\omega$-words on the alphabet $\Sigma$.

For a word $w$, let $w[i]$ denote the $i$-letter of $w$.
Let $w(i,n)$ denote the length-$n$ factor of $w$ starting at $i$ and ending at $i + n - 1$, that is, $w[i \ldots i + n - 1] = w[i] w[i + 1] \cdots w[i + n - 1]$.
Let $|w|$ denote the \term{length} of $w$.
%Let $w^R$ denote the \term{reverse} of $w$, so if $w = w[1] w[2] \cdots w[n]$, then $w^R = w[n] w[n-1] \cdots w[1]$.
%In a binary alphabet, $\Sigma = \{0,1\}$, for a symbol $x \in \Sigma$, let $\overline{x}$ denote the \term{complement} of $x$, so $\overline{0} = 1$ and $\overline{1} = 0$.
%For a word $w$ over a binary alphabet, let $\overline{w}$ denote the \term{complement} of $w$, that is $\overline{w} = \overline{w[1]} \, \overline{w[2]} \cdots \overline{w[n]}$.
%Let $w^n$ denote concatenating $w$ with itself $n$ times, i.e., $w^n = \overbrace{w \cdots w}^{n\text{ times}}$.

\term{B\"uchi automata} are an extension of the standard finite automata to infinite inputs.
A B\"uchi automata $\mathcal{A} = (Q, \Sigma, \delta, q_0, F)$ accepts an infinite word $w \in \Sigma^\omega$ if the run of the automaton on the word $w$ visits an accepting state (i.e., a state in $F$) infinitely many times.
The set of words accepted by $\mathcal{A}$ is its \term{language}, $L(\mathcal{A})$.
Notably, nondeterministic B\"uchi automata and \textbf{not} equivalent to deterministic B\"uchi automata, and many interesting properties are only expressible via nondeterministic B\"uchi automata.
For that reason, we simply refer to nondeterministic B\"uchi automata as B\"uchi automata, without qualification.
Additionally, when we say ``automata'' without qualification, we refer to B\"uchi automata.
Importantly, the languages that B\"uchi automata define are closed under intersection, union, projection, and complementation, and emptiness checking is decidable.

%\cite{BARANWAL2021}
%Informally, a binary sequence $(a_n)_{n \in \N}$ is \term{automatic} if there is some automaton $\mathfrak{A}$ such that $a_n = 1$ if and only if $\rho(n) \in L(\mathfrak{A})$, where $\rho : \N \to \Sigma^*$ is a function mapping natural numbers to their representations in some numeration system. For example, $\rho$ might map natural numbers into their $k$-ary representations. In this case, we call the sequence $(a_n)_{n\in \N}$ $k$-automatic. 




% Using B\"uchi automata has a number of advantages over finite automata:
% \begin{itemize}
%         \item Some problems are more naturally expressed with B\"uchi automata.
%             For example, we think of numbers as having infinitely many leading/leading zeros; using B\"uchi automata, we can encode number so that they actually start with infinitely many zeros.
%             While existential quantification for Pecan works the same regardless of input type, Walnut needs a special algorithm to handle these leading/tailing zeroes because it uses finite automata.
            
%         \item Some problems can \emph{only} be expressed with B\"uchi automata.
%             For example, it is clearly impossible to encode real numbers as finite strings because there are uncountably many real numbers but only countably many finite strings (on a finite alphabet).
%             However, using B\"uchi automata, we are able to automatically prove properties about real numbers.
        
%         \item We can still express properties about finite strings (e.g., by using some symbol as an ``end of input'').
% \end{itemize}

% There are also disadvantages to using B\"uchi automata: in particular, many of the algorithms implementing the various logical operations on B\"uchi automata are much slower than their finite automata counterparts.
% For example, while complementing finite automata is simple, complementing B\"uchi automata requires exponential time.
% In practice, however, we can often calculate complements, even of large automata, in reasonable amounts of time.

\section{Overview}\label{sec:features}

For full documentation on the features of Pecan, see the more comprehensive manual available at our repository~\cite{pecan-repo}.

% \paragraph{Functions}
% This is already shown in the implementation section, no need to discuss here
% Some predicates represent relations which are functions.
% For readability, Pecan allows the use of \term{placeholders}, which allow specifying the output of a function-like predicate.
% For example, if we have a predicate \pecaninline{bin\_add(x, y, z)} accepting triples such that $x + y = z$ for binary numbers $x$, $y$, and $z$, then \pecaninline{bin\_add(x, y, \_)} is equivalent to writing $x + y$.
% So we could have written \pecaninline{bin\_add(x, y, \_) = z} instead of \pecaninline{bin\_add(x, y, z)}.
% Such a formula will be translated into \pecaninline{existsv. bin_add(x, y, v) & v = z}.

% Because the last argument to a function-like relation is typically the output argument, we can simply write \pecaninline{f(x)} to mean \pecaninline{f(x, \_)}.

% \paragraph{Annotations}
% Pecan supports \term{annotations}, which may \term{wrapped} around any formula.
% They do \textbf{not} change the logical value of the automaton, but are instead used to tell Pecan how to evaluate a formula (e.g., when to simplify).
% The most commonly used annotations are:
% \begin{itemize}
%     \item \pecaninline{@postprocess[FORMULA]}: Performs a variety of simplification steps on the automaton representing \pecaninline{PREDICATE}.
%     \item \pecaninline{@no_simplify[FORMULA]}: Turns off simplification (on by default) for the formula and all its subformulas.
% \end{itemize}

\paragraph{Directives} are the interface to Pecan, instructing it to perform actions (e.g., prove a theorem).
We discuss the most important: \pecaninline{Restrict}, \pecaninline{Structure}, and \pecaninline{Theorem}.

\begin{pecan}
Restrict VARIABLES are TYPE_PREDICATE.
\end{pecan}

In all following code in the file in which the \pecaninline{Restrict} appears, the variables specified are now consider to be of the specified type.

\vspace{-0.5em}
\begin{pecan}
Structure TYPE_PREDICATE defining { FUNCTION_PREDICATES }
\end{pecan}
\vspace{-0.5em}

Defines a new structure.
The \pecaninline{TYPE_PREDICATE} is essentially the part written after the \pecaninline{is} in a restriction. 
For example, in \pecaninline{Restrict x is nat.}, the type predicate is \pecaninline{nat}; in \pecaninline{Restrict i is ostrowski(a).}, the type predicate is \pecaninline{ostrowski(a)}.
The function predicates become available to be called using the names in quotes---this feature allows for ad-hoc polymorphism, as described in Section~\ref{sec:implementation}.
It is also used to resolve arithmetic operators, such as \pecaninline{+} (which calls the relevant \pecaninline{adder}) and \pecaninline{<} (which calls the relevant \pecaninline{less}).

\vspace{-0.5em}
\begin{pecan}
Theorem ("THEOREM NAME", { PREDICATE }).
\end{pecan}
\vspace{-0.5em}
\pecaninline{Theorem} is the interface to the theorem proving capabilities of Pecan, stating that Pecan show attempt to prove the specified \pecaninline{PREDICATE} is true.

Below is an example of using all three features from above: specifying a structure called \pecaninline{nat}, restricting variables, and then proving a theorem, which is true because of the dynamic call resolution.
\vspace{-0.5em}
\begin{pecan}
Structure nat defining {
    "adder": bin_add(any, any, any),
    "less": bin_less(any, any)
}
Restrict a, b are nat.
Theorem ("", { forall a,b. a < b <=> bin_less(a,b)}).
\end{pecan}
\vspace{-0.5em}

\paragraph{Automatic Words}
Any predicate $P$ can be interpreted as a word by writing $P[i]$, which is treated as $1$ if $P(i)$ is true, and $0$ if $P(i)$ is false.
Currently only binary automatic words are supported.
We use the following translations into the IR:

\begin{itemize}
    \item $P[i] = 0 \leadsto \lnot P(i)$
    \item $P[i] = 1 \leadsto P(i)$
    \item $P[i] = Q[j] \leadsto P(i) \iff P(j)$
    \item $P[i] \neq Q[j] \leadsto \lnot (P(i) \iff P(j))$
    \item $P[i..j] = P[k..\ell] \leadsto j + k = i + \ell \land \forall n \in \typ{i}. i + n < j \Rightarrow P[i + n] = P[k + n]$
\end{itemize}

% \paragraph{Praline} is a functional language for metaprogramming that is a part of the Pecan system.
% We omit a full description due to space reasons, as it plays only a supporting role to the main system. 
% Some examples of its uses are:
% \begin{itemize}
%     \item Building automata that cannot be specified by a first-order formula involving other automata.
%     \item Defining useful helpers, such as \pecaninline{Theorem}, which generates certain Pecan definitions and statements.
% \begin{pecan}
% Define theoremCheck $\$$ID $\$$BODY := do
%   emit {$\$$ID() := $\$$BODY}; emit {#assert_prop(true,$\$$ID)}.
% Alias "Theorem" ==> Execute uncurry theoremCheck.
% Theorem ("Addition is commutative", 
% { forallx,y are nat. x + y = y + x }).
% \end{pecan}
%     \item Formatting the raw output of Pecan into human readable output.
%         For example, the real number encoding in Pecan represents $17/5$ as $00111(10000010)^\omega$.
%         Using Pecan, we can format this as $11.01(1001)^\omega$ in binary.
% \end{itemize}

% \begin{pecan}
% #save_aut(FILENAME, PREDICATE)
% #save_aut("bin_even.aut", bin_even)
% \end{pecan}

% Saves the predicate as an automaton in the modified HOA~\cite{hoa-format} format to the location specified.

% \begin{pecan}
% #load(FILENAME, FORMAT, PREDICATE)
% #load("bin_add.aut", "hoa", bin_add(a,b,c))
% #load("real/real_equal.txt", "pecan", real_equal(a, b))
% \end{pecan}

% Loads an automaton from a file with the specified name and arguments.
% Note that the number of arguments is \emph{NOT} checked for you.
% In general, you should be careful when loading automata from files because there are not many safeguards (it is assumed you know what you are doing).
% There are three currently supported formats: ``hoa''~\cite{hoa-format}, as described earlier; ``walnut'', the format that the automated theorem prover Walnut~\cite{walnut} uses; and ``pecan'', a custom format that Pecan uses, described below in the manual~\cite{pecan-repo}.

% \begin{pecan}
% #import(FILENAME)
% #import("integers.pn")
% \end{pecan}

% Loads all definitions from the specified file into the current scope.
% Does \emph{NOT} load restrictions into the current scope, but they function normally while evaluating the imported file.
% To control the search path, you can set the \lstinline{PECAN_PATH} environment variable.
% By default, the current working directory (at the time of loading the program), the directory the current source file is in, and the \lstinline{library/} directory of the Pecan directory.

\section{Implementation}\label{sec:implementation}

This section describes at a high-level the implementation of Pecan.
First, we describe the syntax in Section~\ref{sec:ir-syntax}.
Then, we give a formal definition of the Pecan language, starting with the typing rules and associated definitions in Section~\ref{sec:typing}, and then the rules for evaluation in Section~\ref{sec:evaluation}.
Next, we describe each step of the AST to IR conversion in Section~\ref{sec:ast-to-ir}, and finally the various optimizations performed by Pecan prior to execution in Section~\ref{sec:optimizations}.

The process of executing a program is roughly as follows:

\begin{enumerate}
    \item Parse the program into an AST (Abstract Syntax Tree).
    \item Transform the program's AST into a simplified IR  representation, rewriting constructs such as $\iff$ in terms of simpler ones, like $\land$, $\lor$, and $\lnot$.
    \item Load the standard library (if desired); loading the standard library requires going through all of the steps again, starting from step 1, but skipping this step.
    \item Then, for each definition or directive, perform the following steps:
        \begin{enumerate}
            \item Run the untyped optimizer, if enabled.
            \item Perform type inference.
            \item Perform a final IR lowering.
            \item Run the typed optimizer, if enabled.
            \item Interpret the final IR of the current top-level construct.
        \end{enumerate}
\end{enumerate}

\subsection{IR Syntax}\label{sec:ir-syntax}

Below is the IR Syntax for the Pecan language; the full syntax that Pecan supports is much larger, but the rest of it is ``simply'' syntactic sugar which is expanded as explained in Section~\ref{sec:ast-to-ir}.

\begin{tabular}{c c l}
     Prog & \bnfdef & Definition$^*$ \\
     Definition & \bnfdef & $P(x_1 : \tau_1, \ldots, x_n : \tau_n)$ := Pred \\
     & \bnfalt & Restrict $x_1,\ldots,x_n$ are $P(y_1,\ldots,y_m)$ \\
     Pred & \bnfdef & true \bnfalt false \bnfalt Pred $\lor$ Pred \bnfalt $\lnot$ Pred \bnfalt Pred $\land$ Pred \\
     & \bnfalt & $\exists x. $ Pred \bnfalt E $<$ E \bnfalt E $=$ E \bnfalt @A[Pred] \bnfalt $P$(E, E, \ldots, E) \bnfalt Aut($V, \mathcal{A}$) \\
     E & \bnfdef & E + E \bnfalt E - E \bnfalt $x$ \bnfalt $i$ \bnfalt $f$(E, E, \ldots, E) 
\end{tabular}

\subsection{Type Checking}\label{sec:typing}

A \term{type} in Pecan is represented by a B\"uchi automaton.
We say that $x : \tau$ when $x \in L(\tau)$, sometimes simply written, as an analogy to logical predicates, as $\tau(x)$.
Additionally, types may be \term{partially applied}, i.e., $\tau = P(x_1, \ldots, x_n)$, where $P$ is some B\"uchi automaton.
Then $y : \tau$ when $(x_1, \ldots, x_n, y) \in L(P)$; and $\tau(y)$ holds when $y : \tau$.
In the concrete syntax of Pecan, we write \pecaninline{y is tau} or \pecaninline{y} $\in$ \pecaninline{tau}; for one or more variables, we can write \pecaninline{x, y, z are tau} to mean $x : \tau$, $y : \tau$, $z : \tau$.
Finally, there is a special type called $\inferred$, which should be thought of as a type that will be discovered later by how the value is used; or, as its name suggests, the type will eventually be \textbf{inferred} from context.
$L(\inferred)$ is undefined---$\inferred$ is NOT represented by an automata.

The judgement $\Gamma \proves x : \tau$ means that we can prove $\tau(x)$ is true in the environment $\Gamma$, which is a set of assumptions $x : \tau$.
The set $\{ x : \exists \tau. x : \tau \in \Gamma\}$ is the \term{domain} of $\Gamma$, written $\dom{\Gamma}$.
If for every $x \in \dom{\Gamma}$, there is a unique type $\tau$ such that $x : \tau \in \Gamma$, we say that $\Gamma$ is well-formed.
We assume that all contexts are well-formed unless otherwise specified.
The judgement $\Gamma \proves \prop{P}$ means that $P$ is a well-formed proposition in the environment $\Gamma$.
A predicate $P(\overline{x : \tau}) := Q$ is well-formed when $\overline{x : \tau} \proves \prop{Q}$.

We typecheck a sequence of Pecan predicates in order, starting with an empty environemnt $\Gamma = \emptyset$.
Below, we assume that the set of all well-formed predicates, which have already been checked, is ambiently available as $\mathcal{P}$.
Similarly, whenever we encounter a restriction, Restrict $x_1, \ldots, x_n$ are $P(y_1, \ldots, y_m)$, we update the current environment $\Gamma$ to be $\Gamma ~\cup~ \{ x_1 : P(y_1, \ldots, y_m), \ldots, x_n : P(y_1, \ldots, y_m, x_n) \} $; it is only allowed to use $P \in \mathcal{P}$, that is, predicates which have already been confirmed are well-formed.

\paragraph{Structures}

In order to provide a mechanism for ad-hoc polymorphism, Pecan allows the definition and use of \term{structures}.
This feature also facilitates the use of nicer syntax for arithmetic expressions (e.g., $x + (y + z) = w$ instead of $\exists t. \texttt{adder}(x, y, t) \land \texttt{adder}(t, z, w)$) without tying ourselves to a single numeration system.
For example, $\texttt{adder}$ will be resolved to some concrete predicate predicate based on the type of $x$, $y$, and $z$.
The exact rules and definitions related to this feature are given below.
We assume that structure definitions are ambiently available throughout the program.

\begin{definition}
    A \term{structure} is a pair $(t(x_1, \ldots, x_n), D)$ where each $x_i$ is an identifier, such that for some $\tau_1, \ldots, \tau_n$, $x_1 : \tau_1, \ldots, x_n : \tau_n \proves \prop{t(x_1, \ldots, x_n)}$ and $D$ is a map of identifiers to \term{call templates}; that is, it is of the form $f(y_1, \ldots, y_m)$, where each $y_i$ may be either: 1) $x_j$ for some $j$ or 2) $*$, which is pronounced ``any.''
    
    The \term{name} of the structure is $t$.
\end{definition}

We write the sequence of indexes of the arguments that are $*$, called \term{parameters}, as $\params(f(y_1, \ldots, y_m))$.
A call template is called $n$-ary if $|\params(f(y_1, \ldots, y_m))| = n$.
The sequence of the indexes of the other arguments, which are not $*$, called \term{implicits}, is written $\implicits(f(y_1, \ldots, y_m))$.
For example, $\params(f(a, *, *, b, *, *)) = [2, 3, 5, 6]$ and $\implicits(f(a, *, *, b, *, *)) = [1, 4]$.
We write $\params(f(y_1, \ldots, y_m))[i]$ (resp. $\implicits(f(y_1, \ldots, y_m))[i]$) to denote the $i$-th parameter (resp. implicit).

We assume that typechecking has been done before evaluating, because we may need structure information at run-time to resolve \term{dynamic calls}, that is, predicate calls whose name matches some definition inside a type.
We denote the type that an expression $e$ got when typechecking by $\typ{e}$.

We write $t[P] = Q(y_1, \ldots, y_m)$ to look up a definition in the associated map $D$, and we say that $t$ \term{has a definition} for $P$ in this case.
If $t$ does not have a definition for $P$, then we write $t[P] = \bot$.

\begin{definition}
    A structure is called \term{numeric} if it has a ternary definition for $\texttt{adder}$ and a binary definition $\texttt{less}$.
    We write $x + y = z$ when $\texttt{adder}(x, y, z)$ holds and $x < y$ when $\texttt{less}(x, y)$ holds.
    
    A numeric structure may also optionally contain the following definitions; otherwise a default predicate applies.
    
    \begin{itemize}
        \item[] A binary definition $\texttt{equal}$, written $x \equiv y$. 
            If not provided, the default is simply standard equality, $x = y$.
        
        \item[] A unary definition $\texttt{zero}$.
            If not provided, the default is $0^{\omega}$.
            
        \item[] A unary definition $\texttt{one}$.
            If not provided, the default is $x$ such that $0 \leq x \land \forall y. y = 0 \lor x \leq y$.
    \end{itemize}
\end{definition}

\begin{definition}
    The \emph{predecessor} of two types $\tau$ and $\sigma$, written $\tau \join \sigma$ is given by following partial function:
    \[
        \tau \join \sigma = 
        \begin{cases}
            \sigma & \tif \tau = \inferred~\text{and $\sigma$ is numeric} \\
            \sigma & \tif \forall \Free(\tau). \forall x. \tau(x) \implies \sigma(x) \\
            \tau & \tif \forall \Free(\tau). \forall x. \sigma(x) \implies \tau(x) \\
            \tau & \tif \sigma = \inferred~\text{and $\tau$ is numeric} \\
        \end{cases}
    \]
\end{definition}

\begin{definition}
    We can \term{resolve} a call $P(e_1, \ldots, e_n)$ as $Q(a_1 : \tau_1, \ldots, a_m : \tau_m)$, written $P(e_1, \ldots, e_n) \leadsto Q(a_1 : \tau_1, \ldots, a_m : \tau_m)$, if $Q(a_1 : \tau_1, \ldots, a_m : \tau_m) \in \mathcal{P}$ and for some structure $t(x_1, \ldots, x_\ell)$, for each $1 \leq i \leq n$, either:
    \begin{enumerate}
        \item $\typ{e_i} = t(x_1, \ldots, x_\ell)$, and $t[P] = Q(b_1, \ldots, b_m)$ such that
        for each $1 \leq j \leq m$,
        \[
            a_j =
            \begin{cases}
                x_k & \tif \implicits(Q(b_1, \ldots, b_m))[k] = j \\
                e_k & \tif \params(Q(b_1, \ldots, b_m))[k] = j
            \end{cases}
        \]
        
        \item $\typ{e_i} = s(y_1, \ldots, y_p)$, where $s \neq t$ and $s[P] = \bot$.
    \end{enumerate}
    
    or, if none of the arguments have a definition for $P$, then $P(e_1, \ldots, e_n) \leadsto P(a_1 : \tau_1, \ldots, a_m : \tau_m) \in \mathcal{P}$.
\end{definition}

\framebox{$\Gamma \proves e : \tau$} \textbf{Typing}
\begin{mathpar}
\inferrule*[right=Int]{
    i \in \Z
} { \Gamma \proves i : \inferred }

\inferrule*[right=Var]{
    x : \tau \in \Gamma
} { \Gamma \proves x : \tau }

% \reed{Not sure that I really want to support full type inference like this}
% \inferrule*[right=Var-Env]{
%     x \not\in \dom{\Gamma}
% }{ \Gamma \proves x : \inferred }

\inferrule*[right=Op]{ 
    \Gamma \proves a : \tau
    \and 
    \Gamma \proves b : \sigma
}{ \Gamma \proves a \oplus b : \tau \join \sigma }
\twhere \oplus \in \{ +, -, / \}

\inferrule*[right=Func]{
    \forall i \in \{ 1, \ldots, n \}. \Gamma \proves e_i : \tau_i
    \and
    f(x_1 : \tau_1, \ldots, x_n : \tau_n, r : \tau) \in \mathcal{P}
}{ \Gamma \proves f(e_1, \ldots, e_n) : \tau }
    
\end{mathpar}

\framebox{$\Gamma \proves \prop{P}$} \textbf{Well-formed Propositions}
\begin{mathpar}
\inferrule*[right=Rel]{ 
    \Gamma \proves a : \tau
    \and 
    \Gamma \proves b : \sigma
    \and
    \tau \join \sigma \neq \inferred
}{ \Gamma \proves \prop{a \Join b} }
\twhere \Join \in \{ \equiv, < \}

\inferrule*[right=BinPred]{
    \Gamma \proves \prop{P}
    \and
    \Gamma \proves \prop{Q}
}{ \Gamma \proves \prop{P \oplus Q} }
\twhere \oplus \in \{ \lor, \land \}

\inferrule*[right=Comp]{
    \Gamma \proves \prop{P}
}{ \Gamma \proves \prop{\lnot P} }

\inferrule*[right=Exists]{
    \Gamma, x : \tau \proves \prop{P}
}{ \Gamma \proves \exists x : \tau. \prop{P} }

\inferrule*[right=Call]{
    \forall i \in \{1, \ldots, n\}. \Gamma \proves e_i : \tau_i
    \and
    f(x_1 : \tau_1, \ldots, x_n : \tau_n) \in \mathcal{P}
}{ \Gamma \proves \prop{f(e_1, \ldots, e_n)} }

\inferrule*[right=Annotation]{
    \Gamma \proves \prop{P} 
}{ \Gamma \proves \prop{@A[P]} }

\inferrule*[right=Automaton]{
}{ \Gamma \proves \prop{\text{Aut}(V, \mathcal{A})} }
\end{mathpar}

\reed{TODO: Some proofs maybe (e.g., if $\Gamma \proves x : \tau$, then if $\Gamma$, then $x \in L(\tau)$, or if $\Gamma \proves \prop{P}$ then $P \evaluates \mathfrak{A}$ for some $\mathfrak{A}$}

\subsection{Evaluation}\label{sec:evaluation}

The Pecan interpreter is a simple tree-walking interpreter which typechecks, then processes, each top-level construct (see Section~\ref{sec:ir-syntax}) in sequential order.

Operations with non-trivial implementations are described in detail below.
Most basic automata operations (e.g., conjunction, disjunction, complementation, emptiness checking, simplification) are implemented using the Spot library \cite{duret.16.atva2}, so details of these algorithms are not presented here.

\subsubsection{Automata Representation}

Automata are represented by a pair of $(V, \mathcal{A})$, where $V$ is a map taking variable names to an ordered list of APs that represent it, called the \emph{variable map}, and $\mathcal{A}$ is a Spot automaton (specifically, a value of type \texttt{spot.twa\_graph}).
For more information about the underlying representation of Spot automata, see the Spot library \cite{duret.16.atva2}.
We use the convention that calligraphic letters represent actual B\"uchi automata, and Fraktur letters represent automata in the Pecan sense of a pair of a variable map and B\"uchi automaton.

\paragraph{Variable Maps}
A \emph{variable map} $V$ is a finite set of mappings $x \mapsto [\texttt{ap}_1, \ldots, \texttt{ap}_n]$ such that for all distinct variables $x$ and $y$, $V[x] \cap V[y] = []$.

We denote by $V[x]$ the list of APs that $x$ is represented by, and we denote by $V \cup W$ the union of two variables maps union, which is only defined when the only keys that $V$ and $W$ have in common have identical APs.
$V \sqcup W$ is the disjoint union of these maps.
$V \setminus K$ is the variable map containing every entry $x \mapsto a \in V$ such that $x \not\in K$.

For two variable maps $V$ and $W$, $V \ll W$ denotes their \emph{biased merge}, which is a pair $(U, \theta)$ of a variable map $U$ and a substitution $\theta$ such that $U = V \cup W\theta$.
A substitution is a set of mappings $a \mapsto b$ where $a$ and $b$ are both APs, which can be applied to a variable map or an automaton to rename the APs in them.
For example, if $\theta = \{ a \mapsto d, c \mapsto e \}$, then $\{ x \mapsto [ a, b, c ] \} \theta = \{ x \mapsto [ d, b, e ] \}$.
When it is clear, we also write $V \ll W$ to denote just the resulting variable map, without the associated substitution.

\subsubsection{Logical Operations}

Fundamental automata operations (i.e., $\land$ and $\lor$, represented by $\oplus$ below) are defined below.
\[
    (V, \mathcal{A}) \oplus (W, \mathcal{B}) = 
    \begin{cases}
        (V \ll W, \mathcal{A} \oplus \mathcal{B}) & \tif |S(\mathcal{A})| < |S(\mathcal{B})| \\
        (W \ll V, \mathcal{A} \oplus \mathcal{B}) & \owise
    \end{cases}
\]

where $S(\mathcal{A})$ denotes the set of states of $\mathcal{A}$.
For complements, we define $\lnot (V, \mathcal{A}) = (V, \lnot \mathcal{A})$.

Finally, automata literals, written Aut($V, \mathcal{A}$) simply evaluate to be the automata they store:
\begin{mathpar}

\inferrule*[right=Automaton]{
}{ \text{Aut}(V, \mathcal{A}) \evaluates (V, \mathcal{A}) }

\end{mathpar}

\subsubsection{Predicate Calls}

\begin{mathpar}
\inferrule*[right=Call]{
    \forall i. e_i \evaluates (\mathfrak{A}_i, x_i)
    \and
    \nonvar(e_1, \ldots, e_n) = [k_1, \ldots, k_{\ell}]
    \\
    P(x_1, \ldots, x_n) \leadsto Q(y_1, \ldots y_m)
    \and
    Q(z_1, \ldots, z_m) := R
    \and
    R \evaluates \mathfrak{B}
}{ P(e_1, \ldots, e_n) \evaluates \exists x_{k_1}, \ldots, x_{k_{\ell}}. \mathfrak{A}_1 \land \cdots \land \mathfrak{A}_n \land \mathfrak{B}[y_1 / z_1, \ldots, y_m / z_m] }
\end{mathpar}

where $\mathfrak{B}[y_1 / z_1, \ldots, y_m / z_m]$ denotes substituting each $y_j$ for $z_j$ in $\mathfrak{B}$, defined in Section~\ref{sec:impl-substitution}, and $\nonvar(e_1, \ldots, e_n)$ denotes the nonvariable positions in $e_1, \ldots, e_n$.

The rule for evaluating predicate calls is rather complicated, due to the possibility of dynamic calls.
Fundamentally, the rule does the following:

\begin{enumerate}
    \item Evaluates each argument $e_i$ as $(\mathfrak{A_i} x_i)$.
    \item Resolves the call $P$ as some predicate $Q$
    \item Evaluates the body of $Q$, $R$ and then substitutes in the variables $y_i$ (in the resolve call, some of which will be the result variables $x_i$) as appropriate.
    \item Projects out the intermediate result variables $x_{k_p}$, where the $k_p$ are the nonvariable positions in $e_1, \ldots, e_n$.
        This means that it only makes sense to use expressions whose output values will be well-behaved: that is, the resulting automaton should be the same regardless of which output was ``used.''
        Pecan will not automatically check this condition for performance reasons, but it is possible to do automatically.
\end{enumerate}

\subsubsection{Expressions}

Denote by $\Free(E)$ the list of free variables occurring in $E$.
For example, $\Free(a + b + c) = [a, b, c]$.
Denote by $E[v/x]$ the expression $E$ with $v$ substituted for $x$ where $x \in \Free(E)$.

An expression $E$ evaluates to a pair $(\mathfrak{A}, x)$ of an automaton $\mathfrak{A}$ and a variable $x$ where $\mathfrak{A}$ accepts $(v_1, \ldots, v_n, x)$ such that $\Free(E) = [x_1, \ldots, x_n]$ and $E[v_1/x_1, \ldots, v_n/x_n] = x$.

Note that many rules, like \textsc{Add} or \textsc{Equal} may need to evaluate subexpressions.
However, while evaluating a subexpression $e$, it may be that we generate fresh variables to store the result, which must be projected out.
The only case in which this does not occur is when the subexpression is itself a variable.

\begin{mathpar}
\inferrule*[right=Var]{ }{ x \evaluates (\top, x) }

\inferrule*[right=Add]{ 
    a \evaluates (\mathfrak{A}, x)
    \\
    b \evaluates (\mathfrak{B}, y)
    \\
    V = \{ v : (e, v) \in \{ (a,x), (b,y) \}, e \neq v \}
    \\
    (x + y = z) \evaluates \mathfrak{C}
} { a + b \evaluates (\proj_V(\mathfrak{A} \land \mathfrak{B} \land \mathfrak{C}), z) }
\twhere z ~ \text{is fresh}

\inferrule*[right=Sub]{ 
    a \evaluates (\mathfrak{A}, x)
    \\
    b \evaluates (\mathfrak{B}, y)
    \\
    V = \{ v : (e, v) \in \{ (a,x), (b,y) \}, e \neq v \}
    \\
    (z + y = x) \evaluates \mathfrak{C}
} { a - b \evaluates (\proj_V(\mathfrak{A} \land \mathfrak{B} \land \mathfrak{C}), z) }
\twhere z ~ \text{is fresh}

\inferrule*[right=Zero]{ }{ 0 \evaluates (\texttt{zero}(x), x) }
\twhere x ~ \text{is fresh}

\inferrule*[right=One]{ }{ 1 \evaluates (\texttt{one}(x), x) }
\twhere x ~ \text{is fresh}

\inferrule*[right=Int]{
    \overbrace{1 + 1 + \cdots + 1}^{n~\text{times}} \evaluates (\mathfrak{A}, x)
}{ n \evaluates (\mathfrak{A}, x) }
\twhere x ~ \text{is fresh}

\inferrule*[right=Func]{
    f(e_1, \ldots, e_n, x) \evaluates \mathfrak{A}
}{ f(e_1, \ldots, e_n) \evaluates (\mathfrak{A}, x) }
\twhere x ~ \text{is fresh}

\inferrule*[right=Equal]{ 
    a \evaluates (\mathfrak{A}, x)
    \and
    b \evaluates (\mathfrak{B}, y)
    \\
    V = \{ v : (e, v) \in \{ (a,x), (b,y) \}, e \neq v \}
    \\
    (x \equiv y) \evaluates \mathfrak{C}
} { a = b \evaluates \proj_V(\mathfrak{A} \land \mathfrak{B} \land \mathfrak{C}) }

\inferrule*[right=Less]{ 
    a \evaluates (\mathfrak{A}, x)
    \and
    b \evaluates (\mathfrak{B}, y)
    \\
    V = \{ v : (e, v) \in \{ (a,x), (b,y) \}, e \neq v \}
    \\
    (x < y) \evaluates \mathfrak{C}
} { a < b \evaluates \proj_V(\mathfrak{A} \land \mathfrak{B} \land \mathfrak{C}) }

\end{mathpar}

\subsubsection{Existential Quantification}

\begin{mathpar}
\inferrule*[right=Exist]{
    \tau(x) \land P \evaluates (V, \mathcal{A})
}{ \exists x \in \tau. P \evaluates (V \setminus \{x\}, \proj_x(\mathcal{A})) }
\end{mathpar}

Define $\proj_x(\mathfrak{A})$ to be $\proj_{\mathcal{V}[x]}(\mathcal{B})$, defined in the next section.

% \reed{Useless crossref atm, this is the next section...}
% For the actual implementation of the projection operation, see Section \ref{sec:impl-projection}.

\subsubsection{Projection}\label{sec:impl-projection}

Let $\mathfrak{A} = (Q, \Delta, \delta, q_0, F)$ be a B\"uchi automaton where $\Delta$ is the set of formulas involving $\land$, $\lor$, and $\lnot$ on a finite set $X$ of APs.
\reed{This is a nonstandard way to represent B\"uchi automata, but it's basically how Spot does it internally, and it's convenient for defining projection and substitution. I guess the alphabet is \emph{really} the set of equivalence classes of formulas modulo $\equiv$ such that $\varphi \equiv \psi$ if and only if $\varphi$ holds whenever $\psi$ does, but not sure that's necessary to write or interesting enough to include.}
We compute $\proj_Y(\mathcal{A}) = (Q, \Delta', \delta', q_0, F)$, where $Y$ is a finite list of APs and $\Delta'$ is the set of formulas on the APs $X \setminus Y$.
The new transition relation $\delta'$ is defined as
\[
    \delta' = \left\{ \left( s, d, \bigvee_{x \in Y} (\varphi[\top / x] \lor \varphi[\bot / x])  \right)  : (s, d, \varphi) \in \Delta' \right\}
\]
where $\varphi[a/x]$ is denotes $\varphi$ with $a$ substituted for $x$.
That is, for every transition from a state $s$ to a state $d$ on symbol $\varphi$, we have a transition from $s$ to $d$ whenever $\varphi$ holds regardless of whether any variable $x \in Y$ is true or false.

\subsubsection{Substitution}\label{sec:impl-substitution}

We define the substitution $\mathcal{A}[y/x]$, replacing $x$ by $y$ in the automaton $\mathcal{A} = (V, \mathcal{B})$ where $\mathcal{B} = (Q, \Delta, \delta, q_0, F)$ is a B\"uchi automaton as in Section~\ref{sec:impl-projection} with variables $X$ and underlying alphabet $\Sigma$.
Let $A = [a_1, \ldots, a_n]$ be the list of APs representing $x$ (i.e., $A = V[x]$), and let $B = [b_1, \ldots, b_n]$ be the list of APs representing $y$, which we assume is ambiently available.
This can be stored globally, and generated when needed if the variable $y$ has never been used before.
If $y$ has never been used before, we generate a new list of $n$ fresh APs.

Define $\mathcal{A}[y/x] := (V', \mathcal{B}')$ where $V' = (V \setminus \{ x \}) \cup \{ y \mapsto B \}$, and $\mathcal{B}' = (Q, \Delta', \delta', q_0, F)$, with the new set of variables $X' = (X \setminus A) \cup B$ and the same underlying alphabet, such that:
\[
    \Delta' = \{ \varphi[b_1/a_1, \ldots, b_n/a_n] : \varphi \in \Delta \}
\]
and
\[
    \delta' = \{ (s, d, \varphi[b_1/a_1, \ldots, b_n/a_n]) : (s, d, \varphi) \in \delta' \}
\]

\subsection{AST to IR Conversion}\label{sec:ast-to-ir}

The following conversions are performed when converting the AST to the IR:

\begin{itemize}
    \item $a \iff b$ becomes $(a \implies b) \land (b \implies a)$
    \item $a \implies b$ becomes $\lnot a \lor b$
    \item $k \cdot x$ becomes $\overbrace{x + \cdots + x}^{k \text{ times}}$
    \item $a \neq b$ becomes $\lnot (a = b)$
    \item $a > b$ becomes $b < a$
    \item $a \geq b$ becomes $b < a \lor a = b$
    \item $-a$ becomes $0 - a$
    \item $a \leq b$ becomes $a < b \lor a = b$
    \item $W[i] = W[j]$ becomes $W(i) \iff W(j)$
    \item $W[i] \neq W[j]$ becomes $\lnot (W[i] = W[j])$
    \item $W[i..j] = W[k..\ell]$ becomes $j + k = i + \ell \land \forall n \in \typ{i}. i + n < j \implies W[i + n] = W[k + n]$
    \item $W[i..j] \neq W[k..\ell]$ becomes $\lnot (W[i..j] = W[k..\ell])$
    \item $\forall x_1 \in \tau_1, \ldots, x_n \in \tau_n. P$ becomes $\lnot (\exists x_1 \in \tau_1, \ldots, x_n \in \tau_n. \lnot P)$
    \item $\exists x_1 \in \tau_1, \ldots, x_n \in \tau_n. P$ becomes $\exists x_1 \in \tau_1. \ldots. \exists x_n \in \tau_n. P$
\end{itemize}

where $\typ{i}$ denotes the type of $i$.

\section{Evaluation}\label{sec:evaluation}

We evaluate the performance of Pecan by generating automata for fundamental definitions in the field of combinatorics on words and proving theorems about Sturmian words using these definitions.
We consider \term{characteristic} Sturmian words, i.e., where the intercept is $0$, which we write $c_{\alpha} = \mathbf{w}_{\alpha,0}$; all definitions are parameterized by the slope of Sturmian word.
To our knowledge, there are no other tools to which Pecan can be directly compared.
Our results indicate that our approach is practical, as we are able to prove many interesting theorems using only an ordinary computer.
There is not space to discuss the definitions and theorems encoded, but our repository contains the complete code~\cite{sturmian-words-repo}.

We record several metrics for each predicate: the number of \term{atoms}, how many \term{alternating quantifier blocks} it contains (i.e., alternating universal and existential quantifiers), the runtime in seconds, the number of states and edges in the intermediate automaton with the greatest number of states, and the final number of states and edges, when applicable.
Alternating quantifier blocks increase the runtime due to the encoding of $\forall x. P(x)$ as $\lnot(\exists x. \lnot P(x))$, as complementing B\"uchi automata has a very poor worst-case complexity of at least $\Omega((0.76n)^n)$~\cite{Yan2008}.
We write these blocks as $\forall^{n_1}\exists^{n_2}\forall^{n_3}\ldots$.
% If the entire definition is quantifier-free, we write $\mathbf{QF}$.
Quantifiers range over countable domains unless otherwise noted; $\forall_\R$ and $\exists_\R$ are quantifiers ranging over domains of cardinality $|\R|$. %; states tend to have a larger effect on performance than edges, but we include edges for completeness.

As an example of computing these metrics, consider the following definition.
% \begin{definition}
%     A word $w \in \{0,1\}^\omega$ is \term{eventually periodic} if there is some $p > 0$, called the \term{period} and $n > 0$ such that for all $i > n$, we have $w_i = w_{i+p}$.
% \end{definition}
% In Pecan, we can define this for Sturmian words as follows.
% Note that \pecaninline{ostrowski(a)} specifies the numeration system for the variables \pecaninline{i}, \pecaninline{p}, and \pecaninline{n} which we need so that Sturmian words are automatic sequences.
% \begin{pecan}
% Restrict i,p,n are ostrowski(a).
% eventually_periodic(a, p) := 
%   p > 0 & existsn. foralli. i > n => $\$$C[i] = $\$$C[i + p]
% \end{pecan}
% This is expanded by Pecan into:
% \begin{pecan}
% eventually_periodic(a,p) := 
%   (existsz. ostrowski(a,z) & zero(z) & less(z,p)) & 
%   existsn. ostrowski(a,n) & 
%     !(existsi. ostrowski(a,i) & less(n,i) & 
%       (existsip. adder(i,p,ip) & $\$$C(i) & !$\$$C(ip)) | 
%       (existsip. adder(i,p,ip) & !$\$$C(i) & $\$$C(ip)))
% \end{pecan}
% From this, we can see that \pecaninline{eventually_periodic} has $12$ atoms and has complexity $\exists^2 \forall \exists^2$ in prenex normal form.
\begin{definition}
    A factor $x$ of a word $w$ is \term{special} if $x0$ and $x1$ are factors of $w$.
\end{definition}
In Pecan, we can define this for Sturmian words as follows.
\vspace{-0.5em}
\begin{pecan}
Restrict a is bco_standard.
Restrict i,j,k,n are ostrowski(a).
special_factor(a,i,n) :=
    (existsj. factor_lt_len(a,i,n,j) & $\$$C[j+n] = 0) &
    (existsk. factor_lt_len(a,i,n,k) & $\$$C[k+n] = 1)
\end{pecan}
\vspace{-0.5em}
The numeric structure \pecaninline{ostrowski(a)} specifies the numeration system for the variables \pecaninline{i}, \pecaninline{j}, \pecaninline{k} and \pecaninline{n} making Sturmian words into automatic sequences, and \lstinline[mathescape, language=pecan, basicstyle=\small\ttfamily]!$\$$C[i]! denotes the $i$-th letter of the Sturmian word---the same automaton works for every slope.
Pecan expands this to:
\vspace{-0.5em}
\begin{pecan}
special_factor(a,i,n) :=
  (existsj. ostrowski(a,j) & factor_lt_len(a,i,n,j) & 
    !(existsv0. adder(j,n,v0) & $\$$C(v0))) & 
  (existsk. ostrowski(a,k) & factor_lt_len(a,i,n,k) & 
     (existsv1. adder(k,n,v1) & $\$$C(v1)))
\end{pecan}
\vspace{-0.5em}
We can see that \pecaninline{special_factor} has $8$ atoms and has complexity $\exists^3\forall$ in prenex normal form.
Here, \pecaninline{factor_lt_len(a,i,n,j)} means $c_{a}[i..i+n] = c_{a}[j..j+n]$.

\begin{figure}
    \centering
    \hspace{-3em}
    \vspace{-1em}
    \footnotesize
    \begin{tabular}{l|l|r|r|r|r|r|r|r|}
        & & & & \multicolumn{2}{c}{Max} & \multicolumn{2}{c}{Final} \\
        Name & Complexity & Atoms & Runtime (s) & States & Edges & States & Edges \\ \hline
Mirror invariant & $\exists$ & $1$ & $8.1$ & $1440$ & $16840$ & $1129$ & $9666$ \\
Unbordered & $\exists^3$ & $2$ & $0.5$ & $275$ & $1156$ & $92$ & $410$ \\
Cube & $\exists$ & $4$ & $0.7$ & $936$ & $5956$ & $126$ & $561$ \\
Least period & $\forall$ & $4$ & $2605.2$ & $352577$ & $6098198$ & $577$ & $4161$ \\
Max unbordered subfactor & $\forall$ & $4$ & $26.4$ & $25200$ & $196575$ & $585$ & $4345$ \\
Palindrome & $\exists^2$ & $4$ & $5.1$ & $1934$ & $12337$ & $922$ & $6274$ \\
Period & $\exists^2$ & $5$ & $64.1$ & $5853$ & $103886$ & $1660$ & $17570$ \\
Recurrent & $\forall\exists$ & $5$ & $272.6$ & $61713$ & $960207$ & $34$ & $212$ \\
Special factor & $\exists^3\forall$ & $8$ & $1361.8$ & $17738$ & $103274$ & $4594$ & $25349$ \\
Factor Lt (idx) & $\exists \forall^2$ & $11$ & $702.7$ & $1057221$ & $22348882$ & $2204$ & $25026$ \\
Eventually periodic & $\exists^2\forall\exists^2$ & $12$ & $216.6$ & $78338$ & $1001075$ & $1$ & $0$ \\
Reverse factor & $\exists \forall^2$ & $12$ & $842.0$ & $1408050$ & $22780414$ & $1440$ & $16840$ \\
Antipalindrome & $\exists^2\forall^3$ & $13$ & $242.2$ & $78396$ & $1668960$ & $200$ & $834$ \\
Antisquare & $\forall^3$ & $13$ & $1844.3$ & $2542937$ & $31570114$ & $136$ & $539$ \\
Square & $\forall^3$ & $13$ & $2138.0$ & $1908657$ & $23683717$ & $155$ & $747$ \\
$(01)^*|(10)^*$ & $\forall$ & $16$ & $77.9$ & $5409$ & $72739$ & $103$ & $456$ \\
    \end{tabular}
    \vspace{-0.5em}
    \caption{Common definitions about Sturmian words.}
    \vspace{-2em}
    \label{fig:def-performance-table}
\end{figure}

Figure~\ref{fig:def-performance-table} shows performance statistics for creating the automata representing various common definitions in Pecan.
% Overall, the number of atoms and the complexity predicts the runtime, with some notable exceptions, like Least period and Special Factor---these are defined in terms of relatively large automata (Period and Factor Lt, respectively).
The automaton for Eventually Periodic is empty because of the classic result that there are no Sturmian words that are eventually periodic.
One might guess that Cube would be more expensive than Square; however, we can define Cube very efficiently in terms of Square.
The same is true for higher powers, as well as many other predicates: for example, both Mirror Invariant and Palindrome are relatively easy to compute, as they are straighforwardly defined using Reverse Factor.
We can see that, even though the automata often become quite large (e.g., having over $2$ million states in the case of Antisquare), we are still able to handle them relatively easily.

\begin{figure}
    \centering
    \footnotesize
    \vspace{-1em}
    \begin{tabular}{l|r|r|r|r|r|r|}
        & & & & \multicolumn{2}{c}{Avg Max} \\
        Complexity & Atoms & Number & Avg Runtime (sec.) & States & Edges \\ \hline
$\forall_\R\exists$ & $3$ & $1$ & $0.0$ & $12.0$ & $38.0$ \\
$\forall_\R\exists^2$ & $3$ & $1$ & $0.1$ & $868.0$ & $5107.0$ \\
$\exists_\R\exists^2$ & $4$ & $2$ & $0.0$ & $53.0$ & $124.0$ \\
$\forall_\R\forall$ & $4$ & $1$ & $0.1$ & $130.0$ & $516.0$ \\
$\forall_\R\exists^2$ & $5$ & $1$ & $0.1$ & $399.0$ & $2053.0$ \\
$\forall_\R\forall^2$ & $5$ & $2$ & $0.2$ & $146.5$ & $603.0$ \\
$\forall_\R\exists\forall^2$ & $6$ & $2$ & $0.2$ & $598.5$ & $3789.5$ \\
$\forall_\R\exists^2$ & $6$ & $1$ & $0.1$ & $812.0$ & $4770.0$ \\
$\forall_\R\forall\exists$ & $6$ & $1$ & $7.1$ & $1328.0$ & $8985.0$ \\
$\forall_\R\exists^3$ & $7$ & $2$ & $0.3$ & $119.5$ & $271.5$ \\
$\forall_\R\forall\exists^2$ & $7$ & $1$ & $0.0$ & $593.0$ & $3355.0$ \\
$\forall_\R\forall^2$ & $7$ & $3$ & $9.8$ & $1746.0$ & $15430.3$ \\
$\forall_\R\forall^3$ & $7$ & $1$ & $0.1$ & $155.0$ & $1497.0$ \\
$\forall_\R\forall\exists^2$ & $8$ & $1$ & $1.4$ & $922.0$ & $6274.0$ \\
$\forall_\R\forall^2$ & $9$ & $2$ & $0.1$ & $178.0$ & $848.5$ \\
$\forall_\R\forall^2\exists$ & $10$ & $1$ & $0.2$ & $1440.0$ & $16840.0$ \\
$\forall_\R\forall\exists\forall\exists\forall\exists$ & $17$ & $1$ & $3.3$ & $6106.0$ & $46025.0$ \\
$\forall_\R\forall^4\exists^4$ & $18$ & $1$ & $156.6$ & $2032240.0$ & $47851215.0$ \\
$\forall_\R\forall\exists^3\forall^2$ & $22$ & $1$ & $489.8$ & $138223.0$ & $3834628.0$ \\
    \end{tabular}
    \vspace{-0.5em}
    \caption{Theorems about Sturmian words, grouped by complexity and number of atoms.
        Number column shows how many theorems are in each group.
        Theorems evaluate to single state automata, so we omit the data about the final automaton.}
    \vspace{-2em}
    \label{fig:thm-performance-table}
\end{figure}

Figure~\ref{fig:thm-performance-table} shows performance statistics for proving theorems about Sturmian words in Pecan.
These theorems are a mix of classical results, known theorems, and some new results we proved using Pecan, described in \cite{DecStuWor}.
% Again we can see that the number of atoms and complexity roughly predict the runtime.
Overall, our results show that Pecan is a viable theorem proving tool for Sturmian words, and we hypothesize it will also be useful for other B\"uchi-automatic sequences.

%\section{Related Work}\label{sec:related-work}

\reed{talk about walnut here/anything esle relevant? probably mention the fibonacci word paper since we do so much of the same stuff}

\reed{might be worth talking about SPIN?}

\section{Conclusion and Future Work}\label{sec:conclusion}

We presented Pecan, the first system, to our knowledge, implementing a general purpose decision procedure for B\"uchi-automatic sequences, and in particular, statements about Sturmian words.
The system aims to be a convenient interface for specifying definitions and proving theorems about such sequences, with features such as custom numeration systems, enabled by the type system, and convenient syntax for indexing into automatic sequences.
We provide a formal description of the system, then evaluate the performance of Pecan by building automata representing common definitions and proving theorems about Sturmian words.
We show that Pecan has reasonable performance, despite the theoretical worst-case, indicating that the approach is practical.
% We show that Pecan has reasonable performance, despite the theoretical running time of the decision procedure and the size of the intermediate automata, indicating that the approach is practical.

In the future, we hope to expand the statements that Pecan is capable of handling, by integrating known extensions such as multiplication by quadratic irrationals, as described in~\cite{hieronymi2019presburger}.
We also hope to continue using Pecan to prove theorems about Sturmian words, both to provide new proofs of old results, as well as proving more new results.
It may also be interesting to support more expressive kinds of automata that still have the desired closure properties, such as ($\omega$-)operator precedence automata~\cite{panella2013operator}.
We would also like to integrate Pecan into a general-purpose proof assistant, such as Isabelle or Lean~\cite{de2015lean}. %, so that it can be used an external solver.% to expand the fully automated proving capabilities of these systems.


\bibliographystyle{splncs04}
\bibliography{biblio.bib}

\end{document}
