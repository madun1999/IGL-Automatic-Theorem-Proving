\section{Evaluation}\label{sec:evaluation}

We evaluate the performance of Pecan by generating automata for fundamental definitions in the field of combinatorics on words and proving theorems about Sturmian words using these definitions.
We consider \term{characteristic} Sturmian words, i.e., where the intercept is $0$, which we write $c_{\alpha} = \mathbf{w}_{\alpha,0}$; all definitions are parameterized by the slope of Sturmian word.
To our knowledge, there are no other tools to which Pecan can be directly compared.
Our results indicate that our approach is practical, as we are able to prove many interesting theorems using only an ordinary computer.
There is not space to discuss the definitions and theorems encoded, but our repository contains the complete code~\cite{sturmian-words-repo}.

We record several metrics for each predicate: the number of \term{atoms}, how many \term{alternating quantifier blocks} it contains (i.e., alternating universal and existential quantifiers), the runtime in seconds, the number of states and edges in the intermediate automaton with the greatest number of states, and the final number of states and edges, when applicable.
Alternating quantifier blocks increase the runtime due to the encoding of $\forall x. P(x)$ as $\lnot(\exists x. \lnot P(x))$, as complementing B\"uchi automata has a very poor worst-case complexity of at least $\Omega((0.76n)^n)$~\cite{Yan2008}.
We write these blocks as $\forall^{n_1}\exists^{n_2}\forall^{n_3}\ldots$.
% If the entire definition is quantifier-free, we write $\mathbf{QF}$.
Quantifiers range over countable domains unless otherwise noted; $\forall_\R$ and $\exists_\R$ are quantifiers ranging over domains of cardinality $|\R|$. %; states tend to have a larger effect on performance than edges, but we include edges for completeness.

As an example of computing these metrics, consider the following definition.
% \begin{definition}
%     A word $w \in \{0,1\}^\omega$ is \term{eventually periodic} if there is some $p > 0$, called the \term{period} and $n > 0$ such that for all $i > n$, we have $w_i = w_{i+p}$.
% \end{definition}
% In Pecan, we can define this for Sturmian words as follows.
% Note that \pecaninline{ostrowski(a)} specifies the numeration system for the variables \pecaninline{i}, \pecaninline{p}, and \pecaninline{n} which we need so that Sturmian words are automatic sequences.
% \begin{pecan}
% Restrict i,p,n are ostrowski(a).
% eventually_periodic(a, p) := 
%   p > 0 & existsn. foralli. i > n => $\$$C[i] = $\$$C[i + p]
% \end{pecan}
% This is expanded by Pecan into:
% \begin{pecan}
% eventually_periodic(a,p) := 
%   (existsz. ostrowski(a,z) & zero(z) & less(z,p)) & 
%   existsn. ostrowski(a,n) & 
%     !(existsi. ostrowski(a,i) & less(n,i) & 
%       (existsip. adder(i,p,ip) & $\$$C(i) & !$\$$C(ip)) | 
%       (existsip. adder(i,p,ip) & !$\$$C(i) & $\$$C(ip)))
% \end{pecan}
% From this, we can see that \pecaninline{eventually_periodic} has $12$ atoms and has complexity $\exists^2 \forall \exists^2$ in prenex normal form.
\begin{definition}
    A factor $x$ of a word $w$ is \term{special} if $x0$ and $x1$ are factors of $w$.
\end{definition}
In Pecan, we can define this for Sturmian words as follows.
\vspace{-0.5em}
\begin{pecan}
Restrict a is bco_standard.
Restrict i,j,k,n are ostrowski(a).
special_factor(a,i,n) :=
    (existsj. factor_lt_len(a,i,n,j) & $\$$C[j+n] = 0) &
    (existsk. factor_lt_len(a,i,n,k) & $\$$C[k+n] = 1)
\end{pecan}
\vspace{-0.5em}
The numeric structure \pecaninline{ostrowski(a)} specifies the numeration system for the variables \pecaninline{i}, \pecaninline{j}, \pecaninline{k} and \pecaninline{n} making Sturmian words into automatic sequences, and \lstinline[mathescape, language=pecan, basicstyle=\small\ttfamily]!$\$$C[i]! denotes the $i$-th letter of the Sturmian word---the same automaton works for every slope.
Pecan expands this to:
\vspace{-0.5em}
\begin{pecan}
special_factor(a,i,n) :=
  (existsj. ostrowski(a,j) & factor_lt_len(a,i,n,j) & 
    !(existsv0. adder(j,n,v0) & $\$$C(v0))) & 
  (existsk. ostrowski(a,k) & factor_lt_len(a,i,n,k) & 
     (existsv1. adder(k,n,v1) & $\$$C(v1)))
\end{pecan}
\vspace{-0.5em}
We can see that \pecaninline{special_factor} has $8$ atoms and has complexity $\exists^3\forall$ in prenex normal form.
Here, \pecaninline{factor_lt_len(a,i,n,j)} means $c_{a}[i..i+n] = c_{a}[j..j+n]$.

\begin{figure}
    \centering
    \hspace{-3em}
    \vspace{-1em}
    \footnotesize
    \begin{tabular}{l|l|r|r|r|r|r|r|r|}
        & & & & \multicolumn{2}{c}{Max} & \multicolumn{2}{c}{Final} \\
        Name & Complexity & Atoms & Runtime (s) & States & Edges & States & Edges \\ \hline
Mirror invariant & $\exists$ & $1$ & $8.1$ & $1440$ & $16840$ & $1129$ & $9666$ \\
Unbordered & $\exists^3$ & $2$ & $0.5$ & $275$ & $1156$ & $92$ & $410$ \\
Cube & $\exists$ & $4$ & $0.7$ & $936$ & $5956$ & $126$ & $561$ \\
Least period & $\forall$ & $4$ & $2605.2$ & $352577$ & $6098198$ & $577$ & $4161$ \\
Max unbordered subfactor & $\forall$ & $4$ & $26.4$ & $25200$ & $196575$ & $585$ & $4345$ \\
Palindrome & $\exists^2$ & $4$ & $5.1$ & $1934$ & $12337$ & $922$ & $6274$ \\
Period & $\exists^2$ & $5$ & $64.1$ & $5853$ & $103886$ & $1660$ & $17570$ \\
Recurrent & $\forall\exists$ & $5$ & $272.6$ & $61713$ & $960207$ & $34$ & $212$ \\
Special factor & $\exists^3\forall$ & $8$ & $1361.8$ & $17738$ & $103274$ & $4594$ & $25349$ \\
Factor Lt (idx) & $\exists \forall^2$ & $11$ & $702.7$ & $1057221$ & $22348882$ & $2204$ & $25026$ \\
Eventually periodic & $\exists^2\forall\exists^2$ & $12$ & $216.6$ & $78338$ & $1001075$ & $1$ & $0$ \\
Reverse factor & $\exists \forall^2$ & $12$ & $842.0$ & $1408050$ & $22780414$ & $1440$ & $16840$ \\
Antipalindrome & $\exists^2\forall^3$ & $13$ & $242.2$ & $78396$ & $1668960$ & $200$ & $834$ \\
Antisquare & $\forall^3$ & $13$ & $1844.3$ & $2542937$ & $31570114$ & $136$ & $539$ \\
Square & $\forall^3$ & $13$ & $2138.0$ & $1908657$ & $23683717$ & $155$ & $747$ \\
$(01)^*|(10)^*$ & $\forall$ & $16$ & $77.9$ & $5409$ & $72739$ & $103$ & $456$ \\
    \end{tabular}
    \vspace{-0.5em}
    \caption{Common definitions about Sturmian words.}
    \vspace{-2em}
    \label{fig:def-performance-table}
\end{figure}

Figure~\ref{fig:def-performance-table} shows performance statistics for creating the automata representing various common definitions in Pecan.
% Overall, the number of atoms and the complexity predicts the runtime, with some notable exceptions, like Least period and Special Factor---these are defined in terms of relatively large automata (Period and Factor Lt, respectively).
The automaton for Eventually Periodic is empty because of the classic result that there are no Sturmian words that are eventually periodic.
One might guess that Cube would be more expensive than Square; however, we can define Cube very efficiently in terms of Square.
The same is true for higher powers, as well as many other predicates: for example, both Mirror Invariant and Palindrome are relatively easy to compute, as they are straighforwardly defined using Reverse Factor.
We can see that, even though the automata often become quite large (e.g., having over $2$ million states in the case of Antisquare), we are still able to handle them relatively easily.

\begin{figure}
    \centering
    \footnotesize
    \vspace{-1em}
    \begin{tabular}{l|r|r|r|r|r|r|}
        & & & & \multicolumn{2}{c}{Avg Max} \\
        Complexity & Atoms & Number & Avg Runtime (sec.) & States & Edges \\ \hline
$\forall_\R\exists$ & $3$ & $1$ & $0.0$ & $12.0$ & $38.0$ \\
$\forall_\R\exists^2$ & $3$ & $1$ & $0.1$ & $868.0$ & $5107.0$ \\
$\exists_\R\exists^2$ & $4$ & $2$ & $0.0$ & $53.0$ & $124.0$ \\
$\forall_\R\forall$ & $4$ & $1$ & $0.1$ & $130.0$ & $516.0$ \\
$\forall_\R\exists^2$ & $5$ & $1$ & $0.1$ & $399.0$ & $2053.0$ \\
$\forall_\R\forall^2$ & $5$ & $2$ & $0.2$ & $146.5$ & $603.0$ \\
$\forall_\R\exists\forall^2$ & $6$ & $2$ & $0.2$ & $598.5$ & $3789.5$ \\
$\forall_\R\exists^2$ & $6$ & $1$ & $0.1$ & $812.0$ & $4770.0$ \\
$\forall_\R\forall\exists$ & $6$ & $1$ & $7.1$ & $1328.0$ & $8985.0$ \\
$\forall_\R\exists^3$ & $7$ & $2$ & $0.3$ & $119.5$ & $271.5$ \\
$\forall_\R\forall\exists^2$ & $7$ & $1$ & $0.0$ & $593.0$ & $3355.0$ \\
$\forall_\R\forall^2$ & $7$ & $3$ & $9.8$ & $1746.0$ & $15430.3$ \\
$\forall_\R\forall^3$ & $7$ & $1$ & $0.1$ & $155.0$ & $1497.0$ \\
$\forall_\R\forall\exists^2$ & $8$ & $1$ & $1.4$ & $922.0$ & $6274.0$ \\
$\forall_\R\forall^2$ & $9$ & $2$ & $0.1$ & $178.0$ & $848.5$ \\
$\forall_\R\forall^2\exists$ & $10$ & $1$ & $0.2$ & $1440.0$ & $16840.0$ \\
$\forall_\R\forall\exists\forall\exists\forall\exists$ & $17$ & $1$ & $3.3$ & $6106.0$ & $46025.0$ \\
$\forall_\R\forall^4\exists^4$ & $18$ & $1$ & $156.6$ & $2032240.0$ & $47851215.0$ \\
$\forall_\R\forall\exists^3\forall^2$ & $22$ & $1$ & $489.8$ & $138223.0$ & $3834628.0$ \\
    \end{tabular}
    \vspace{-0.5em}
    \caption{Theorems about Sturmian words, grouped by complexity and number of atoms.
        Number column shows how many theorems are in each group.
        Theorems evaluate to single state automata, so we omit the data about the final automaton.}
    \vspace{-2em}
    \label{fig:thm-performance-table}
\end{figure}

Figure~\ref{fig:thm-performance-table} shows performance statistics for proving theorems about Sturmian words in Pecan.
These theorems are a mix of classical results, known theorems, and some new results we proved using Pecan, described in \cite{DecStuWor}.
% Again we can see that the number of atoms and complexity roughly predict the runtime.
Overall, our results show that Pecan is a viable theorem proving tool for Sturmian words, and we hypothesize it will also be useful for other B\"uchi-automatic sequences.
