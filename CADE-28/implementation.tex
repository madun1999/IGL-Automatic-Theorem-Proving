\section{Implementation}\label{sec:implementation}

This section describes the high-level the implementation of Pecan.
% First, we describe the syntax in Section~\ref{sec:ir-syntax}.
We give a formal definition of the Pecan language, starting with the typing rules and associated definitions in Section~\ref{sec:typing}, and then the rules for evaluation in Section~\ref{sec:evaluation-rules}.

% The process of executing a program is roughly as follows:

% \begin{enumerate}
%     \item Parse the program into an AST (Abstract Syntax Tree).
%     \item Transform the program's AST into a simplified IR  representation, rewriting constructs such as $\iff$ in terms of simpler ones, like $\land$, $\lor$, and $\lnot$.
%     \item Load the standard library (if desired); loading the standard library requires going through all of the steps again, starting from step 1, but skipping this step.
%     \item Then, for each definition or directive, perform the following steps:
%         \begin{enumerate}
%             \item Run the untyped optimizer, if enabled.
%             \item Perform type inference.
%             \item Perform a final IR lowering.
%             \item Run the typed optimizer, if enabled.
%             \item Interpret the final IR of the current top-level construct.
%         \end{enumerate}
% \end{enumerate}

% \subsection{Syntax}\label{sec:ir-syntax}

Figure~\ref{fig:syntax} is the syntax for the core of the Pecan language.
Pecan also supports some simple syntactic sugar, such as \pecaninline{if P then Q}, which expands into \pecaninline{!P |$~$Q}, or \pecaninline{n*x} for some literal number \pecaninline{n}, which expands into \pecaninline{x+x+x+...+x} with \pecaninline{n} repetitions.
\begin{figure}
    \centering
    \vspace{-4em}
    \begin{align*}
        P,f &\in \textsc{PredicateNames} & V &\in \textsc{VariableMaps} & a,x,y,z &\in \textsc{Identifiers} \\ 
        n,m &\in \N & \mathfrak{A} &\in \textsc{Automata} \\
    \end{align*}
    \vspace{-1em}
    \begin{tabular}{l r l}
         Prog & \bnfdef & $\overline{\text{Definition}}$ \\
         $\tau$ & \bnfdef & $P$ \bnfalt $P(\overline{x})$ \\
         Definition & \bnfdef & $P(\overline{x : \tau})$ := Pred \\
         & \bnfalt & Restrict $\overline{x}$ are $P(\overline{x})$ \\
         Pred & \bnfdef & true \bnfalt false \bnfalt Pred $\lor$ Pred \bnfalt $\lnot$ Pred \bnfalt Pred $\land$ Pred \\
         & \bnfalt & $\exists x. $ Pred \bnfalt E $<$ E \bnfalt E $=$ E \bnfalt $P(\overline{E})$ \bnfalt $\text{Aut}(V, \mathfrak{A})$ \\
         E & \bnfdef & E + E \bnfalt E - E \bnfalt $x$ \bnfalt $n$ \bnfalt $f(\overline{E})$ 
    \end{tabular}
    \caption{Syntax of the core of Pecan.}
    \vspace{-1em}
    \label{fig:syntax}
\end{figure}

\subsection{Type Checking}\label{sec:typing}

A \term{type} in Pecan is represented by a B\"uchi automaton.
We say that $x : \tau$ when $x \in L(\tau)$, sometimes simply written, as an analogy to logical predicates, as $\tau(x)$.
Types may be \term{partially applied}.
For example, if $\tau = P(x_1, \ldots, x_n)$, where $P$ is some B\"uchi automaton, then $y : \tau$ when $(x_1, \ldots, x_n, y) \in L(P)$; and $\tau(y)$ holds when $y : \tau$.
In the concrete syntax of Pecan, we write \pecaninline{y is tau} or \pecaninline{y} $\in$ \pecaninline{tau}; for one or more variables, we can write \pecaninline{x, y, z are tau} to mean $x : \tau$, $y : \tau$, $z : \tau$.
% Finally, there is a special type called $\inferred$, which should be thought of as a type that will be discovered later by how the value is used; or, as its name suggests, the type will eventually be \textbf{inferred} from context.
% $L(\inferred)$ is undefined---$\inferred$ is NOT represented by an automata.

The judgement $\Gamma \proves x : \tau$ means that we can prove $\tau(x)$ is true in the type environment $\Gamma$, consisting of pairs $x : \tau$.
We write the \term{domain} of $\Gamma$ as $\dom{\Gamma}$.
The judgement $\Gamma \proves \prop{P}$ means that $P$ is a \emph{well-formed proposition} in the environment $\Gamma$.
A predicate $P(\overline{x : \tau}) := Q$ is \emph{well-formed} when $\overline{x : \tau} \proves \prop{Q}$.
% We check a sequence of Pecan predicates in order, starting with the empty environment $\Gamma = \emptyset$.
Below, we assume that the set of all well-formed predicates, which have already been checked, is ambiently available as $\mathcal{P}$.
% Whenever we encounter a restriction, \pecaninline{Restrict x1,...,xn are P(y1,...yn)}, every mention of Restrict $x_1, \ldots, x_n$ are $P(y_1, \ldots, y_m)$, we update the current environment $\Gamma$ to be $\Gamma ~\cup~ \{ x_1 : P(y_1, \ldots, y_m), \ldots, x_n : P(y_1, \ldots, y_m, x_n) \} $; it is only allowed to use $P \in \mathcal{P}$, that is, predicates which have already been confirmed are well-formed.

\paragraph{Structures.}
In order to support ad-hoc polymorphism, Pecan allows the definition and use of \term{structures}.
This feature facilitates the use of nicer syntax for arithmetic expressions (e.g., $x + (y + z) = w$ instead of $\exists t. \texttt{adder}(x, y, t) \land \texttt{adder}(t, z, w)$) without tying ourselves to a single numeration system.
For example, $\texttt{adder}$ will be resolved to some concrete predicate predicate based on the type of $x$, $y$, and $z$.
% The exact rules and definitions related to this feature are given below.
We assume structure definitions are ambiently available in the program.

\begin{definition}
    A \term{structure} is a pair $(t(\overline{x}), D)$ where $\overline{x}$ are identifiers and $D$ is a map of identifiers to \term{call templates} of the form $f(\overline{y})$, where for each $\overline{y} \subseteq \overline{x} \cup \{*\}$; $*$ denotes ``any.''
    The \term{name} of the structure is $t$.
\end{definition}

We write the sequence of indexes of the arguments that are $*$, called \term{parameters}, as $\params(f(\overline{y}))$.
A call template is $n$-ary if $|\params(f(\overline{y}))| = n$.
The sequence of the indexes of the other arguments (i.e., not $*$), called \term{implicits}, is $\implicits(f(\overline{y}))$.
For example, $\params(f(a, *, b, *)) = [2, 4]$ and $\implicits(f(a, *, b, *)) = [1, 3]$.
% We write $\params(f(\overline{y}))[i]$ (resp. $\implicits(f(\overline{y}))[i]$) to denote the $i$-th parameter (resp. implicit).
We assume that typechecking has been done before evaluating, because we may need structure information at runtime to resolve \term{dynamic calls}, that is, calls whose name matches some definition inside a structure.
We denote the type that an expression $e$ got when typechecking by $\typ{e}$.

We write $t[P] = Q(y_1, \ldots, y_m)$ to look up a definition in the associated map $D$, and we say that $t$ \term{has an $m$-ary definition} for $P$ in this case.
If $t$ does not have a definition for $P$, then we write $t[P] = \bot$.

\begin{definition}
    A structure is called \term{numeric} if it has a ternary definition for $\texttt{adder}$ and a binary definition $\texttt{less}$.
    We write $x + y = z$ when $\texttt{adder}(x, y, z)$ holds and $x < y$ when $\texttt{less}(x, y)$ holds.
    A numeric structure may also optionally contain the following definitions:
    \begin{itemize}
        \item A binary definition $\texttt{equal}(x,y)$, written $x \equiv y$.
            The default is equality, $x = y$.
        
        \item A unary definition $\texttt{zero}(z)$.
            The default is $z = 0^{\omega}$.
            
        \item A unary definition $\texttt{one}(x)$.
            The default is $0 \leq x \land \forall y. y = 0 \lor x \leq y$.
    \end{itemize}
\end{definition}

% \reed{Could put this back (either all of it, or without inferred; without inferred, it's still nice ebcause it gives us some subtyping, but that's not strictly necessary)}
% \begin{definition}
%     The \emph{predecessor} of two types $\tau$ and $\sigma$, written $\tau \join \sigma$ is given by following partial function:
%     \[
%         \tau \join \sigma = 
%         \begin{cases}
%             \sigma & \tif \tau = \inferred~\text{and $\sigma$ is numeric} \\
%             \sigma & \tif \forall \Free(\tau). \forall x. \tau(x) \implies \sigma(x) \\
%             \tau & \tif \forall \Free(\tau). \forall x. \sigma(x) \implies \tau(x) \\
%             \tau & \tif \sigma = \inferred~\text{and $\tau$ is numeric} \\
%         \end{cases}
%     \]
% \end{definition}

\begin{definition}
    We can \term{resolve} a call $P(\overline{e})$ as $Q(\overline{a : \tau})$, written $P(\overline{e}) \leadsto Q(\overline{a : \tau})$, if $Q(\overline{a : \tau}) \in \mathcal{P}$ and for some structure $t(x_1, \ldots, x_\ell)$, for each $1 \leq i \leq |\overline{e}|$, either:
    \begin{enumerate}
        \item $\typ{e_i} = t(x_1, \ldots, x_\ell)$, and $t[P] = Q(b_1, \ldots, b_m)$ such that
        for each $1 \leq j \leq m$,
        \[
            a_j =
            \begin{cases}
                x_k & \tif \implicits(Q(b_1, \ldots, b_m))[k] = j \\
                e_k & \tif \params(Q(b_1, \ldots, b_m))[k] = j
            \end{cases}
        \]
        
        \item $\typ{e_i} = s(y_1, \ldots, y_p)$, where $s \neq t$ and $s[P] = \bot$.
    \end{enumerate}
    
    or, if none of the arguments have a definition for $P$, then $P(\overline{e}) \leadsto P(\overline{a : \tau})$.
\end{definition}

\framebox{$\Gamma \proves e : \tau$} \textbf{Expression Typing}
The expression typing rules are standard, with the exception that we can treat any predicate as a function by not writing it's last argument (e.g., $f(x)$ denotes the $y$ such that $f(x,y)$ holds).
% All integer constants are of type \inferred until used in some position that determines what their type is; e.g., $1 : \inferred$, but if $x : \tau$ for some numeric type $\tau$, then $1 + x : \tau$.
\begin{mathpar}
\small
\inferrule*[right=Int]{
    \tau~\text{is numeric}
} { \Gamma \proves i : \tau }

\inferrule*[right=Var]{
    x : \tau \in \Gamma
} { \Gamma \proves x : \tau }

\inferrule*[right=Op]{ 
    \Gamma \proves a : \tau
    \\ 
    \Gamma \proves b : \tau
    \\
    \oplus \in \{ +, - \}
}{ \Gamma \proves a \oplus b : \tau }

\inferrule*[right=Func]{
    \Gamma \proves \overline{e : \tau}
    \\
    (f(\overline{x : \tau}, r : \sigma) := e') \in \mathcal{P}
}{ \Gamma \proves f(\overline{e}) : \sigma }
\end{mathpar}

\framebox{$\Gamma \proves \prop{P}$} \textbf{Well-formed Propositions}
% The rules for well-formed propositions are also straightforward.
% We require that types are \term{concrete} (i.e., \textbf{not} \inferred) at this point, so that during execution, we are able to resolve the correct operations for them.
\begin{mathpar}
\small
\inferrule*[right=Rel]{ 
    \Gamma \proves a : \tau
    \\
    \Gamma \proves b : \tau
    \\
    \Join \in \{ \equiv, < \}
}{ \Gamma \proves \prop{a \Join b} }

\inferrule*[right=Aut]{
}{ \Gamma \proves \prop{\text{Aut}(V, \mathcal{A})} }

\inferrule*[right=BinPred]{
    \Gamma \proves \prop{P}
    \\
    \Gamma \proves \prop{Q}
    \\
    \oplus \in \{ \lor, \land \}
}{ \Gamma \proves \prop{P \oplus Q} }

\inferrule*[right=Comp]{
    \Gamma \proves \prop{P}
}{ \Gamma \proves \prop{\lnot P} }

\inferrule*[right=Exists]{
    \Gamma, x : \tau \proves \prop{P}
}{ \Gamma \proves \exists x : \tau. \prop{P} }

\inferrule*[right=Call]{
    \Gamma \proves \overline{e : \tau}
    \\
    (f(\overline{x : \tau}) := e') \in \mathcal{P}
}{ \Gamma \proves \prop{f(\overline{e})} }

% \inferrule*[right=Annotation]{
%     \Gamma \proves \prop{P} 
% }{ \Gamma \proves \prop{@A[P]} }
\end{mathpar}

\subsection{Evaluation}\label{sec:evaluation-rules}

Pecan is a simple tree-walking interpreter written in Python 3~\cite{python3} which typechecks and processes each top-level construct in order.
Operations with non-trivial implementations are described in detail below.
Most basic automata operations (e.g., conjunction, disjunction, complementation, emptiness checking, simplification) are implemented using the Spot library \cite{duret.16.atva2}. %, so details of these algorithms are not presented here.

\textbf{Automata Representation}
Automata are represented by a pair of $(V, \mathcal{A})$, where $V$ is a map taking variable names to an ordered list of APs that represent it, called the \emph{variable map}, and $\mathcal{A}$ is a Spot automaton (specifically, a value of type \texttt{spot.twa\_graph}).
% For information about the representation of Spot automata, see the Spot library \cite{duret.16.atva2}.
We use the convention that calligraphic letters represent actual B\"uchi automata, and Fraktur letters represent automata in the Pecan sense of a pair of a variable map and B\"uchi automaton.

A \emph{variable map} $V$ is a finite set of mappings $x \mapsto [\texttt{ap}_1, \ldots, \texttt{ap}_n]$ such that for all distinct variables $x$ and $y$, $V[x] \cap V[y] = []$.
We denote by $V[x]$ the list of APs that $x$ is represented by, and we denote by $V \cup W$ the union of two variables maps union, which is only defined when the only keys that $V$ and $W$ have in common have identical APs.
$V \sqcup W$ is the disjoint union of these maps.
$V \setminus K$ is the variable map containing every entry $x \mapsto a \in V$ such that $x \not\in K$.

For two variable maps $V$ and $W$, $V \ll W$ denotes their \emph{biased merge}, which is a pair $(U, \theta)$ of a variable map $U$ and a substitution $\theta$ such that $U = V \cup W\theta$.
A substitution is a set of mappings $a \mapsto b$ where $a$ and $b$ are both APs, which can be applied to a variable map or an automaton to rename the APs in them.
For example, if $\theta = \{ a \mapsto d, c \mapsto e \}$, then $\{ x \mapsto [ a, b, c ] \} \theta = \{ x \mapsto [ d, b, e ] \}$.
When it is clear, we also write $V \ll W$ to denote just the resulting variable map, without the associated substitution.

Below, we describe the evaluation of Pecan programs via a big-step relation $E \evaluates \mathcal{A}$.
Automata literals (generally loaded from files), written $\text{Aut}(V, \mathcal{A})$, simply evaluate to be the automata they store: $\text{Aut}(V, \mathcal{A}) \evaluates (V, \mathcal{A})$.

\textbf{Logical Operations}
Fundamental automata operations (i.e., $\land$ and $\lor$, represented by $\oplus$ below) are defined below.
\[
\small
    (V, \mathcal{A}) \oplus (W, \mathcal{B}) = 
    \begin{cases}
        (V \ll W, \mathcal{A} \oplus \mathcal{B}) & \tif |S(\mathcal{A})| < |S(\mathcal{B})| \\
        (W \ll V, \mathcal{A} \oplus \mathcal{B}) & \owise
    \end{cases}
\]
where $S(\mathcal{A})$ denotes the set of states of $\mathcal{A}$.
We also define $\lnot (V, \mathcal{A}) = (V, \lnot \mathcal{A})$.

\textbf{Substitution}
Let $\mathfrak{A} = (V, \mathcal{A})$, where $\mathcal{A} = (Q, \Delta, \delta, q_0, F)$ be a B\"uchi automaton where $\Delta$ is the set of formulas involving $\land$, $\lor$, and $\lnot$ on a finite set $X$ of APs.
We now define the substitution $\mathcal{A}[y/x]$, replacing $x$ by $y$.
Let $A = [\overline{a}]$ be the list of APs representing $x$ (i.e., $A = V[x]$), and let $B = [\overline{b}]$ be the list of APs representing $y$, which we assume is ambiently available.
This can be stored globally, and generated when needed if the variable $y$ has never been used before.

Define $\mathfrak{A}[y/x] = (V', \mathcal{A}')$ where $V' = (V \setminus \{ x \}) \cup \{ y \mapsto B \}$, and $\mathcal{A}' = (Q, \Delta', \delta', q_0, F)$, with the new set of variables $X' = (X \setminus A) \cup B$ and the same underlying alphabet, such that:
\[
    \Delta' = \{ \varphi[\overline{b/a}] : \varphi \in \Delta \}; \tand \delta' = \{ (s, d, \varphi[\overline{b/a}]) : (s, d, \varphi) \in \delta' \}
\]

\textbf{Predicate Calls}
\begin{mathpar}
\small
\inferrule*[right=Call]{
    \overline{e \evaluates (\mathfrak{A}, x)}
    \and
    \nonvar(\overline{e}) = [k_1, \ldots, k_{\ell}]
    \\
    P(\overline{X}) \leadsto Q(\overline{y})
    \and
    (Q(\overline{z : \tau}) := R) \in \mathcal{P}
    \and
    R \evaluates \mathfrak{B}
}{ P(\overline{e}) \evaluates \proj_{x_{k_1}, \ldots, x_{k_{\ell}}}\left( \bigwedge \overline{\mathfrak{A}} \land \mathfrak{B}[\overline{y/z}]\right) }
\end{mathpar}
where $\nonvar(\overline{e})$ denotes the nonvariable positions in $\overline{e}$.

% The rule for evaluating predicate calls is rather complicated, due to the possibility of dynamic calls.
% Fundamentally, the rule does the following:

% \begin{enumerate}
%     \item Evaluates each argument $e_i$ as $(\mathfrak{A_i} x_i)$.
%     \item Resolves the call $P$ as some predicate $Q$
%     \item Evaluates the body of $Q$, $R$ and then substitutes in the variables $y_i$ (in the resolve call, some of which will be the result variables $x_i$) as appropriate.
%     \item Projects out the intermediate result variables $x_{k_p}$, where the $k_p$ are the nonvariable positions in $e_1, \ldots, e_n$.
%         This means that it only makes sense to use expressions whose output values will be well-behaved: that is, the resulting automaton should be the same regardless of which output was ``used.''
%         Pecan could, but does not check this condition for performance reasons.
% \end{enumerate}

\textbf{Existential Quantification}
\begin{mathpar}
\small
\inferrule*[right=Exist]{
    (\tau(x) \land P) \evaluates (V, \mathcal{A})
}{ (\exists x \in \tau. P) \evaluates \proj_{V[x]}(V, \mathcal{A}) }
\end{mathpar}

Here $\proj_{V[x]}(\mathfrak{A})$ denotes the automaton $\mathfrak{A}$ after projecting out every AP representing $x$ in the variable map $V[x]$; this operation is implemented in Spot.

\textbf{Expressions}
An expression $E$ evaluates to a pair $(\mathfrak{A}, x)$ of an automaton $\mathfrak{A}$ and a variable $x$.
% Denote by $\Free(E)$ the list of free variables occurring in $E$.
% For example, $\Free(\forall a. a + b + c) = [b, c]$.
% Denote by $E[v/x]$ the expression $E$ with $v$ substituted for $x$ where $x \in \Free(E)$.
Many rules, like \textsc{Add}, need to evaluate subexpressions.
While evaluating a subexpression $e$, it may be that we generate fresh variables to store the result, which must be projected out.
The only case in which this does not occur is when the subexpression is itself a variable.
We write $\proj_{\overline{x}}(\mathfrak{A})$ to denote projecting out the intermediate variables resulting from computing expressions that are \textbf{not} variables.
For example, if $a \evaluates (\mathfrak{A}, x)$ and $b \evaluates (\mathfrak{B}, y)$ then $\proj_{a,b}(\mathfrak{A})$ denotes $\proj_V(\mathfrak{A})$ where $V = \{ v : (e, v) \in \{ (a,x), (b,y) \}, e \neq v \}$.

\begin{mathpar}
\small
\inferrule*[right=Var]{ }{ x \evaluates (\top, x) }

\inferrule*[right=Zero]{ x ~ \text{fresh} }{ 0 \evaluates (\texttt{zero}(x), x) }

\inferrule*[right=One]{ x ~ \text{fresh} }{ 1 \evaluates (\texttt{one}(x), x) }

\inferrule*[right=Add]{ 
    a \evaluates (\mathfrak{A}, x)
    \\
    b \evaluates (\mathfrak{B}, y)
    \\
    (x + y = z) \evaluates \mathfrak{C}
    \\
    z ~ \text{fresh}
} { a + b \evaluates (\proj_{a,b}(\mathfrak{A} \land \mathfrak{B} \land \mathfrak{C}), z) }

\inferrule*[right=Sub]{ 
    a \evaluates (\mathfrak{A}, x)
    \\
    b \evaluates (\mathfrak{B}, y)
    \\
    (z + y = x) \evaluates \mathfrak{C}
    \\
    z ~ \text{fresh}
} { a - b \evaluates (\proj_{a,b}(\mathfrak{A} \land \mathfrak{B} \land \mathfrak{C}), z) }

\inferrule*[right=Int]{
    \overbrace{1 + 1 + \cdots + 1}^{n~\text{times}} \evaluates (\mathfrak{A}, x)
}{ n \evaluates (\mathfrak{A}, x) }

\inferrule*[right=Func]{
    f(\overline{e}, x) \evaluates \mathfrak{A}
    \\ x ~ \text{fresh}
}{ f(\overline{e}) \evaluates (\proj_{\overline{e}}(\mathfrak{A}), x) }

\inferrule*[right=Rel]{ 
    a \evaluates (\mathfrak{A}, x)
    \and
    b \evaluates (\mathfrak{B}, y)
    \\
    (x \Join y) \evaluates \mathfrak{C}
    \\
    \Join \in \{ \equiv, < \}
} { a \Join b \evaluates \proj_{a,b}(\mathfrak{A} \land \mathfrak{B} \land \mathfrak{C}) }
% \inferrule*[right=Less]{ 
%     a \evaluates (\mathfrak{A}, x)
%     \and
%     b \evaluates (\mathfrak{B}, y)
%     \\
%     (x < y) \evaluates \mathfrak{C}
% } { a < b \evaluates \proj_{a,b}(\mathfrak{A} \land \mathfrak{B} \land \mathfrak{C}) }
\end{mathpar}

% Define $\proj_x(\mathfrak{A})$ to be $\proj_{\mathcal{V}[x]}(\mathcal{B})$, defined in the next section.

% \reed{Useless crossref atm, this is the next section...}
% For the actual implementation of the projection operation, see Section \ref{sec:impl-projection}.

% \subsubsection{Projection}\label{sec:impl-projection}

% Let $\mathfrak{A} = (Q, \Delta, \delta, q_0, F)$ be a B\"uchi automaton where $\Delta$ is the set of formulas involving $\land$, $\lor$, and $\lnot$ on a finite set $X$ of APs.
% We compute $\proj_Y(\mathcal{A}) = (Q, \Delta', \delta', q_0, F)$, where $Y$ is a finite list of APs and $\Delta'$ is the set of formulas on the APs $X \setminus Y$.
% The new transition relation $\delta'$ is defined as
% \[
%     \delta' = \left\{ \left( s, d, \bigvee_{x \in Y} (\varphi[\top / x] \lor \varphi[\bot / x])  \right)  : (s, d, \varphi) \in \Delta' \right\}
% \]
% where $\varphi[a/x]$ is denotes $\varphi$ with $a$ substituted for $x$.
% That is, for every transition from a state $s$ to a state $d$ on symbol $\varphi$, we have a transition from $s$ to $d$ whenever $\varphi$ holds regardless of whether any variable $x \in Y$ is true or false.
