\section{Automata Formats}\label{sec:aut-format}

Below are details on all the automata formats that Pecan supports.
Note that all formats support both deterministic and nondeterministic automata.

\subsection{HOA}

Pecan supports the HOA format \cite{hoa-format}, in which case when loading the parameter \textbf{must} be given the same names as the APs in the automaton, in the same order.
This format only supports binary alphabets; to avoid this limitation you must either binary-encode your automaton yourself or use another format.
Both the Walnut and Pecan formats support arbitrary integer base alphabets.

To use this format, write:
\begin{pecan}
#load("filename.txt", "hoa", pred_name(arg1, arg2, ...))
\end{pecan}

An example is shown in Figure~\ref{fig:example-hoa-aut}.
\begin{figure}
    \centering
    \begin{lstlisting}[basicstyle=\normalsize\ttfamily]
HOA: v1
States: 2
Start: 1
AP: 1 "a" 
acc-name: Buchi
Acceptance: 1 Inf(0)
properties: trans-labels explicit-labels state-acc complete
properties: deterministic stutter-invariant terminal
--BODY--
State: 0 {0} 
[t] 0
State: 1
[!0] 0
[0] 1
--END--
    \end{lstlisting}
    \caption{An automaton in the HOA format, accepting words that contain at least one \texttt{0}.}
    \label{fig:example-hoa-aut}
\end{figure}

\subsubsection{HOA (Modified)}

Pecan also supports the HOA format as described above, except that the first line may be of the form:

\begin{lstlisting}[basicstyle=\normalsize\ttfamily]
VAR_MAP: PYTHON_DICT
// Example:
VAR_MAP: {'a': ['__ap76', '__ap77'], 'n': ['__ap78', '__ap79']}
\end{lstlisting}

where \lstinline{PYTHON_DICT} is a valid Python dictionary mapping variable names to the ordered collection of APs (atomic propositions) that represents them.
The rest of the file follows the HOA format.

\subsection{Walnut}

Pecan supports the same format that Walnut \cite{walnut} uses.
Pecan uses the Unix standard of using files encoded using UTF-8 with no BOM (as this would be redundant for UTF-8), unlike Walnut which (at the time of writing) requires UTF-16BE and allows, but does not require, a BOM.
This format supports automata in any integer base (i.e., those with an alphabet like $\{0,1,2,\ldots\}$).

To use this format, write:
\begin{pecan}
#load("filename.txt", "walnut", pred_name(arg1, arg2, ...))
\end{pecan}

An example is shown below in Figure~\ref{fig:example-walnut-aut}.

\begin{figure}
    \centering
    \begin{lstlisting}[basicstyle=\normalsize\ttfamily]
    {0,1,2}
    0 0
    0 -> 0
    0 -> 1
    1 -> 0
    2 -> 0
    1 1
    0 -> 1
    2 -> 1
    \end{lstlisting}
    \caption{An automaton in the Walnut format that accepts words that eventually end in $(0^* 2)^{\omega}$.}
    \label{fig:example-walnut-aut}
\end{figure}

See the Walnut manual for details \url{https://github.com/hamousavi/Walnut/blob/master/Automatic\%20Theorem\%20Proving\%20in\%20Walnut.pdf}.

\subsection{Pecan}

Pecan also supports it's own automata format, which is based on the Walnut format but allows for more descriptive state names and comments.
This format supports automata in any integer base (i.e., those with an alphabet like $\{0,1,2,\ldots\}$).

To use this format, write:
\begin{pecan}
#load("filename.txt", "pecan", pred_name(arg1, arg2, ...))
\end{pecan}

Somewhat informally, the format is:

\begin{lstlisting}[basicstyle=\normalsize\ttfamily]
AUTOMATON ::= ALPHABET_LINE STATE*

ALPHABET_LINE ::= INPUT+
INPUT ::= "{" 0, 1, ... "}"

STATE ::= LABEL ":" (0 | 1) TRANSITION*
SYMBOL ::= INTEGER*
LABEL ::= [^:]+
TRANSITION ::= SYMBOL "->" LABEL

COMMENT ::= "//" .*
\end{lstlisting}

An example is shown in Figure~\ref{fig:example-pecan-aut}.

\begin{figure}
    \centering
    \begin{lstlisting}[basicstyle=\normalsize\ttfamily]
// Input is a natural number with each digit encoded in binary and separated by 2s.
{0,1,2}
// All inputs must start with 2
start: 0
2 -> zero,even
zero,even: 0
0 -> zero,even
2 -> zero,odd
zero,odd: 0
0 -> zero,odd
1 -> done,odd
2 -> zero,even
done,odd: 1
0 -> done,odd
1 -> done,odd
2 -> done,odd
    \end{lstlisting}
    \caption{An automaton in the Pecan format, accepting $n$ such that the $n$-th digit of the characteristic Sturmian word (of any slope) is $1$.}
    \label{fig:example-pecan-aut}
\end{figure}
