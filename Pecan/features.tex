\section{Features}\label{sec:features}

\subsection{Functions}

Some predicates represent relations which are functions.
For readability, Pecan allows the use of \term{placeholders}, which allow specifying the output of a function-like predicate.
For example, if we have a predicate \pecaninline{bin\_add(x, y, z)} accepting triples such that $x + y = z$ for binary numbers $x$, $y$, and $z$, then \pecaninline{bin\_add(x, y, \_)} is equivalent to writing $x + y$.
So we could have written \pecaninline{bin\_add(x, y, \_) = z} instead of \pecaninline{bin\_add(x, y, z)}.
Such a formula will be translated into:
\begin{pecan}
$\color{red} \exists$v. bin_add(x, y, v) & v = z
\end{pecan}

Because the last argument to a function-like relation is typically the output argument, we can simply write \pecaninline{f(x)} to mean \pecaninline{f(x, \_)}.

\subsection{Annotations}

Pecan supports \term{annotations}, which may \term{wrapped} around any formula.
They do \textbf{not} change the logical value of the automaton, but are instead used to tell Pecan how to evaluate a formula (e.g., when to simplify).

The follow annotations are available:

\begin{pecan}
@postprocess[FORMULA]
\end{pecan}

Performs a series of simplifications on the automaton representing \pecaninline{FORMULA}.

\begin{pecan}
@simplify[FORMULA]
\end{pecan}

Turns on simplification (the default) for the formula and all its subformulas.

\begin{pecan}
@no_simplify[FORMULA]
\end{pecan}

Turns off simplification for the formula and all its subformulas.

\begin{pecan}
@simplify_states[FORMULA]
\end{pecan}

Performs basic simplifications on states, such as merging equivalent states and purging unreachable and dead states.

\begin{pecan}
@simplify_edges[FORMULA]
\end{pecan}

Performs basic simplifications on states with the goal of reducing the number of edges.
See the Spot documentation for \pecaninline{merge_edges} for more information.

\subsection{Automatic Words}

Currently only binary automatic words are supported, which are just syntactic sugar for predicates.
Any predicate $P$ can be interpreted as a word by writing $P[i]$, which is treated as $1$ if $P(i)$ is true, and $0$ if $P(i)$ is false.
Then we have the following translations into the IR:

\begin{itemize}
    \item $P[i] = 1$ becomes $P(i)$
    \item $P[i] = 0$ becomes $\lnot P(i)$
    \item $P[i] = Q[j]$ becomes $P(i) \iff P(j)$
    \item $P[i] \neq Q[j]$ becomes $\lnot (P(i) \iff P(j))$
\end{itemize}

\subsection{Directives}

\begin{pecan}
#save_aut(FILENAME, PREDICATE)
#save_aut("bin_even.aut", bin_even)
\end{pecan}

Saves the predicate as an automaton in the HOA~\cite{hoa-format} format to the location specified.

\begin{pecan}
#save_aut_img(FILENAME, PREDICATE_NAME)
#save_aut_img("bin_even.svg", bin_even)
\end{pecan}

Saves the predicate as an SVG file at the location specified.

\begin{pecan}
#context(KEY, VALUE)
#context("adder", "bin_add")
\end{pecan}

Sets the key to the value in the current context.
The context is a global key-value store that is used to manage some parts of Pecan such as defaults (e.g., the default adder, equals, etc.).

\begin{pecan}
#end_context(KEY)
#end_context("adder")
\end{pecan}

Deletes the specified key from the context.

\begin{pecan}
#load(FILENAME, FORMAT, PREDICATE)
#load("bin_add.aut", "hoa", bin_add(a,b,c))
#load("real/real_equal.txt", "pecan", real_equal(a, b))
\end{pecan}

Loads an automaton from a file with the specified name and arguments.
Note that the number of arguments is \emph{NOT} checked for you.
In general, you should be careful when loading automata from files because there are not many safeguards (it is assumed you know what you are doing).
There are three currently supported formats: ``hoa''~\cite{hoa-format}, as described earlier; ``walnut'', the format that the automated theorem prover Walnut~\cite{walnut} uses; and ``pecan'', a custom format that Pecan uses, described below in~\ref{sec:aut-format}.

\begin{pecan}
#assert_prop(PROP_VAL, PREDICATE_NAME)
#assert_prop(true, int_add_comm)
\end{pecan}

The main interface to the theorem proving capabilities of Pecan.
The possible values for \pecaninline{PROP_VAL} are \pecaninline{true}, \pecaninline{false}, and \pecaninline{sometimes}.
While theorems are typically represented by predicates that take no arguments, this directive still works if the predicate does take arguments.
In that case, asserting a predicate is \pecaninline{true} is the same as asserting that it is true for all possible inputs; asserting that it is \pecaninline{sometimes} true checks if there is any input that it accepts.

\begin{pecan}
#import(FILENAME)
#import("integers.pn")
\end{pecan}

Loads all definitions from the specified file into the current scope.
Does \emph{NOT} load restrictions into the current scope, but they function normally while evaluating the imported file.
To control the search path, you can set the \lstinline{PECAN_PATH} environment variable.
By default, the current working directory (at the time of loading the program), the directory the current source file is in, and the \lstinline{library/} directory of the Pecan directory.

\begin{pecan}
#forget(VARIABLE)
#forget(x)
\end{pecan}

Removes all restrictions on the specified variable.

\begin{pecan}
Structure TYPE_PREDICATE defining {
    FUNCTION_PREDICATES
}

Structure nat defining {
    "adder": bin_add(any, any, any),
    "less": bin_less(any, any)
}
\end{pecan}

Defines a new type.
\pecaninline{TYPE_PREDICATE} should be the part you write after the \pecaninline{is} in a restriction. For example, in \pecaninline{Restrict x is nat.}, the relevant part is \pecaninline{nat}; in \pecaninline{Restrict i is ostrowski(a).}, the relevant part is \pecaninline{ostrowski(a)}.
The function predicates become available to be called using the names in quotes---this feature allows for ad-hoc polymorphism.
It is also used to resolve arithmetic operators, such as \pecaninline{+} (which calls the relevant \pecaninline{adder}) and \pecaninline{<} (which calls the relevant \pecaninline{less}).
For example, in the following code, \pecaninline{f} and \pecaninline{g} are equivalent.
\begin{pecan}
Restrict a, b are nat.

f(a, b) := a < b
g(a, b) := bin_less(a, b)
\end{pecan}

The full list of special names is as follows:

\begin{itemize}
    \item[\texttt{adder}] Used to resolve \pecaninline{+}
    
    \item[\texttt{less}] Used to resolve \pecaninline{<}
    
    \item[\texttt{zero}] Used to resolve \pecaninline{0}
    
    \item[\texttt{one}] Used to resolve \pecaninline{1}, and all integer constants other than \pecaninline{0} (e.g., \pecaninline{2} is calculated as \pecaninline{1 + 1})
    
    \item[\texttt{equal}] Used to resolve \pecaninline{=}
    
\end{itemize}

\begin{pecan}
#shuffle(PREDICATE, PREDICATE, PREDICATE)
#shuffle(int_less(a, b), any2(a, b), integral_less(a, b))
\end{pecan}

Shuffles the two predicates into one, by alternating between the two; shuffling is defined as below.

\begin{definition}
    The \textbf{shuffle} of two automata $\mathcal{A}$ and $\mathcal{B}$ is an automaton $\mathcal{C}$ such that if $a = a_1 a_2 \cdots$ and $b = b_1 b_2 \cdots$ are $\omega$-words accepted by $\mathcal{A}$ and $\mathcal{B}$ respectively, then $c = a_1 b_1 a_2 b_2 \cdots$ is accepted by $\mathcal{C}$.
\end{definition}

In the example above, \pecaninline{int_less(a, b)} accepts $a$ and $b$ such that $a < b$ where $a, b \in \Z$ and \pecaninline{any2(a, b)} accepts every pair of strings $a$ and $b$.
If we think of the shuffled string as a pair (e.g., if our string is $c = a_1 b_1 a_2 b_2 \cdots$, we could think of it as $c = (a_1 a_2 \cdots, b_1 b_2 \cdots)$) whose first component is the integral part of a real number and the second component is the fractional part, then their shuffle is an automaton that accepts real numbers $a$ and $b$ such that the integral part of $a$ is less than that of $b$, and ignores the fractional part.

\begin{pecan}
#shuffle_or(FILENAME, FORMAT, PREDICATE)
#shuffle_or(one_int(x), zeros(x), real_one(x))
\end{pecan}

This shuffle operation is the same as above, except that it only requires that one of the strings in the shuffle is accepted by one of the automata.
More specifically:

\begin{definition}
    The \textbf{(disjunctive) shuffle} of two automata $\mathcal{A}$ and $\mathcal{B}$ is an automaton $\mathcal{C}$ such that if $a = a_1 a_2 \cdots$ and $b = b_1 b_2 \cdots$ are $\omega$-words such that \emph{either} $\mathcal{A}$ accepts $a$ \emph{or} $\mathcal{B}$ accepts $b$ (or both are accepted), then $c = a_1 b_1 a_2 b_2 \cdots$ is accepted by $\mathcal{C}$.
\end{definition}

\begin{pecan}
Restrict VARIABLES (is|are|$\color{red} \in$) TYPE.
\end{pecan}

In all following code in this file, the variables specified are now consider to be of the specified type, unless deleted by a \pecaninline{#forget}.
