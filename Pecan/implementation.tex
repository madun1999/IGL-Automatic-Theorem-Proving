\section{Implementation}\label{sec:implementation}

This section describes at a high-level the implementation of Pecan.
First, we describe the syntax in Section~\ref{sec:ir-syntax}.
Then, we give a formal definition of the Pecan language, starting with the typing rules and associated definitions in Section~\ref{sec:typing}, and then the rules for evaluation in Section~\ref{sec:evaluation}.
Next, we describe each step of the AST to IR conversion in Section~\ref{sec:ast-to-ir}, and finally the various optimizations performed by Pecan prior to execution in Section~\ref{sec:optimizations}.

The process of executing a program is roughly as follows:

\begin{enumerate}
    \item Parse the program into an AST (Abstract Syntax Tree).
    \item Transform the program's AST into a simplified IR  representation, rewriting constructs such as $\iff$ in terms of simpler ones, like $\land$, $\lor$, and $\lnot$.
    \item Load the standard library (if desired); loading the standard library requires going through all of the steps again, starting from step 1, but skipping this step.
    \item Then, for each definition or directive, perform the following steps:
        \begin{enumerate}
            \item Run the untyped optimizer, if enabled.
            \item Perform type inference.
            \item Perform a final IR lowering.
            \item Run the typed optimizer, if enabled.
            \item Interpret the final IR of the current top-level construct.
        \end{enumerate}
\end{enumerate}

\subsection{IR Syntax}\label{sec:ir-syntax}

Below is the IR Syntax for the Pecan language; the full syntax that Pecan supports is much larger, but the rest of it is ``simply'' syntactic sugar which is expanded as explained in Section~\ref{sec:ast-to-ir}.

\begin{tabular}{c c l}
     Prog & \bnfdef & Definition$^*$ \\
     Definition & \bnfdef & $P(x_1 : \tau_1, \ldots, x_n : \tau_n)$ := Pred \\
     & \bnfalt & Restrict $x_1,\ldots,x_n$ are $P(y_1,\ldots,y_m)$ \\
     Pred & \bnfdef & true \bnfalt false \bnfalt Pred $\lor$ Pred \bnfalt $\lnot$ Pred \bnfalt Pred $\land$ Pred \\
     & \bnfalt & $\exists x. $ Pred \bnfalt E $<$ E \bnfalt E $=$ E \bnfalt @A[Pred] \bnfalt $P$(E, E, \ldots, E) \bnfalt Aut($V, \mathcal{A}$) \\
     E & \bnfdef & E + E \bnfalt E - E \bnfalt $x$ \bnfalt $i$ \bnfalt $f$(E, E, \ldots, E) 
\end{tabular}

\subsection{Type Checking}\label{sec:typing}

A \term{type} in Pecan is represented by a B\"uchi automaton.
We say that $x : \tau$ when $x \in L(\tau)$, sometimes simply written, as an analogy to logical predicates, as $\tau(x)$.
Additionally, types may be \term{partially applied}, i.e., $\tau = P(x_1, \ldots, x_n)$, where $P$ is some B\"uchi automaton.
Then $y : \tau$ when $(x_1, \ldots, x_n, y) \in L(P)$; and $\tau(y)$ holds when $y : \tau$.
In the concrete syntax of Pecan, we write \pecaninline{y is tau} or \pecaninline{y} $\in$ \pecaninline{tau}; for one or more variables, we can write \pecaninline{x, y, z are tau} to mean $x : \tau$, $y : \tau$, $z : \tau$.
Finally, there is a special type called $\inferred$, which should be thought of as a type that will be discovered later by how the value is used; or, as its name suggests, the type will eventually be \textbf{inferred} from context.
$L(\inferred)$ is undefined---$\inferred$ is NOT represented by an automata.

The judgement $\Gamma \proves x : \tau$ means that we can prove $\tau(x)$ is true in the environment $\Gamma$, which is a set of assumptions $x : \tau$.
The set $\{ x : \exists \tau. x : \tau \in \Gamma\}$ is the \term{domain} of $\Gamma$, written $\dom{\Gamma}$.
If for every $x \in \dom{\Gamma}$, there is a unique type $\tau$ such that $x : \tau \in \Gamma$, we say that $\Gamma$ is well-formed.
We assume that all contexts are well-formed unless otherwise specified.
The judgement $\Gamma \proves \prop{P}$ means that $P$ is a well-formed proposition in the environment $\Gamma$.
A predicate $P(\overline{x : \tau}) := Q$ is well-formed when $\overline{x : \tau} \proves \prop{Q}$.

We typecheck a sequence of Pecan predicates in order, starting with an empty environemnt $\Gamma = \emptyset$.
Below, we assume that the set of all well-formed predicates, which have already been checked, is ambiently available as $\mathcal{P}$.
Similarly, whenever we encounter a restriction, Restrict $x_1, \ldots, x_n$ are $P(y_1, \ldots, y_m)$, we update the current environment $\Gamma$ to be $\Gamma ~\cup~ \{ x_1 : P(y_1, \ldots, y_m), \ldots, x_n : P(y_1, \ldots, y_m, x_n) \} $; it is only allowed to use $P \in \mathcal{P}$, that is, predicates which have already been confirmed are well-formed.

\paragraph{Structures}

In order to provide a mechanism for ad-hoc polymorphism, Pecan allows the definition and use of \term{structures}.
This feature also facilitates the use of nicer syntax for arithmetic expressions (e.g., $x + (y + z) = w$ instead of $\exists t. \texttt{adder}(x, y, t) \land \texttt{adder}(t, z, w)$) without tying ourselves to a single numeration system.
For example, $\texttt{adder}$ will be resolved to some concrete predicate predicate based on the type of $x$, $y$, and $z$.
The exact rules and definitions related to this feature are given below.
We assume that structure definitions are ambiently available throughout the program.

\begin{definition}
    A \term{structure} is a pair $(t(x_1, \ldots, x_n), D)$ where each $x_i$ is an identifier, such that for some $\tau_1, \ldots, \tau_n$, $x_1 : \tau_1, \ldots, x_n : \tau_n \proves \prop{t(x_1, \ldots, x_n)}$ and $D$ is a map of identifiers to \term{call templates}; that is, it is of the form $f(y_1, \ldots, y_m)$, where each $y_i$ may be either: 1) $x_j$ for some $j$ or 2) $*$, which is pronounced ``any.''
    
    The \term{name} of the structure is $t$.
\end{definition}

We write the sequence of indexes of the arguments that are $*$, called \term{parameters}, as $\params(f(y_1, \ldots, y_m))$.
A call template is called $n$-ary if $|\params(f(y_1, \ldots, y_m))| = n$.
The sequence of the indexes of the other arguments, which are not $*$, called \term{implicits}, is written $\implicits(f(y_1, \ldots, y_m))$.
For example, $\params(f(a, *, *, b, *, *)) = [2, 3, 5, 6]$ and $\implicits(f(a, *, *, b, *, *)) = [1, 4]$.
We write $\params(f(y_1, \ldots, y_m))[i]$ (resp. $\implicits(f(y_1, \ldots, y_m))[i]$) to denote the $i$-th parameter (resp. implicit).

We assume that typechecking has been done before evaluating, because we may need structure information at run-time to resolve \term{dynamic calls}, that is, predicate calls whose name matches some definition inside a type.
We denote the type that an expression $e$ got when typechecking by $\typ{e}$.

We write $t[P] = Q(y_1, \ldots, y_m)$ to look up a definition in the associated map $D$, and we say that $t$ \term{has a definition} for $P$ in this case.
If $t$ does not have a definition for $P$, then we write $t[P] = \bot$.

\begin{definition}
    A structure is called \term{numeric} if it has a ternary definition for $\texttt{adder}$ and a binary definition $\texttt{less}$.
    We write $x + y = z$ when $\texttt{adder}(x, y, z)$ holds and $x < y$ when $\texttt{less}(x, y)$ holds.
    
    A numeric structure may also optionally contain the following definitions; otherwise a default predicate applies.
    
    \begin{itemize}
        \item[] A binary definition $\texttt{equal}$, written $x \equiv y$. 
            If not provided, the default is simply standard equality, $x = y$.
        
        \item[] A unary definition $\texttt{zero}$.
            If not provided, the default is $0^{\omega}$.
            
        \item[] A unary definition $\texttt{one}$.
            If not provided, the default is $x$ such that $0 \leq x \land \forall y. y = 0 \lor x \leq y$.
    \end{itemize}
\end{definition}

\begin{definition}
    The \emph{predecessor} of two types $\tau$ and $\sigma$, written $\tau \join \sigma$ is given by following partial function:
    \[
        \tau \join \sigma = 
        \begin{cases}
            \sigma & \tif \tau = \inferred~\text{and $\sigma$ is numeric} \\
            \sigma & \tif \forall \Free(\tau). \forall x. \tau(x) \implies \sigma(x) \\
            \tau & \tif \forall \Free(\tau). \forall x. \sigma(x) \implies \tau(x) \\
            \tau & \tif \sigma = \inferred~\text{and $\tau$ is numeric} \\
        \end{cases}
    \]
\end{definition}

\begin{definition}
    We can \term{resolve} a call $P(e_1, \ldots, e_n)$ as $Q(a_1 : \tau_1, \ldots, a_m : \tau_m)$, written $P(e_1, \ldots, e_n) \leadsto Q(a_1 : \tau_1, \ldots, a_m : \tau_m)$, if $Q(a_1 : \tau_1, \ldots, a_m : \tau_m) \in \mathcal{P}$ and for some structure $t(x_1, \ldots, x_\ell)$, for each $1 \leq i \leq n$, either:
    \begin{enumerate}
        \item $\typ{e_i} = t(x_1, \ldots, x_\ell)$, and $t[P] = Q(b_1, \ldots, b_m)$ such that
        for each $1 \leq j \leq m$,
        \[
            a_j =
            \begin{cases}
                x_k & \tif \implicits(Q(b_1, \ldots, b_m))[k] = j \\
                e_k & \tif \params(Q(b_1, \ldots, b_m))[k] = j
            \end{cases}
        \]
        
        \item $\typ{e_i} = s(y_1, \ldots, y_p)$, where $s \neq t$ and $s[P] = \bot$.
    \end{enumerate}
    
    or, if none of the arguments have a definition for $P$, then $P(e_1, \ldots, e_n) \leadsto P(a_1 : \tau_1, \ldots, a_m : \tau_m) \in \mathcal{P}$.
\end{definition}

\framebox{$\Gamma \proves e : \tau$} \textbf{Typing}
\begin{mathpar}
\inferrule*[right=Int]{
    i \in \Z
} { \Gamma \proves i : \inferred }

\inferrule*[right=Var]{
    x : \tau \in \Gamma
} { \Gamma \proves x : \tau }

% \reed{Not sure that I really want to support full type inference like this}
% \inferrule*[right=Var-Env]{
%     x \not\in \dom{\Gamma}
% }{ \Gamma \proves x : \inferred }

\inferrule*[right=Op]{ 
    \Gamma \proves a : \tau
    \and 
    \Gamma \proves b : \sigma
}{ \Gamma \proves a \oplus b : \tau \join \sigma }
\twhere \oplus \in \{ +, -, / \}

\inferrule*[right=Func]{
    \forall i \in \{ 1, \ldots, n \}. \Gamma \proves e_i : \tau_i
    \and
    f(x_1 : \tau_1, \ldots, x_n : \tau_n, r : \tau) \in \mathcal{P}
}{ \Gamma \proves f(e_1, \ldots, e_n) : \tau }
    
\end{mathpar}

\framebox{$\Gamma \proves \prop{P}$} \textbf{Well-formed Propositions}
\begin{mathpar}
\inferrule*[right=Rel]{ 
    \Gamma \proves a : \tau
    \and 
    \Gamma \proves b : \sigma
    \and
    \tau \join \sigma \neq \inferred
}{ \Gamma \proves \prop{a \Join b} }
\twhere \Join \in \{ \equiv, < \}

\inferrule*[right=BinPred]{
    \Gamma \proves \prop{P}
    \and
    \Gamma \proves \prop{Q}
}{ \Gamma \proves \prop{P \oplus Q} }
\twhere \oplus \in \{ \lor, \land \}

\inferrule*[right=Comp]{
    \Gamma \proves \prop{P}
}{ \Gamma \proves \prop{\lnot P} }

\inferrule*[right=Exists]{
    \Gamma, x : \tau \proves \prop{P}
}{ \Gamma \proves \exists x : \tau. \prop{P} }

\inferrule*[right=Call]{
    \forall i \in \{1, \ldots, n\}. \Gamma \proves e_i : \tau_i
    \and
    f(x_1 : \tau_1, \ldots, x_n : \tau_n) \in \mathcal{P}
}{ \Gamma \proves \prop{f(e_1, \ldots, e_n)} }

\inferrule*[right=Annotation]{
    \Gamma \proves \prop{P} 
}{ \Gamma \proves \prop{@A[P]} }

\inferrule*[right=Automaton]{
}{ \Gamma \proves \prop{\text{Aut}(V, \mathcal{A})} }
\end{mathpar}

\reed{TODO: Some proofs maybe (e.g., if $\Gamma \proves x : \tau$, then if $\Gamma$, then $x \in L(\tau)$, or if $\Gamma \proves \prop{P}$ then $P \evaluates \mathfrak{A}$ for some $\mathfrak{A}$}

\subsection{Evaluation}\label{sec:evaluation}

The Pecan interpreter is a simple tree-walking interpreter which typechecks, then processes, each top-level construct (see Section~\ref{sec:ir-syntax}) in sequential order.

Operations with non-trivial implementations are described in detail below.
Most basic automata operations (e.g., conjunction, disjunction, complementation, emptiness checking, simplification) are implemented using the Spot library \cite{duret.16.atva2}, so details of these algorithms are not presented here.

\subsubsection{Automata Representation}

Automata are represented by a pair of $(V, \mathcal{A})$, where $V$ is a map taking variable names to an ordered list of APs that represent it, called the \emph{variable map}, and $\mathcal{A}$ is a Spot automaton (specifically, a value of type \texttt{spot.twa\_graph}).
For more information about the underlying representation of Spot automata, see the Spot library \cite{duret.16.atva2}.
We use the convention that calligraphic letters represent actual B\"uchi automata, and Fraktur letters represent automata in the Pecan sense of a pair of a variable map and B\"uchi automaton.

\paragraph{Variable Maps}
A \emph{variable map} $V$ is a finite set of mappings $x \mapsto [\texttt{ap}_1, \ldots, \texttt{ap}_n]$ such that for all distinct variables $x$ and $y$, $V[x] \cap V[y] = []$.

We denote by $V[x]$ the list of APs that $x$ is represented by, and we denote by $V \cup W$ the union of two variables maps union, which is only defined when the only keys that $V$ and $W$ have in common have identical APs.
$V \sqcup W$ is the disjoint union of these maps.
$V \setminus K$ is the variable map containing every entry $x \mapsto a \in V$ such that $x \not\in K$.

For two variable maps $V$ and $W$, $V \ll W$ denotes their \emph{biased merge}, which is a pair $(U, \theta)$ of a variable map $U$ and a substitution $\theta$ such that $U = V \cup W\theta$.
A substitution is a set of mappings $a \mapsto b$ where $a$ and $b$ are both APs, which can be applied to a variable map or an automaton to rename the APs in them.
For example, if $\theta = \{ a \mapsto d, c \mapsto e \}$, then $\{ x \mapsto [ a, b, c ] \} \theta = \{ x \mapsto [ d, b, e ] \}$.
When it is clear, we also write $V \ll W$ to denote just the resulting variable map, without the associated substitution.

\subsubsection{Logical Operations}

Fundamental automata operations (i.e., $\land$ and $\lor$, represented by $\oplus$ below) are defined below.
\[
    (V, \mathcal{A}) \oplus (W, \mathcal{B}) = 
    \begin{cases}
        (V \ll W, \mathcal{A} \oplus \mathcal{B}) & \tif |S(\mathcal{A})| < |S(\mathcal{B})| \\
        (W \ll V, \mathcal{A} \oplus \mathcal{B}) & \owise
    \end{cases}
\]

where $S(\mathcal{A})$ denotes the set of states of $\mathcal{A}$.
For complements, we define $\lnot (V, \mathcal{A}) = (V, \lnot \mathcal{A})$.

Finally, automata literals, written Aut($V, \mathcal{A}$) simply evaluate to be the automata they store:
\begin{mathpar}

\inferrule*[right=Automaton]{
}{ \text{Aut}(V, \mathcal{A}) \evaluates (V, \mathcal{A}) }

\end{mathpar}

\subsubsection{Predicate Calls}

\begin{mathpar}
\inferrule*[right=Call]{
    \forall i. e_i \evaluates (\mathfrak{A}_i, x_i)
    \and
    \nonvar(e_1, \ldots, e_n) = [k_1, \ldots, k_{\ell}]
    \\
    P(x_1, \ldots, x_n) \leadsto Q(y_1, \ldots y_m)
    \and
    Q(z_1, \ldots, z_m) := R
    \and
    R \evaluates \mathfrak{B}
}{ P(e_1, \ldots, e_n) \evaluates \exists x_{k_1}, \ldots, x_{k_{\ell}}. \mathfrak{A}_1 \land \cdots \land \mathfrak{A}_n \land \mathfrak{B}[y_1 / z_1, \ldots, y_m / z_m] }
\end{mathpar}

where $\mathfrak{B}[y_1 / z_1, \ldots, y_m / z_m]$ denotes substituting each $y_j$ for $z_j$ in $\mathfrak{B}$, defined in Section~\ref{sec:impl-substitution}, and $\nonvar(e_1, \ldots, e_n)$ denotes the nonvariable positions in $e_1, \ldots, e_n$.

The rule for evaluating predicate calls is rather complicated, due to the possibility of dynamic calls.
Fundamentally, the rule does the following:

\begin{enumerate}
    \item Evaluates each argument $e_i$ as $(\mathfrak{A_i} x_i)$.
    \item Resolves the call $P$ as some predicate $Q$
    \item Evaluates the body of $Q$, $R$ and then substitutes in the variables $y_i$ (in the resolve call, some of which will be the result variables $x_i$) as appropriate.
    \item Projects out the intermediate result variables $x_{k_p}$, where the $k_p$ are the nonvariable positions in $e_1, \ldots, e_n$.
        This means that it only makes sense to use expressions whose output values will be well-behaved: that is, the resulting automaton should be the same regardless of which output was ``used.''
        Pecan will not automatically check this condition for performance reasons, but it is possible to do automatically.
\end{enumerate}

\subsubsection{Expressions}

Denote by $\Free(E)$ the list of free variables occurring in $E$.
For example, $\Free(a + b + c) = [a, b, c]$.
Denote by $E[v/x]$ the expression $E$ with $v$ substituted for $x$ where $x \in \Free(E)$.

An expression $E$ evaluates to a pair $(\mathfrak{A}, x)$ of an automaton $\mathfrak{A}$ and a variable $x$ where $\mathfrak{A}$ accepts $(v_1, \ldots, v_n, x)$ such that $\Free(E) = [x_1, \ldots, x_n]$ and $E[v_1/x_1, \ldots, v_n/x_n] = x$.

Note that many rules, like \textsc{Add} or \textsc{Equal} may need to evaluate subexpressions.
However, while evaluating a subexpression $e$, it may be that we generate fresh variables to store the result, which must be projected out.
The only case in which this does not occur is when the subexpression is itself a variable.

\begin{mathpar}
\inferrule*[right=Var]{ }{ x \evaluates (\top, x) }

\inferrule*[right=Add]{ 
    a \evaluates (\mathfrak{A}, x)
    \\
    b \evaluates (\mathfrak{B}, y)
    \\
    V = \{ v : (e, v) \in \{ (a,x), (b,y) \}, e \neq v \}
    \\
    (x + y = z) \evaluates \mathfrak{C}
} { a + b \evaluates (\proj_V(\mathfrak{A} \land \mathfrak{B} \land \mathfrak{C}), z) }
\twhere z ~ \text{is fresh}

\inferrule*[right=Sub]{ 
    a \evaluates (\mathfrak{A}, x)
    \\
    b \evaluates (\mathfrak{B}, y)
    \\
    V = \{ v : (e, v) \in \{ (a,x), (b,y) \}, e \neq v \}
    \\
    (z + y = x) \evaluates \mathfrak{C}
} { a - b \evaluates (\proj_V(\mathfrak{A} \land \mathfrak{B} \land \mathfrak{C}), z) }
\twhere z ~ \text{is fresh}

\inferrule*[right=Zero]{ }{ 0 \evaluates (\texttt{zero}(x), x) }
\twhere x ~ \text{is fresh}

\inferrule*[right=One]{ }{ 1 \evaluates (\texttt{one}(x), x) }
\twhere x ~ \text{is fresh}

\inferrule*[right=Int]{
    \overbrace{1 + 1 + \cdots + 1}^{n~\text{times}} \evaluates (\mathfrak{A}, x)
}{ n \evaluates (\mathfrak{A}, x) }
\twhere x ~ \text{is fresh}

\inferrule*[right=Func]{
    f(e_1, \ldots, e_n, x) \evaluates \mathfrak{A}
}{ f(e_1, \ldots, e_n) \evaluates (\mathfrak{A}, x) }
\twhere x ~ \text{is fresh}

\inferrule*[right=Equal]{ 
    a \evaluates (\mathfrak{A}, x)
    \and
    b \evaluates (\mathfrak{B}, y)
    \\
    V = \{ v : (e, v) \in \{ (a,x), (b,y) \}, e \neq v \}
    \\
    (x \equiv y) \evaluates \mathfrak{C}
} { a = b \evaluates \proj_V(\mathfrak{A} \land \mathfrak{B} \land \mathfrak{C}) }

\inferrule*[right=Less]{ 
    a \evaluates (\mathfrak{A}, x)
    \and
    b \evaluates (\mathfrak{B}, y)
    \\
    V = \{ v : (e, v) \in \{ (a,x), (b,y) \}, e \neq v \}
    \\
    (x < y) \evaluates \mathfrak{C}
} { a < b \evaluates \proj_V(\mathfrak{A} \land \mathfrak{B} \land \mathfrak{C}) }

\end{mathpar}

\subsubsection{Existential Quantification}

\begin{mathpar}
\inferrule*[right=Exist]{
    \tau(x) \land P \evaluates (V, \mathcal{A})
}{ \exists x \in \tau. P \evaluates (V \setminus \{x\}, \proj_x(\mathcal{A})) }
\end{mathpar}

Define $\proj_x(\mathfrak{A})$ to be $\proj_{\mathcal{V}[x]}(\mathcal{B})$, defined in the next section.

% \reed{Useless crossref atm, this is the next section...}
% For the actual implementation of the projection operation, see Section \ref{sec:impl-projection}.

\subsubsection{Projection}\label{sec:impl-projection}

Let $\mathfrak{A} = (Q, \Delta, \delta, q_0, F)$ be a B\"uchi automaton where $\Delta$ is the set of formulas involving $\land$, $\lor$, and $\lnot$ on a finite set $X$ of APs.
\reed{This is a nonstandard way to represent B\"uchi automata, but it's basically how Spot does it internally, and it's convenient for defining projection and substitution. I guess the alphabet is \emph{really} the set of equivalence classes of formulas modulo $\equiv$ such that $\varphi \equiv \psi$ if and only if $\varphi$ holds whenever $\psi$ does, but not sure that's necessary to write or interesting enough to include.}
We compute $\proj_Y(\mathcal{A}) = (Q, \Delta', \delta', q_0, F)$, where $Y$ is a finite list of APs and $\Delta'$ is the set of formulas on the APs $X \setminus Y$.
The new transition relation $\delta'$ is defined as
\[
    \delta' = \left\{ \left( s, d, \bigvee_{x \in Y} (\varphi[\top / x] \lor \varphi[\bot / x])  \right)  : (s, d, \varphi) \in \Delta' \right\}
\]
where $\varphi[a/x]$ is denotes $\varphi$ with $a$ substituted for $x$.
That is, for every transition from a state $s$ to a state $d$ on symbol $\varphi$, we have a transition from $s$ to $d$ whenever $\varphi$ holds regardless of whether any variable $x \in Y$ is true or false.

\subsubsection{Substitution}\label{sec:impl-substitution}

We define the substitution $\mathcal{A}[y/x]$, replacing $x$ by $y$ in the automaton $\mathcal{A} = (V, \mathcal{B})$ where $\mathcal{B} = (Q, \Delta, \delta, q_0, F)$ is a B\"uchi automaton as in Section~\ref{sec:impl-projection} with variables $X$ and underlying alphabet $\Sigma$.
Let $A = [a_1, \ldots, a_n]$ be the list of APs representing $x$ (i.e., $A = V[x]$), and let $B = [b_1, \ldots, b_n]$ be the list of APs representing $y$, which we assume is ambiently available.
This can be stored globally, and generated when needed if the variable $y$ has never been used before.
If $y$ has never been used before, we generate a new list of $n$ fresh APs.

Define $\mathcal{A}[y/x] := (V', \mathcal{B}')$ where $V' = (V \setminus \{ x \}) \cup \{ y \mapsto B \}$, and $\mathcal{B}' = (Q, \Delta', \delta', q_0, F)$, with the new set of variables $X' = (X \setminus A) \cup B$ and the same underlying alphabet, such that:
\[
    \Delta' = \{ \varphi[b_1/a_1, \ldots, b_n/a_n] : \varphi \in \Delta \}
\]
and
\[
    \delta' = \{ (s, d, \varphi[b_1/a_1, \ldots, b_n/a_n]) : (s, d, \varphi) \in \delta' \}
\]

\subsection{AST to IR Conversion}\label{sec:ast-to-ir}

The following conversions are performed when converting the AST to the IR:

\begin{itemize}
    \item $a \iff b$ becomes $(a \implies b) \land (b \implies a)$
    \item $a \implies b$ becomes $\lnot a \lor b$
    \item $k \cdot x$ becomes $\overbrace{x + \cdots + x}^{k \text{ times}}$
    \item $a \neq b$ becomes $\lnot (a = b)$
    \item $a > b$ becomes $b < a$
    \item $a \geq b$ becomes $b < a \lor a = b$
    \item $-a$ becomes $0 - a$
    \item $a \leq b$ becomes $a < b \lor a = b$
    \item $W[i] = W[j]$ becomes $W(i) \iff W(j)$
    \item $W[i] \neq W[j]$ becomes $\lnot (W[i] = W[j])$
    \item $W[i..j] = W[k..\ell]$ becomes $j + k = i + \ell \land \forall n \in \typ{i}. i + n < j \implies W[i + n] = W[k + n]$
    \item $W[i..j] \neq W[k..\ell]$ becomes $\lnot (W[i..j] = W[k..\ell])$
    \item $\forall x_1 \in \tau_1, \ldots, x_n \in \tau_n. P$ becomes $\lnot (\exists x_1 \in \tau_1, \ldots, x_n \in \tau_n. \lnot P)$
    \item $\exists x_1 \in \tau_1, \ldots, x_n \in \tau_n. P$ becomes $\exists x_1 \in \tau_1. \ldots. \exists x_n \in \tau_n. P$
\end{itemize}

where $\typ{i}$ denotes the type of $i$.
