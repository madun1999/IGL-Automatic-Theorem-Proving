\section{Background}\label{sec:background}

This section contains an informal introduction to words, automata, and the notation that we use.
For precise statements and proof, we refer the reader to Allouche and Shallit \cite{MR1997038} or Khoussainov and Nerode ~\cite{aut_theory}.

Let $\Sigma^*$ denote the set of finite words on the alphabet $\Sigma$, let $\Sigma^+$ denote the set of nonempty finite words on the alphabet $\Sigma$, and let $\Sigma^\omega$ denote the set of $\omega$-words on the alphabet $\Sigma$.

For a word $w$, let $w[i]$ denote the $i$-letter of $w$.
Let $w(i,n)$ denote the length-$n$ factor of $w$ starting at $i$ and ending at $i + n - 1$, that is, $w[i \ldots i + n - 1] = w[i] w[i + 1] \cdots w[i + n - 1]$.
Let $|w|$ denote the \term{length} of $w$.
Let $w^R$ denote the \term{reverse} of $w$, so if $w = w[1] w[2] \cdots w[n]$, then $w^R = w[n] w[n-1] \cdots w[1]$.
In a binary alphabet, $\Sigma = \{0,1\}$, for a symbol $x \in \Sigma$, let $\overline{x}$ denote the \term{complement} of $x$, so $\overline{0} = 1$ and $\overline{1} = 0$.
For a word $w$ over a binary alphabet, let $\overline{w}$ denote the \term{complement} of $w$, that is $\overline{w} = \overline{w[1]} \, \overline{w[2]} \cdots \overline{w[n]}$.
A positive integer $p$ is a \term{period} of a word $w$ if $w[i] = w[i + p]$ for all $i$, and the least period of $w$ is the smallest of period of $w$.
A word is \term{periodic} if it has any period, and \text{aperiodic} if there are no periods.

A \term{finite automaton} is a ``machine'' that accepts some \term{finite word}, or string of letters, such as ``abc.''
We say a finite automaton accepts a language $L(\mathcal{A})$ if and only if every string in the language is accepted by the automaton and every string not in the language is rejected by it.
They can be represented as a \emph{finite} collection of \term{states} and \term{transitions}.

Finite automata have nice closure properties: we can combine them using first-order logic operations, because they are closed under intersection, union, complementation, and projection.
Therefore, there is an algorithm that can decide any statement expressed solely in terms of first order logic operations and properties defined by finite automata.

\begin{theorem}
    Finite automata are closed under intersection, union, and complements.
\end{theorem}

For the quantifiers, $\forall$ and $\exists$, we define automata accepting multiple inputs. 
\begin{definition}
    An automaton $\mathcal{A}$ accepts a pair of words $ (x_1x_2\dots x_n, y_1y_2\dots y_n)$ if it accepts the word $(x_1,y_1)(x_2,y_2)\dots(x_n,y_n)$.
\end{definition}

We can generalize this definition to $n$-tuples in the natural way.

\begin{theorem}
    If $\varphi(x,y)$ is a predicate with two inputs, and automaton $\mathcal{A}$ over an alphabet $\Sigma$ accepts pairs of words $(x, y)$ that makes $\varphi(x, y)$ true then we can construct an automaton which accepts words that makes $\varphi'(y) = \exists x \in \Sigma. \varphi(x,y)$ true.
\end{theorem}

Note that we can express constructs such as $P \implies Q$ as $\lnot P \lor Q$ and $\forall x. P(x)$ as $\lnot (\exists x. \lnot P(x))$.
Therefore, we can combine predicates using first-order quantifiers and construct the automaton associated with the predicate.

\textbf{B\"uchi Automata} are an extension of the standard finite automata to infinite inputs.
A B\"uchi automata $(Q, \Sigma, \delta, q_0, F)$ accepts an infinite word $w \in \Sigma^\omega$ if the run of the automaton on the word $w$ visits an accepting state infinitely many times.
Notably, nondeterministic B\"uchi automata and \textbf{not} equivalent to deterministic B\"uchi automata, and many interesting properties are only expressible via nondeterministic B\"uchi automata.
For that reason, we simply refer to nondeterministic B\"uchi automata as B\"uchi automata, without qualification.
Additionally, when we say ``automata'' without qualification through the rest of the manual, we refer to B\"uchi automata.
Importantly, the languages that B\"uchi automata define are closed under intersection, union, projection, and complementation, and emptiness checking is decidable.
Using B\"uchi automata has a number of advantages over finite automata:
\begin{itemize}
        \item Some problems are more naturally expressed with B\"uchi automata.
            For example, we think of numbers as having infinitely many leading/leading zeros; using B\"uchi automata, we can encode number so that they actually start with infinitely many zeros.
            While existential quantification for Pecan works the same regardless of input type, Walnut needs a special algorithm to handle these leading/tailing zeroes because it uses finite automata.
            
        \item Some problems can \emph{only} be expressed with B\"uchi automata.
            For example, it is clearly impossible to encode real numbers as finite strings because there are uncountably many real numbers but only countably many finite strings (on a finite alphabet).
            However, using B\"uchi automata, we are able to automatically prove properties about real numbers.
        
        \item We can still express properties about finite strings (e.g., by using some symbol as an ``end of input'').
\end{itemize}

There are also disadvantages to using B\"uchi automata: in particular, many of the algorithms implementing the various logical operations on B\"uchi automata are much slower than their finite automata counterparts.
For example, while complementing finite automata is simple, complementing B\"uchi automata requires exponential time.
In practice, however, we can often calculate complements, even of large automata, in reasonable amounts of time.
%\reed{it would probably be good to find the exact time complexity that these things take}
