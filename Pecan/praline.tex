\section{Praline}\label{sec:praline}

Praline is a scripting language intended to be a macro system that makes writing Pecan programs easier.
It is a fairly ``standard'' functional language.
As the usage of Praline is not strictly required and exists only for convenience, we omit a formal definition of it.
The major features of Praline are:

\begin{itemize}
    \item Built-in support for integers, strings, booleans, (linked) lists, and tuples.
    
\begin{pecan}
Display 42^6. // 5489031744
Display "hello " ^ "world". // hello world
Display [1..5]. // [1,2,3,4,5]
Display snd ("test", 1). // 1
\end{pecan}
    
    \item Partial application
    
\begin{pecan}
Define add x y := x + y.
Display map (add 10) [1..10]. // [11,12,13,14,15,16,17,18,19,20]
\end{pecan}
    
    \item Lambda functions:
\begin{pecan}
Display (\i => i^2) 10. // 100
\end{pecan}
    
    \item Pattern matching
\begin{pecan}
Display match [1..10] with 
        case fst :: snd :: _ => fst + snd 
        end. // 3
\end{pecan}
        
    \item Ability to interact easily with Pecan:
\begin{pecan}
Display example natFormat { x > 33 & exists y. x=2*y }. // [(x,48)]
\end{pecan}

\end{itemize}

\subsection{Usage}

One of the main uses of Praline is to automate the process of checking certain sets of properties.
For example, suppose we have a predicate \pecaninline{R(x, y)} which is supposed to represent some partial order on a type \pecaninline{T}, and some equality predicate \pecaninline{EQ(x, y)}.
Ordinarily, we would have to write three predicates, and three assertions, as follows.

\begin{pecan}
R_irrefl() := forall x is T. !R(x, x)
#assert_prop(true, R_irrefl)
R_antisym() := forall x is T. forall y is T. if R(x, y) & R(y, x) then EQ(x, y)
#assert_prop(true, R_antisym)
R_trans() := forall x is T. forall y is T. forall z is T. 
              if R(x, y) & R(y, z) then R(x, z)
#assert_prop(true, R_trans)
\end{pecan}

However, using Praline, we can write a macro that generates these definitions for us, given \pecaninline{R}, \pecaninline{T}, and \pecaninline{EQ}.
Such a macro exists in Pecan's standard library, which we can call as follows:

\begin{pecan}
Execute partialOrderCheck R EQ T.
\end{pecan}

There are three commands for interacting with Praline:

\begin{pecan}
Define function args := body.
\end{pecan}

Defines a new function with the specified arguments and body.

\begin{pecan}
Display term.
\end{pecan}

Evaluates the term and then prints the result to the screen.

\begin{pecan}
Execute term.
\end{pecan}

The same as \pecaninline{Display}, except the result is not printed to the screen.

\subsubsection{Examples}

\paragraph{Factorial}

Praline can also be used as a general purpose programming language, although that is not its main purpose.
For example, the following is the factorial function defined in Praline:

\begin{pecan}
Define factorial n :=
    if n = 0 then 1
    else n * factorial (n - 1).

Display factorial 24. // 620448401733239439360000
\end{pecan}

\paragraph{Integer formatting}

The \pecaninline{Example} directive is often very useful for debugging, but it can be confusing to see raw $\omega$-words, so instead it is convenient to define \textbf{formats}, which can then be used as follows.
There is also an alternate syntax which is somewhat more readable, shown below.

\begin{pecan}
Example (myFormat, { P(x) }).
Example using myFormat of { P(x) }.
\end{pecan}

The following program formats an accepting word (which has been converted to binary) as a natural number.
Natural numbers in Pecan are of the form $x0^w$ where $x$ is some binary string, because natural numbers are finite.

\reed{Make sure these examples are up to date}
\begin{pecan}
Define fromBinary := foldr (\a b => a + 2*b) 0.
Define natFormat var reps :=
    let [(prefix, cycle)] = reps in
        if cycle = [0] then
            (var, if isEmpty prefix then "0" else toString (fromBinary prefix))
        else
            (var, omegaWord prefix cycle ^ " (NOT A VALID NATURAL NUMBER)").
\end{pecan}

Note that \pecaninline{stdFormat} formats the accepting word as an $\omega$-word.

\paragraph{Graphing}

The following program can be used to obtain a subset of the graph of some function defined in Pecan.

\begin{pecan}
Define graphPoint format func x :=
    let y be { func(x, y) } in
    (x, lookup "y" (example format y)).
Define graph format F := map (graphPoint format F).

Restrict x, y are nat.
double_f(x, y) := 2*x = y
Display graph natFormat double_f [1,2,3,4]. // [(1,2),(2,4),(3,6),(4,8)]
\end{pecan}

\subsection{Features}

\subsubsection{Directives}

Praline has three primitive directives: \pecaninline{Alias}, \pecaninline{Execute}, and \pecaninline{Define}.

\pecaninline{Alias} allows defining \textbf{new} directives which are simply abbreviations for other directives, possibly even other aliases. 
See Section~\ref{sec:aliases} for more details.

\subsubsection{Aliases}

The aliases defined in the standard library are:

\begin{pecan}
Alias "Display" ==> Execute print .
Alias "Example" ==> Display uncurry example .
Alias "Theorem" ==> Execute uncurry theoremCheck .
\end{pecan}

To use these (as we have already in some of the above examples), we may write the following:
\begin{pecan}
Display factorial 24.
Example (natFormat, { x > 10 }) .
\end{pecan}
These rewrite into:
\begin{pecan}
Execute print (factorial 24).
Execute print (uncurry example (natFormat, {x > 10 })).
\end{pecan}

It is generally preferable to use \pecaninline{Theorem} as opposed to \pecaninline{#assert_prop}, because it allows for more descriptive names, like the following:
\begin{pecan}
Theorem ("Addition of natural numbers is commutative.", {
    $\color{red} \forall$ x, y are nat. x + y = y + x
}).
\end{pecan}

\subsection{Builtins}

\begin{pecan}
check pecanTerm
\end{pecan}

Returns \pecaninline{true} if the input term is always true, and \pecaninline{false} otherwise.

\begin{pecan}
toString term
\end{pecan}

Converts the evaluated term to a string.

\begin{pecan}
print term
\end{pecan}

Prints the term to the screen, and returns \pecaninline{true}.

\begin{pecan}
emit pecanTerm
\end{pecan}

Outputs the specified Pecan term as a definition in the current location in the file and returns \pecaninline{true}.

\begin{pecan}
freshVar
\end{pecan}

Returns a string representing a fresh (i.e., unused variable name).

\begin{pecan}
acceptingWord pecanTerm
\end{pecan}

Returns an associative list mapping variable names to \emph{accepting words}, a pair of a \emph{prefix} and a \emph{cycle}.
For example, the accepting word \pecaninline{([true,false,true], [false])} represents the $\omega$-word $1010^\omega$.

\begin{pecan}
toChars str
\end{pecan}

Converts \pecaninline{str} into a list of strings representing the characters in the string.
For example, \pecaninline{toChars "blah" = ["b", "l", "a", "h"]}.

\begin{pecan}
split sep str
\end{pecan}

\begin{pecan}
cons head tail
\end{pecan}

Constructs a list starting with \pecaninline{head} and ending with \pecaninline{tail}.

\begin{pecan}
compare a b
\end{pecan}

Compares \pecaninline{a} and \pecaninline{b} if they are of the same type, giving an error otherwise.
If \pecaninline{a < b}, then returns -1; if \pecaninline{a = b}, then returns 0; if \pecaninline{a > b}, returns 1.

\begin{pecan}
mkAutomaton inputNames inputBases
\end{pecan}

Creates a new automaton, in the internal Praline format, with the given list of \pecaninline{inputNames} and \pecaninline{inputBases}.
If \pecaninline{inputNames = [x1,x2,...,xn]} and \pecaninline{inputBases = [b1,b2,...,bn]}, then input \pecaninline{xi} will be in base \pecaninline{bi}.

Note that to use this automaton, you will need to call \pecaninline{buidlAut} to convert it into a Spot automaton.
This is due to Spot only working with binary encoded inputs.
Pecan/Praline, however, support any integer base, so the automaton will be automatically encoded for you.
Unfortunately, this does mean that an encoding step is necessary.

\begin{pecan}
addState aut stateLabel isAccepting
\end{pecan}

\begin{pecan}
addTransition aut src dst syms
\end{pecan}

\begin{pecan}
buildAut aut
\end{pecan}

Converts \pecaninline{aut} from the internal Praline implementation into a Spot automaton.
See \pecaninline{mkAutomaton} for an explanation of why this is necessary.

\begin{pecan}
autToStr aut
\end{pecan}

Converts a Spot automaton to a string in the HOA format.

\begin{pecan}
writeFile path contents
\end{pecan}

Write the string \pecaninline{contents} to the file at the location \pecaninline{path}.
Will create the file if it does not exist, but not the intermediate directories.

\begin{pecan}
readFile path
\end{pecan}

Returns a string containing the contents of the file at the location \pecaninline{path}.

\begin{pecan}
deleteFile path
\end{pecan} 

Deletes the file at the location \pecaninline{path}.
