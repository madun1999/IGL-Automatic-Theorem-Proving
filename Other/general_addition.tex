\documentclass{article}
\usepackage[utf8]{inputenc}
\usepackage{geometry}
\usepackage{amsthm}
\usepackage{amsmath}
\usepackage{amssymb}
\usepackage{hyperref}
\hypersetup{
    colorlinks=true,
    linkcolor=blue,
    filecolor=magenta,      
    urlcolor=blue,
}

\theoremstyle{definition}
\newtheorem{definition}{Definition}[section]
\theoremstyle{remark}
\newtheorem{remark}[definition]{Remark}
\theoremstyle{remark}
\newtheorem{example}[definition]{Example}
\theoremstyle{plain}
\newtheorem{theorem}[definition]{Theorem}

\newcommand{\R}{\mathbb{R}}
\newcommand{\Q}{\mathbb{Q}}
\newcommand{\Z}{\mathbb{Z}}
\newcommand{\N}{\mathbb{N}}

\theoremstyle{definition}
\newtheorem{tldr}[definition]{TLDR}

\usepackage{graphicx}
\graphicspath{ {./images/} }

\newcommand{\reed}[1]{{\color{magenta}\bfseries [#1]}}


\newcommand{\term}[1]{\emph{\textbf{#1}}}
%%Use \term to highlight new terminologies.
\begin{document}

\section{Introduction}

If you read nothing else, read section~\ref{sec:usage}.

It is assumed you are familiar with Ostrowski-$\alpha$ numeration systems.

\reed{Below terminology/notation probably needs improvements}

Let $\alpha = [a_0, a_1, \ldots]$ be the continued fraction representation of some number $\alpha$.
Define $\alpha_{<k}$, the set of \textit{$k$-bounded continued fractions}, by:

\begin{equation*}
    \alpha_{<k} := \{ \alpha : \alpha = [ a_0, a_1, \ldots] \text{ where } \forall a_i, 0 < a_i \leq k\}
\end{equation*}

A \textit{$k$-bounded Ostrowski numeration system} \reed{is there a better term for this?} is an Ostrowski numeration system for some $\alpha \in \alpha_{<k}$.

We are interested in creating automata that can recognize/work in arbitrary bounded Ostrowski numeration systems, that is, one automata defining each of the following operations for every $\alpha \in \alpha_{<k}$ for arbitrary $k$:

\begin{itemize}
    \item addition, $+_{\alpha}$
    \item less than, $<_{\alpha}$
    \item recognition, $V_{\alpha}$
\end{itemize}

\section{Algorithm}

The algorithm for generation is a modified version of the algorithm from the paper ``Ostrowski Numeration Systems, Addition and Finite Automata''.
The algorithms follow exactly their description, except we include the values of the continued fraction, $\alpha$ in each state.

For example, in the unmodified Algorithm 1, we kept track of the values $(v_1,v_2,v_3)$ and $(w_1,w_2,w_3)$ of the input and output of the window, respectively, as well as the index in the repeated part of the continued fraction (and some other bookkeeping information).
We simply extend each state to continue $(u_1,u_2,u_3)$, the corresponding continued fraction values for the window possible.
The transitions are also extended, instead of just containing values $v_4$ and $w_4$, corresponding to the next values in the window for the input and output, we also include $u_4$, the next digit in the continued fraction.

\section{Usage}\label{sec:usage}
\subsection{Automata Generation}
To use the program, clone it from GitHub (\url{https://github.com/ReedOei/OstrowskiAutomata}), and install using \texttt{stack install}.

If you are using Linux (or a Mac, probably), you can follow these directions.
Assuming Walnut is installed in \texttt{\$HOME/Walnut}, you can simply run \texttt{bash general-automata.sh k} to generate and prove the correctness of the general automata for all bounded Ostrowski numerations systems with $k$.

\subsection{Interfacing with Walnut}
\reed{Maybe be a way to improve this interface. Not familiar enough with Walnut to say for sure, but it doesn't seem possible. We could write a simple DSL for writing Walnut queries, however, if it becomes too cumbersome.}

Because we have essentially defined infinitely many numeration systems, we are unable write Walnut queries in the typical syntax.
Instead, we must use special predicates for each operation.
For example, to write $x +_{\alpha} y = z$ for $\alpha \in \alpha_{<k}$, we write:

\texttt{\$recog\_k($\alpha$,x) \& \$recog\_k($\alpha$,y) \& \$recog\_k($\alpha$, z) \& \$add\_k($\alpha$,x,y,z)}.

It is critical that \textbf{every variable} is in same numeration system (that is, the same $\alpha$); otherwise adding/performing other operations makes no sense \reed{tbh, I haven't checked this true, but I'd be surprised if it wasn't}.
While some of the predicates (e.g., addition) do check this internally, it's probably better not to rely on that behavior.

When queries are written like this, we may then quantify over all $k$-bounded Ostrowski numeration systems.

The following operations are defined by the program, in addition to the general $k$-bounded Sturmian word, $C_{<k}$.
Note we are able to define the general $k$-bounded Sturmian word because it's digits depend only on the number of zeroes at the beginning (in least significant digit) of the index in the Ostrowski-$\alpha$ numeration system, so as long as the index is a valid number, we may use the same automatic word automaton for every $k$-bounded numeration system.
Again, take care to ensure all numbers are in the same numeration system, or risk getting incorrect results when using the general Sturmian word.

\begin{itemize}
    \item $x <_{\alpha} y$: \texttt{\$lt\_k(x,y)}
    \item $x =_{\alpha} y$: \texttt{\$eq\_k(x,y)}(\reed{You can probably rely on the default Walnut = for this one}
    \item $V_{\alpha}(x)$: \texttt{\$recog\_k($\alpha$,x)}
    \item $x +_{\alpha} y = z$: \texttt{\$add\_k($\alpha$,x,y,z)}
    \item $0_{\alpha}$: \texttt{Ez \$zero\_k(z)}. \textbf{IMPORTANT}: You need to create a ``new'' 0 for every time you use it--numbers can be various lengths, one zero will always be the same length... \reed{Perhaps we can use something else for this, given $0$ has the same representation in every base}
    \item $1_{\alpha}$: \texttt{Eo \$one\_k($\alpha$,o}. Note we must defined $1$ for every base, because, when the continued fraction begins with a $1$ (i.e., $a_0 = 1$), then $1$ is of the form $010^*$.
    Otherwise, it is of the form $10^*$ (in least significant digit form).
    \item $2 \times_{\alpha} x = y$: \texttt{\$double\_k(a,x,y)} This is equivalent to $x +_{\alpha} x = z$, and is included for convenience only.
    \item $x -_{\alpha} y = z$: \texttt{\$sub\_k(a,x,y,z)} This is also included for convenience only, as it is not strictly necessary, as subtraction can (as this automaton is) be defined in terms of $+_{\alpha}$.
\end{itemize}

Note all constants/multiplication by constants can be achieved through repeated adding \reed{Perhaps there is some way to simplify this, but I suspect it's probably what Walnut does}

\section{TODO}
\subsection{Check the Walnut Queries}
Walnut has verified that the addition automata for $k = 2$ and $k = 3$ are correct.
That is, for every $\alpha_{< k}$, the automata successfully recognize the language $\{(\alpha,a,b,c) : \alpha \in \alpha_{< k}, a +_{\alpha} b = c \}$.
However, it is possible that my queries themselves are wrong, so somebody should double-check them.
The proofs and associated definitions can be found here: \url{https://github.com/ReedOei/OstrowskiAutomata/tree/master/results/proofs}.

\subsection{Perfomance}

Currently, the program is rather inefficient in terms of memory complexity.
While the final automata for each algorithm are quite small, the intermediate automata, especially for Algorithm 1, are huge.
This will need improvement

For example, before minimization and pruning unreachable states, the automaton for Algorithm 1 with $k = 3$ has about $4$ million states (to be exact, $4194305$ states).
After removing unreachable states, there are $831841$ states.
After minimization, there are just $37$, indicating there are definitely some major improvements to be made.

Initial generation takes very little time, but a lot of memory.
Minimization is the longest running of the steps (by far), but it allocates about as much memory as the initial generation step.
Pruning unreachable states is negligible in terms of runtime and memory.

\subsection{Prove Theorems}
I have yet to prove any interesting theorems, other than that addition works, so we should probably work on this at some point.

\subsection{Generalize Further}
If possible, we could generalize even further to arbitrary $\alpha$.
However, we will (probably) need a completely different approach, even if there weren't memory issues.

\end{document}
