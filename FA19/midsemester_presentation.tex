\documentclass[leqno,presentation]{beamer}
\DeclareGraphicsExtensions{.eps,.jpg,.png,.tif}
\usepackage{amssymb, amsmath, pdfpages, amsfonts, calc, times, type1cm, latexsym, xcolor, colortbl, hyperref, bookmark}
\usepackage{graphicx}
\usepackage{tabularx}
\usepackage{multirow}
\graphicspath{ {images/} }

\usepackage[latin1]{inputenc}
\usepackage[english]{babel}
\usetheme{Darmstadt}
\newcommand{\R}{\mathbb{R}}
\newcommand{\N}{\mathbb{N}}
\newcommand{\overbar}[1]{\mkern1.5mu\overline{\mkern-1.5mu#1\mkern-1.5mu}\mkern1.5mu}
\newcommand{\highlight}[1]{
  \addtolength{\fboxrule}{.2ex}
  \begin{block}{}
    \begin{quote}#1
    \end{quote}
  \end{block}
}

\newtheorem*{conjecture}{Conjecture}
\newtheorem*{proposition}{Proposition}

\title{Automatic Theorem Proving}
\author[names]{ Reed Oei, Eric Ma, Abdullah Dean, Tatum Schmidt \\ Christian Schulz (Team Leader) \\ Philipp Hieronymi (Faculty mentor)}

\institute{
  \\[-3ex]
  University of Illinois at Urbana-Champaign
  \\[2ex]
  \includegraphics[width = 0.3\textwidth]{UIUC_logo.png}
  \hspace{.30cm}
  \includegraphics[width = 0.07\textwidth]{igl-logo-small.png}
  \\[3ex]
  Illinois Geometry Lab  \\  Midterm Presentation\\ October 23, 2019\\[2ex] }

\date{}

\begin{document}
% Title slide
\frame{\titlepage}

\section{Theorem Proving}

\begin{frame}{Automatic Theorem Proving}
    \begin{itemize}
        \item An \emph{automated theorem prover} is a program that takes a statement and proves if it is true or false, or gives the condition for the statement to be true.
        
        \item Automated theorem provers have attractive qualities
            \begin{itemize}
                \item Computers are very reliable, and never get tired/bored
                \item Can easily explore of new hypotheses
            \end{itemize}
            
        \item With some restrictions, there are programs that can prove first order logic statements.
    \end{itemize}
\end{frame}

\begin{frame}{Automata}
    \begin{itemize}
        \item A finite automaton is a ``machine'' that accepts some \emph{finite word}, which is finite string of letters. For example string ``abc''.
        \item A finite automaton can be represented as a \textbf{finite} collection of \emph{states} and \emph{transitions}.
        \item For example, in the following automaton, the word ``aac'' is accepted and the words ``ab'' and ``caa'' are rejected.
        \center{\includegraphics[width=.65\textwidth]{images/Automata.png}}
    \end{itemize}
\end{frame}

\begin{frame}{Automata and Automated Theorem Proving}
    \begin{itemize}
        \item Finite automaton can represent a variety of properties of words:
            \begin{itemize}
                \item ``Word $x$ has even length.''
                \item ``Word $x$ is a valid email address.''
            \end{itemize}
            
        \item Automata have nice closure properties that allows us to combine predicates about words using first-order logic operations. 
        \item Example: The automaton for ``Word $x$ is a valid email address and has even length.'' can be constructed from  the  ``Even length'' automaton and ``Valid email address'' automaton.
            
        \item Expressing predicates with automata lets us decide first order logic statements consisting those predicates; we can even decide statements with quantifiers like $\forall$ and $\exists$~!
    \end{itemize}
\end{frame}

\begin{frame}{Walnut}
    \begin{itemize}
        \item Walnut is an automated theorem prover.
        \begin{itemize}
        \item Utilizes the closure property of automata.
        \item Input: A first order logic statement of some predicates about finite words, and automata for these predicates.
        \item Output: The condition for the statement to be true, or if the statement is always true or false.
        \end{itemize}
        \item 
        \begin{tabular}{|c|c}
        \cline{1-1}
        Inputs:                                   & \multirow{3}{*}{\includegraphics[width=0.5\linewidth]{Walnut.png}}\\\cline{1-1}
        $p := \exists x (A(x,y) \land B(x))$,      &   \\
        Automata for $A$ and $B$                  &   \\ \cline{1-1}
        Output:                                   &   \\ \cline{1-1}
        An automaton accepts        &   \\
         $y$ that makes $p$ true                       &   \\ \cline{1-1}
        \end{tabular}
        
    \end{itemize}
\end{frame}
\section{Sturmian Words}


\begin{frame}
    \frametitle{Sturmian Words}
    
    \begin{itemize}
        \item Imagine hitting a billiard ball at some angle $\theta$
        \item When the ball crosses a vertical line, write a 0; for horizontal lines, write a 1
        \item Each $\theta$ defines a \emph{characteristic Sturmian word}, $C_{\theta}$
        \item $C_{1/\phi} = 0100101001\ldots$
    \end{itemize}
    \includegraphics[height=0.4\textheight]{images/Fibonacci_word_cutting_sequence.png}
\end{frame}

\begin{frame}{Propositions about Sturmian Words}
    \begin{itemize}
        \item Questions about Sturmian words:
            \begin{itemize}
                \item Are they \emph{eventually periodic}?
                \item Are they \emph{balanced} (how similar are different sections of the word)?
                \item Do they contain \emph{palindromes}, and if so, what do the palindromes look like?
                \item Do they contain \emph{squares} (e.g., ``mama'', ``hotshots'')?
                \item Do they contain \emph{antisquares} (e.g., ``010101''), and if so, how many?
                \item And many more...
            \end{itemize}
            
        \item Conclusion: we need an automaton to calculate the digits of Sturmian words
    \end{itemize}
\end{frame}

% Reed: Should we even talk about numerations systems? They're not in our project title anymore :)
\begin{frame}{Numeration systems}
    
    \begin{itemize}
      \item A \emph{numeration system} is a method of writing numbers
      \item Examples include: base 10, binary (base 2), etc.
    \end{itemize}
    
    \center{$\overbar{\ldots \text{dcba}} = \textcolor{red}{1} \cdot \text{a } + \textcolor{red}{10} \cdot \text{b } + \textcolor{red}{100} \cdot \text{c } + \textcolor{red}{1000} \cdot \text{d } + \ldots$}
    
    \begin{itemize}
          \item For our purposes, we use \emph{Ostrowski} numeration systems
    \item Fibonacci numeration is a special case of Ostrowski numeration system that utilizes the Fibonacci sequence (1, 2, 3, 5, 8, 13, ...):
    \end{itemize}
    
    \center{$\overbar{\ldots \text{dcba}} = \textcolor{red}{1} \cdot \text{a } + \textcolor{red}{2} \cdot \text{b } + \textcolor{red}{3} \cdot \text{c } + \textcolor{red}{5} \cdot \text{d } + \ldots$}
  
\end{frame}

% Slide 3
% \begin{frame}
  
%   \frametitle{$\sqrt{2}$-Ostrowski numeration system}
  
%   \begin{itemize}
%   \item {The $\sqrt{2}$-Ostrowski numeration system utilizes the continued fraction expansion of $\sqrt{2}-1$}
%   \item {Each place in the numeration system is defined recursively as:}
%   \end{itemize}
%   \center{$q_{\text{ -1}} = 0, q_{\text{ 0}} = 1, q_{\text{ k + 1}} = 2q_{\text{ k}} + q_{\text{ k - 1}}$}
  
  
%   \begin{itemize}
%   \item We get $q_{\text{ 0}} = 1, q_{\text{ 1}} = 2, q_{\text{ 2}} = 5, q_{\text{ 3}} = 12, \ldots$
%   \item E.g. $21 = \textcolor{red}{12} \cdot 1 + \textcolor{red}{5} \cdot 1 + \textcolor{red}{2} \cdot 2 + \textcolor{red}{1} \cdot 0 = (1120)_{\sqrt{2}}$
%   \end{itemize}
  
% \end{frame}

\begin{frame}
    \frametitle{Generating Sturmian Words}
    \begin{itemize}
        \item For each Sturmian word $C_{\alpha}$, there is an automaton that generates the $n$-th digit of the Sturmian word, $C_{\alpha}[n]$
            \begin{enumerate}
                \item Write $n$ in the Ostrowski-$\alpha$ numeration system
                \item Count the number of $0$s at the end of the string
                \item If odd, then $C_{\alpha}[n] = 1$; otherwise $C_{\alpha}[n] = 0$
            \end{enumerate}
            
        \item Because we can do arithmetic (e.g., addition) with numbers in their Ostrowski-$\alpha$ representation, automata can express many properties of Sturmian words
    
        \begin{figure}
            \centering
            \includegraphics[height=0.5\textheight]{images/c2.jpg}
            \caption{An automaton that outputs $C_{\sqrt{2}}$}
            \label{fig:c2}
        \end{figure}
    \end{itemize}
\end{frame}

%         \item To calculate $C_{\sqrt{2}}[N]$:
%             \begin{itemize}
%                 \item Write $N$ in Ostrowski-$\sqrt{2}$
%                 \item Count the number trailing of $0$s, $\#_0(N)$
%                 \item $C_{\sqrt{2}}[N] = \#_0(N) \mod{2}$
%             \end{itemize}
    
%         \item $2 = 10_{\sqrt{2}} \mapsto 1$
%         \item $102 = 110011_{\sqrt{2}} \mapsto 0$
%     \end{itemize}
% \end{frame}

% Slide 5
% Reed: Probably should drop this slide altogether, but I'll leave commented out for now
% \begin{frame}
%   \frametitle{Evaluating an automata example}
%   \begin{tabular}{cl}
%     \begin{tabular}{c}
%       \parbox{0.5\linewidth}{
%           Acceptable input \begin{itemize}
%           \item $30 = 1 + 29$
%           \item $30_{10} = {10001}_{\sqrt{2}}$
%           \item lsd: $1,0,0,0,1$
%           \end{itemize}
%           %vspace{}
%           Rejected input \begin{itemize}
%           \item $30 = 1 + 5 + 2 \times 12$
%           \item $30_{10} = {2101}_{\sqrt{2}}$
%           \item lsd: $1,0,1,2$
%           \end{itemize}
%         }
%     \end{tabular}
%     & \begin{tabular}{l}
%         \includegraphics[width=.45\textwidth]{images/lsd_rt2.png}
%       \end{tabular}
%   \end{tabular}
% \end{frame}

\section{Prior Work}

\begin{frame}{$k$-bounded Sturmian words}
    \begin{itemize}
        \item Last semester, we created automata that decide properties of so-called $k$-bounded Sturmian words: Sturmian words whose slopes' continued fraction has only coefficients bounded by some integer
        
        \item For example, $C_{\sqrt{3}}$ is a $2$-bounded Sturmian word:
        
        \[
            \sqrt{3} = 1 + \cfrac{1}{1 + \cfrac{1}{2 + \cfrac{1}{1 + \cfrac{1}{2 + \cdots}}}}
        \]

    \end{itemize}
\end{frame}

\begin{frame}{Prior Results}
    \begin{itemize}
        \item Last semester, we made several conjectures about various properties of Sturmian words, including:
        
            \begin{conjecture}\label{conj:antisq}
                Every Sturmian word has only finitely many antisquares.
            \end{conjecture}
                        
            \begin{conjecture}\label{conj:grouped}
                Every Sturmian word contains grouped factors.
            \end{conjecture}
        
        \item Using the automata we created and an automated theorem prover called Walnut, we proved statements for all $k$-bounded Sturmian words up to $k = 5$
        
        \item Limitation: Our current approach cannot prove the above for \textbf{all} Sturmian words
    \end{itemize}
\end{frame}

\section{Progress}

\begin{frame}
  \frametitle{Infinite Inputs}
  \begin{itemize}
      \item $\omega$-words are infinite length strings
      \begin{itemize}
          \item Example: $01001000100001...$
          \item Can represent real numbers
      \end{itemize}
      \item There are a couple types of automata that take $\omega$-words with different acceptance conditions
  \end{itemize}
\end{frame}

\begin{frame}{B\"uchi Automata}
    \begin{itemize}
          \item B\"uchi automata take $\omega$-words
          \item Accept sequences if they visit an accepting state infinitely often
          \item Corresponding statements are closed under first order logic operations (ex: and, or, $\forall$, $\exists$)
      \end{itemize}
       \begin{figure}
            \centering
            \includegraphics[height=0.45\textheight]{images/buchi_ex.jpg}
            \caption{B\"uchi automaton accepting all nonzero words}
        \end{figure}
\end{frame}


\begin{frame}{Creating an Automated Theorem Prover: Pecan}
    \frametitle{Creating an Automated Theorem Prover: Pecan}
    
\begin{table}[]
\begin{tabular}{|c|c|c|}
\hline
& \textbf{Walnut}  & \textbf{Pecan} \\ \hline
\multirow{3}{*}{\textbf{Inputs}}       & Predicates about & Predicates about \\

  & finite words    & infinite words \\\cline{2-3}
  & \multicolumn{2}{c|}{A first order statement of the predicates.} \\\hline

\textbf{Works using}                   & Finite automata  & B\"uchi automata   \\\hline
\textbf{Output} & \multicolumn{2}{c|}{An automaton for when the statement is correct. }\\\hline
\end{tabular}
\end{table}
    
\end{frame}

% Reed: Commented out for now, because this is sort of an old result
% \begin{frame}
%   \frametitle{Example Theorems}
%   Let $C_{\sqrt{2}}$ be the Sturmian word with slope $\sqrt{2}-1$.\\
%   \vspace{1em}
% %   \begin{table}[]
%       \begin{tabularx}{\textwidth}{Xl}
%       Theorem: $C_{\sqrt{2}}$ contains only anti-squares of order 1,2,3, and 4. \newline\texttt{\footnotesize?msd\_rt2 Ei Ak k<n => (C2[i+k] != C2[i+k+n])} & \raisebox{-0.9\height}{\includegraphics[width=0.5\linewidth]{theorem11.jpg}}  \\
%       \hline
%           \vspace{0em}Theorem: $C_{\sqrt{2}}$ contains no fifth power. \newline\texttt{\footnotesize?msd\_rt2 ((n> 0) \& (Ei At (t < 4 * n => C2[i+t] = C2[i+n+t])))}& \raisebox{-1.5\height}{\includegraphics[width=0.5\linewidth]{theorem5_1.jpg}}
%       \end{tabularx}
% %   \end{table}
% %   \includegraphics[width=\linewidth]{Walnut.png}
% \end{frame}

\section{Future Work}
\begin{frame}
  \frametitle{Future work}
  \begin{enumerate}
  \item Continue last semester's work on proving statements about \emph{every} Sturmian words using Pecan. For example prove or disprove that every Sturmian word contains grouped factors. 
  
  \item Prove properties about real numbers using Pecan since Pecan accepts infinite words.

  \end{enumerate}
  
\end{frame}


\end{document}